\documentclass[12pt,a4paper]{article}

\usepackage[left=2cm,right=2cm,top=2cm,bottom=2cm]{geometry}
\usepackage{amsmath}
\usepackage{amsthm}
\usepackage{thmtools}
\usepackage{amsfonts}
\usepackage{amssymb}
\usepackage[utf8]{inputenc}
\usepackage[T1]{fontenc}
\usepackage[francais]{babel}
\usepackage{mathtools}   % loads »amsmath
\usepackage[dvipsnames]{xcolor} % Ajout du package xcolor
\usepackage{mdframed} %Pour le cadre du théorème

\setlength{\parindent}{0pt}

%Raccourcies
\newcommand{\N}{\mathbb{N}}
\newcommand{\Z}{\mathbb{Z}}
\newcommand{\Q}{\mathbb{Q}}
\newcommand{\R}{\mathbb{R}}
\newcommand{\C}{\mathbb{C}}
\newcommand{\Can}{(e_1,...,e_n)}
\newcommand{\Mat}{\mathcal{M}_n(\R)}
\newcommand{\Aa}{\mathcal{A}}
\newcommand{\Oo}{\mathcal{O}}
\newcommand{\Ll}{\mathcal{L}}
\newcommand{\Tup}{\bigtriangleup}
\newcommand{\Tdn}{\bigtriangledown}

\definecolor{rltred}{rgb}{0.75,0,0}
	\definecolor{rltgreen}{rgb}{0,0.5,0}
	\definecolor{oneblue}{rgb}{0,0,0.75}
	\definecolor{marron}{rgb}{0.64,0.16,0.16}
	\definecolor{forestgreen}{rgb}{0.13,0.54,0.13}
	\definecolor{purple}{rgb}{0.62,0.12,0.94}
	\definecolor{dockerblue}{rgb}{0.11,0.56,0.98}
	\definecolor{freeblue}{rgb}{0.25,0.41,0.88}
	\definecolor{myblue}{rgb}{0,0.2,0.4}


%Style définition
\declaretheoremstyle[
  headfont=\color{blue}\normalfont\bfseries,
  bodyfont=\color{blue}\normalfont,
]{colored}
\declaretheorem[
  style=colored,
  name=Définition,
]{df}

%Style Proposition
\declaretheoremstyle[
  headfont=\color{black}\normalfont\bfseries,
  bodyfont=\color{black}\normalfont,
]{coloredProp}
\declaretheorem[
  style=coloredProp,
  name=Proposition,
]{prop}


%Style Exercice
\declaretheoremstyle[
  headfont=\color{orange}\normalfont\bfseries,
  bodyfont=\color{black}\normalfont,
]{coloredExo}
\declaretheorem[
  style=coloredExo,
  name=Exercice,
]{exo}

% Style Théorème
\newmdtheoremenv[
  linecolor=red,
  frametitlefont=\normalfont\bfseries\color{red},
  frametitlerule=true,
  frametitlerulecolor=red,
  backgroundcolor=blue!10,
]{theo}{Théorème}

% Style Exemple
\newmdtheoremenv[
  linecolor=black,
  frametitlefont=\normalfont\bfseries\color{green},
  frametitlerule=true,
  frametitlerulecolor=green,
  backgroundcolor=green!20,
]{exemple}{Exemple}
%Style Théorème 
%\declaretheoremstyle[
%  headfont=\color{black}\normalfont\bfseries,
%  bodyfont=\color{BrickRed}\normalfont,
%]{coloredTheo}
%\declaretheorem[
%  style=coloredTheo,
%  name=Théorème,
%]{theo}

%Style Corollaire 
\declaretheoremstyle[
  headfont=\color{black}\normalfont\bfseries,
  bodyfont=\color{black}\normalfont,
]{coloredProp}
\declaretheorem[
  style=coloredProp,
  name=Corollaire,
]{coro}


\let\oldproof\proof
\renewcommand{\proof}{\color{olive}\oldproof}

%\theoremstyle{definition}
%\newtheorem{prop}{Proposition}
%\newtheorem{exo}{Exercice}
%\newtheorem{df}{Définition}

% Commande pour ajouter une solution (en vert)
\newcommand{\solution}[1]{\par\noindent\textbf{Solution :} \textcolor{red}{#1}}



\title{Les isométries}
%\author{Floryan Jourdan}
%\date{2023-2024}


\begin{document}
\maketitle

\begin{df}
Soit $E$ un espace vectoriel euclidien de dimension finie. Les isométries de $E$ ne sont autres que les automorphismes orthogonaux de $E$.
\end{df}
\section{Isométries et matrices}
\subsection{Matrices orthogonales}
Attardons nous sur un sujet important : Les matrices orthogonales. Ce sujet prend suite directement après le changement de base orthonormée. On a en effet parlé d'un corollaire intéressant : 
\begin{coro}
Si $P$ est la matrice de passage de la base orthonormée $\Can$ à la base $(u_1,...,u_n)$ de $E$, alors $(u_1,...,u_n)$ est une base orthonormée si et seulement si $^tPP = I_n$.
\end{coro}
Introduisons donc une définition :
\begin{df}
On dit qu'une matrice $A \in \Mat$ est orthogonale si l'on a $^tAA = I_n$.
\end{df}
\begin{prop}
Soit $A$ une matrice orthogonale. Alors $A$ est inversible, $A^{-1}=A^t$ et le determinant de $A$ est égal à $1$ ou à $-1$.
\end{prop}

\begin{prop}
Une matrice $A$ appartenant à $\Mat$ est orthognale si et seulement si les vecteurs colonnes de $A$ forment une base orthonormée de l'espace euclidien usuel $\R^n$.
\end{prop}

\begin{prop}
L'ensemble des matrices orthogonales de $\Mat$, noté $\Oo_n$ est un sous-groupe du groupe $GL_n(\R)$ des matrices inversibles.
L'ensemble des matrices de $\Oo_n$ qui sont de déterminant $1$ est noté $S\Oo_n$, c'est un sous-groupe de $\Oo_n$.
\end{prop}

Ainsi nous savons maintenant que les matrices de passage d'une base orthonormée à une autre base orthonormée sont des matrices orthogonales dont le déterminant vaut $1$ ou $-1$. Nous verrons que cela va déterminer l'orientation de l'espace.

\subsection{Isométries}
Examinons un peu ce que sont les isométries :

\begin{df}
Soit $E$ un espace vectoriel euclidien. Une isométrie de $E$ est un endomorphisme qui préserve le produit scalaire, ie : $\forall x,y \in E$ on a $<f(x)\mid f(y)>~ =~ <x\mid y>$
\end{df}

\begin{prop}
Soit $f \in \Ll(E)$, $f$ est une isométrie si et seulement si  $f$ préserve la norme euclidienne, ie si  et seulement si $\|f(x) \| = \| x \|,~ \forall x \in E$. 
\end{prop}

\begin{coro}
Une isométrie est forcément inversible, en effet, par la proposition précédente on a $\|f(x)\| = 0 \Rightarrow \|f(x)\| = \|x\| = 0$. 
\end{coro}

\begin{prop}
L'ensemble des isométries de $E$ forme un groupe pour la composition appelé \textbf{groupe d'isométries de $E$}, on le note généralement $\Oo(E)$.
\end{prop}

\begin{prop}
Une valeur propre d'une ismétrie de $E$ ne peut être égale qu'à $1$ ou $-1$.
\end{prop}

\begin{proof}

\end{proof}
\begin{theo}
Supposons $E$ de dimension $n \leq 2$. Toute isométrie de $E$ est composée d'au plus $n$ symétries orthogonales de $E$ par rapport à des hyperplans (ces symétries sont des réflexions).
\end{theo}

\section{Composition d'isométries}
\subsection{Orientation}
\begin{prop}
La relation binaire définie sur l'ensemble des bases orthonormées de $E$ par 
\end{prop}

\end{document}
