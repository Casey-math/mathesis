\documentclass[12pt,a4paper]{article}
\usepackage[left=2cm,right=2cm,top=2cm,bottom=2cm]{geometry}
\usepackage{amsmath}
\usepackage{amsthm}
\usepackage{thmtools}
\usepackage{amsfonts}
\usepackage{amssymb}
\usepackage[utf8]{inputenc}
\usepackage[T1]{fontenc}
\usepackage[francais]{babel}
\usepackage{mathtools}   % loads »amsmath
\usepackage[dvipsnames]{xcolor}

\usepackage{calc}
\usepackage[nothm]{thmbox}
\usepackage{pifont}

%Raccourcies
\newcommand{\N}{\mathbb{N}}
\newcommand{\Z}{\mathbb{Z}}
\newcommand{\K}{\mathbb{K}}
\newcommand{\Q}{\mathbb{Q}}
\newcommand{\R}{\mathbb{R}}
\newcommand{\C}{\mathbb{C}}
\newcommand{\Aa}{\mathcal{A}}
\newcommand{\Vv}{\mathcal{V}}
\newcommand{\Mm}{\mathcal{M}}
\newcommand{\Chi}{\mathcal{X}}
\newcommand{\Ll}{\mathcal{L}}
\newcommand{\Tup}{\bigtriangleup}
\newcommand{\Tdn}{\bigtriangledown}

%Vecteur
\renewcommand*{\overrightarrow}[1]{\vbox{\halign{##\cr
  \tiny\rightarrowfill\cr\noalign{\nointerlineskip\vskip1pt}
  $#1\mskip2mu$\cr}}}

\definecolor{rltred}{rgb}{0.75,0,0}
	\definecolor{rltgreen}{rgb}{0,0.5,0}
	\definecolor{oneblue}{rgb}{0,0,0.75}
	\definecolor{marron}{rgb}{0.64,0.16,0.16}
	\definecolor{forestgreen}{rgb}{0.13,0.54,0.13}
	\definecolor{purple}{rgb}{0.62,0.12,0.94}
	\definecolor{dockerblue}{rgb}{0.11,0.56,0.98}
	\definecolor{freeblue}{rgb}{0.25,0.41,0.88}
	\definecolor{myblue}{rgb}{0,0.2,0.4}


%Style définition
\declaretheoremstyle[
  headfont=\color{blue}\normalfont\bfseries,
  bodyfont=\color{blue}\normalfont,
]{colored}
\declaretheorem[
  style=colored,
  name=Définition,
]{df}

%Style Proposition
\declaretheoremstyle[
  headfont=\color{black}\normalfont\bfseries,
  bodyfont=\color{black}\normalfont,
]{coloredProp}
\declaretheorem[
  style=coloredProp,
  name=Proposition,
]{prop}

%Style exo
\declaretheoremstyle[
  headfont=\color{orange}\normalfont\bfseries,
  bodyfont=\color{black}\normalfont,
]{coloredexo}
\declaretheorem[
  style=coloredexo,
  name=Exercice,
]{exo}

%Règle soulignée
\def\regle{\begin{thmbox}[M]{\textbf{Rappel :}}\noindent\color{DarkOrchid}}
\def\endregle{\end{thmbox}}


\let\oldproof\proof
\renewcommand{\proof}{\color{olive}\oldproof}
%\theoremstyle{definition}
%\newtheorem{prop}{Proposition}
%\newtheorem{exo}{Exercice}
%\newtheorem{df}{Définition}
\def\be{\begin{exo}}
\def\ee{\end{exo}}


\def\br{\begin{regle}}
\def\er{\end{regle}}

\def\bd{\begin{df}}
\def\ed{\end{df}}
%%%%%%%%%%%%%%%%%%%%%%%%%%%%%%%%%%%solution%%%%%%%%%%%%%%%%%%
\newcommand{\solution}[1]{\par\noindent\textbf{\color{OliveGreen}Solution :} \textcolor{OliveGreen}{#1}}
%%%%%%%%%%%%%%%%%%%%%%%%%%%%%%%%%%%%%%%%%%%%%%%%%%%%%%%%%%%%%

\title{Ce qu'il faut savoir du cours d'Algèbre Linéaire de D.Thomine}
\author{Floryan Jourdan}
%\date{2023-2024}


\begin{document}
\maketitle
\bd
Soit $E$ un espace vectoriel et $V \subset E$. $\mathcal{V}$ est un sous espace affine de $E$ s'il existe un sous espace vectoriel $V \subset E$ et $a \in E$ tels que $\mathcal{V} = a+V = \{a+v \mid v \in V\}$.
\ed
\br
Une représentation paramétrique de $\Vv$ est une écriture de la forme $\Vv = a + vect(v_1,...,v_k)$ ou $\Vv =  a+\R v_1 + ... + \R v_k$ ou $\Vv = \{a+ t_1 v_1 + ... +t_k v_k \mid t_1,...,t_k \in \R\}$ où $(v_1,...,v_k)$ est libre.
\er
\be
Soit $\Vv$ un sous espace affine de $\R^n$ de direction $V$. \\
Montrer que si $P\in \Vv$, alors $\Vv$ est le sous espace affine passant par $P$ et de direction $V$.
\solution{P158 L1, \color{white} Par hypothèse, il existe un point $A \in \Vv$ tel que $\Vv$ est l'ensemble des points $A+u$ ou $u\in V$...}
\ee
\be
Montrer que si $\Vv$ est un sous espace affine de $\R^n$ alors sa direction est l'ensemble des vecteurs $PQ$, où $P$ et $Q$ sont des points de $\Vv$.
\solution{\color{white} Supposons que $P$ est un point de $\Vv$. Si $Q \in \Vv$, alors d'après l'exo précedent, il existe un vecteur $u \in V$ tel que $Q = P + u$ ; on a donc $u = \overrightarrow{PQ}$ et par suite $\overrightarrow{PQ} \in V$. Réciproquement, si $v\in V$, alors le point $Q = P + v$ appartient à $\Vv$ et l'on a $v = \overrightarrow{PQ}$.}
\ee



\be

\ee



\end{document}
