\documentclass[12pt,a4paper]{article}
\usepackage[left=2cm,right=2cm,top=2cm,bottom=2cm]{geometry}
\usepackage{amsmath}
\usepackage{amsthm}
\usepackage{thmtools}
\usepackage{amsfonts}
\usepackage{amssymb}
\usepackage[utf8]{inputenc}
\usepackage[T1]{fontenc}
\usepackage[francais]{babel}
\usepackage{mathtools}   % loads »amsmath
\usepackage[dvipsnames]{xcolor}
\usepackage{ulem}
\usepackage{enumitem}
\usepackage{pifont}
\usepackage[most]{tcolorbox}
\usepackage[scr=rsfs]{mathalpha}
\usepackage{esvect} %vecteur avec \vv
\usepackage{dsfont}


\usepackage{calc}
\usepackage[nothm]{thmbox}



%Raccourcies
\newcommand{\N}{\mathbb{N}}
\newcommand{\1}{\mathds{1}}
\newcommand{\PP}{\mathds{P}}
\newcommand{\Z}{\mathbb{Z}}
\newcommand{\E}{\mathds{E}}
\newcommand{\K}{\mathbb{K}}
\newcommand{\Q}{\mathbb{Q}}
\newcommand{\R}{\mathbb{R}}
\newcommand{\C}{\mathbb{C}}
\newcommand{\ie}{\textit{i.e }}
\newcommand{\A}{\mathscr{A}}
\newcommand{\Aa}{\mathcal{A}}
\newcommand{\Mm}{\mathcal{M}}
\newcommand{\Chi}{\mathcal{X}}
\newcommand{\Ll}{\mathcal{L}}
\newcommand{\Tup}{\bigtriangleup}
\newcommand{\Tdn}{\bigtriangledown}
\newcommand{\bps}{\langle}
\newcommand{\eps}{\rangle}
\newcommand{\pv}{\wedge}
\newcommand{\doigt}{\ding{43} ~}
\newcommand{\cray}{\ding{46} ~}


% Définir une commande pour la boîte encadrée
\newcommand{\todo}[1]{%
    \begin{tcolorbox}[colback=red!10!white, colframe=red!75!black, title=To-do :]
        #1
    \end{tcolorbox}%
}
%\begin{tcolorbox}[colback=red!10!white, colframe=red!75!black, title=Ma Boîte Encadrée]
%    Ceci est un exemple de texte à l'intérieur de la boîte.
%\end{tcolorbox}





\def\BC{base canonique }
\def\ev{espace vectoriel }
\def\evs{espaces vectoriels }
\def\sevs{sous-espaces vectoriels }
\def\sev{sous-espace vectoriel }
\def\sep{sous-espace propre }
\def\seps{sous-espaces propres }
\def\sea{sous-espace affine }
\def\seas{sous-espaces affines }
\def\AL{application lin\'eaire }
\def\ALs{applications lin\'eaires }
\def\AA{application affine }
\def\AAs{applications affines }
\def\Aff{ \hbox{\it Aff}}
\def\vep{vecteur propre }
\def\vap{valeur propre }
\def\ssi{si et seulement si }

\newcommand{\dessous}[2]{\underset{#1}{#2}}

\def\bar#1{\overline{#1}}
\def\e#1{{\hbox{e}^{#1}}}
\def\mat#1{\begin{pmatrix}#1\end{pmatrix}}
\def\vect#1{\overrightarrow{\kern-1pt#1\kern 2pt}}
\def\card#1{{\hbox{Card}(#1)}}
\def\bin(#1,#2){ \left(\!\begin{smallmatrix} #2 \\ #1 \end{smallmatrix}\!\right)}
\def\Aff{ \hbox{\it Aff}}
\def\lims{\,\overline{\lim}\;}
\def\limi{\,\underline{\lim}\;}
\def\vide{\emptyset}
\def\lims{\,\overline{\lim}\;}
\def\limi{\,\underline{\lim}\;}
\def\vide{\emptyset}
\def\vfi{\varphi}
\def\dim#1{\text{dim }(#1)}

\def\ben{\begin{enumerate}}
\def\een{\end{enumerate}}
\def\bi{\begin{itemize}}
\def\ei{\end{itemize}}

\DeclareMathOperator{\Id}{Id}
\DeclareMathOperator{\Tr}{Tr}
\DeclareMathOperator{\rg}{rg}

\definecolor{rltred}{rgb}{0.75,0,0}
	\definecolor{rltgreen}{rgb}{0,0.5,0}
	\definecolor{oneblue}{rgb}{0,0,0.75}
	\definecolor{marron}{rgb}{0.64,0.16,0.16}
	\definecolor{forestgreen}{rgb}{0.13,0.54,0.13}
	\definecolor{purple}{rgb}{0.62,0.12,0.94}
	\definecolor{dockerblue}{rgb}{0.11,0.56,0.98}
	\definecolor{freeblue}{rgb}{0.25,0.41,0.88}
	\definecolor{myblue}{rgb}{0,0.2,0.4}

%Note
\newenvironment{note}{\par\medskip\noindent\begin{tabular}{l|p{\linewidth-8cm}}
\ding{46}& \color{black}}
{\end{tabular}\par\medskip}
\def\bn{\begin{note}}
\def\en{\end{note}}


%Attention
\def\war{\color{red}\medskip\begin{thmbox}[M]{\ding{39}\textbf{Attention :}}\noindent\color{black}}
\def\endwar{\end{thmbox}}
\def\bw{\begin{war}}
\def\ew{\end{war}}


%Style définition
\declaretheoremstyle[
  headfont=\color{RoyalBlue}\normalfont\bfseries,
  bodyfont=\color{NavyBlue}\normalfont,
]{colored}
\declaretheorem[
  style=colored,
  name=Définition,
]{df}
\def\bd{\begin{df}}
\def\ed{\end{df}}

%Style Proposition
\declaretheoremstyle[
  headfont=\color{BrickRed}\normalfont\bfseries,
  bodyfont=\color{black}\normalfont,
]{coloredProp}
\declaretheorem[
  style=coloredProp,
  name=Proposition,
]{prop}
\def\bp{\begin{prop}}
\def\ep{\end{prop}}

%Style exo
\declaretheoremstyle[
  headfont=\color{orange}\normalfont\bfseries,
  bodyfont=\color{black}\normalfont,
]{coloredexo}
\declaretheorem[
  style=coloredexo,
  name=Exercice,
]{exo}
\def\be{\begin{exo}}
\def\ee{\end{exo}}

%Règle soulignée
\def\regle{\color{blue}\begin{thmbox}[M]{\textbf{Règle :}}\noindent\color{blue}}
\def\endregle{\end{thmbox}}
\def\br{\begin{regle}}
\def\er{\end{regle}}

%Démonstration
\let\oldproof\proof
\renewcommand{\proof}{\color{olive}\oldproof}
\def\bpf{\begin{proof}}
\def\epf{\end{proof}}

%\theoremstyle{definition}
%\newtheorem{prop}{Proposition}
%\newtheorem{exo}{Exercice}
%\newtheorem{df}{Définition}

\newcommand{\bpropo}{
    \begin{tcolorbox}[colback=Yellow!10!Yellow, colframe=red!75!black]
    \bp
}
\newcommand{\epropo}{
    \ep
    \end{tcolorbox}}

%Solution
\newcommand{\solution}[1]{\par\noindent\textbf{\color{OliveGreen}Solution :} \textcolor{OliveGreen}{#1}}

%Titre
%\title{Espaces affines, notes   }
\author{$\mathcal{F.J}$}
%\date{2023-2024}


\title{TD3 Algèbre linéaire}

\begin{document}
\maketitle

\begin{center}
\bf{Feuille d'exercices 3 : Espaces euclidiens / Espaces affines}
\end{center}

\todo{Exos : fin 3, 4, 8, 9, fin 11, 12}

\section{Mise en bouche}
\medskip

%%%%%%%%%
%Exo 1
%%%%%%%%%
\be
Dans $\R^3$, soit $p$ le projecteur orthogonal sur $\mathcal{P}$ d'équation $x+2y-2z = 0$ et $u$ le vecteur 
$u = (1,1,1)$.
Déterminer l'image $p(u)$ de $u$. En déduire la distance $d(u,\mathcal{P})$.
\solution{Notons $F$ le plan $\mathcal{P}$, $G$ l'orthogonal de $F$, $p_F$ la projection orthogonale sur $F$ et $p_G$ la projection orthogonale sur l'othogonal de $F$ \ie G. Un schéma rapide nous donne la relation entre $p_F$ et $p_G$ : $ \forall u \in \R^3, ~ p_F(u) = u - p_G(u)$.\\
$F = \{(x,y,z) \in \R^3 \mid x+2y-2z = 0 \} = \{(x,y,z) \in \R^3 \mid \bps (x,y,z)\mid (1,2,-2) \eps = 0 \} = vect((1,2,-2))^\perp$.\\
On a donc $G = F^\perp = \mathcal{P}^\perp = vect((1,2,-2))$, la projection sur $G$ est donc plus facile étant donné que nous n'avons qu'un seul vecteur, ainsi : $p_G((x,y,z)) = \frac{\bps (x,y,z) \mid (1,2,-2) \eps}{\|(1,2,-2) \|^2}(1,2,-2)$. D'où $p_F((1,1,1)) = (1,1,1) - \frac{\bps (1,1,1) \mid (1,2,-2) \eps}{\|(1,2,-2) \|^2}(1,2,-2) = (1,1,1) - \frac{1}{9}(1,2,-2) = \frac{1}{9}(8,7,11)$.\\
On a $d(u,\mathcal{P}) := \|u - p_F(u)\| = \|p_G(u)\| = 1/3$
}
\ee

\medskip

%%%%%%%%%
%Exo 2
%%%%%%%%%
\be
Soit $M = \dfrac{1}{9}\mat{
1&8&-4\\
8&1&4\\
-4&4&7}$
et soit $f$ l'application linéaire associée à $M$ dans la base canonique.
Montrer que $f$ est une isométrie et préciser sa nature.
\solution{On a $^tMM = I_3$ donc $M$ est la matrice d'une isométrie. Calculons le déterminant pour savoir si il s'agit d'une isométrie directe ou indirecte : $\mid M \mid = -1$. On peut aussi comparer le signe du déterminant du bloc $2\times 2$ en bas à gauche et on le compare au signe du coefficent en haut à droite : si c'est de même signe alors le déterminant sera $1$ sinon $-1$, cette astuce ne fonctionne que si l'on sait que c'est la matrice d'une isométrie. Donc $M$ est la matrice d'une symétrie.
}
\ee

\medskip

%%%%%%%%%
%Exo 3
%%%%%%%%%
\be
Pour $A,B \in M_n(\R)$, on définit $\varphi (A,B) = \Tr(^tA B)$.
Montrer que $\varphi$ définit un produit scalaire sur $M_n(\R)$.
\solution{Un produit scalaire est une application bilinéaire, symétrique et définie positive. Commençons par montrer la symétrie :\\
Soit $A,B \in M_n(\R)$ $\varphi (A,B) = \Tr(^tA B)$ et $\varphi (B,A) = \Tr(^tB A)$, mais on a $\varphi (A,B) = \Tr(^t(^tA B))$ \ie $\varphi (A,B) = \Tr(^tB A))$, donc $\varphi$ est symétrique.\\
Nous avons montré la symétrie avant la bilinéarité pour n'avoir à montrer que la linéarité par rapport à une variable, montrons la linéarité à gauche : soit $A,B, C \in M_n(\R), \gamma \in \K,~ \varphi (A,B + \gamma C) = \Tr(^tA (B + \gamma C)) = \Tr(^tA B + ^tA \gamma C) = \Tr(^tA B) + \gamma \Tr(^tA C) = \varphi (A,B) + \gamma \varphi (A,C)$, donc $\varphi$ est linéaire par rapport à sa seconde variable, étant également symétrique, elle est donc bilinéaire.\\
Il ne nous reste plus qu'à montrer que $\vfi$ est définie positive : soit $A \in M_n(\R), ~ \vfi(A,A) = \Tr(^tA A) = a_{i,i}^2, ~  \forall i \in \{1,...,n\}$, donc $\vfi$ est positive, de plus $\Tr(^tA A) = 0 \iff A = 0$ 
}
\ee

\medskip

%%%%%%%%%%%%%%
%exo 4
%%%%%%%%%%%%%%
\be
\ben
\item Dans $\R^2$ muni du produit scalaire usuel,
démontrer que pour tout $u,v\in \R^2$ de norme 1, il existe une unique réflexion $r$ telle que $r(u)=v$. Préciser ses éléments caractéristiques.
\item  Lorsque $u=(\frac 1 2;\frac {\sqrt 3} 2)$ et $v =(-1,0)$,   représenter ces éléments caratéristiques sur un schéma,  puis donner  la matrice associée à  la réflexion dans la base canonique.
\een
\ee

\medskip

%%%%%%%%%%%%%
%exo 5
%%%%%%%%%%%%%
\be
Soit $E$ un $\R$-\ev euclidien de dimension $n$. 
Soient $u,v$ deux vecteurs orthogonaux non nuls de $E$ et de norme 1. On note $F$ le \sev engendré par $u$ et $v$.
Pour tout $x\in E$, on pose $$f(x)=x-<u,x>u-<v,x>v.$$ \ben
\item Montrer que pour tout $x$, $f(x)\in F^\perp$.
    \solution{$ \forall x \in E, f(x)\in F^\perp \iff \forall x \in E, \bps f(x) \mid u\eps = \bps f(x) \mid v\eps = 0$, ($\{u, v\}$ BON de $F$), on a donc $\bps f(x) \mid u\eps = \bps x- \bps u,x \eps u- \bps v,x \eps v \mid u\eps = \bps x \mid u \eps - \bps u \mid x \eps \|u\|^2 - \bps v \mid x \eps \bps v \mid u \eps = \bps x \mid u \eps - \bps u \mid x \eps = 0$, on a le même résultat pour $v$, donc $\forall x \in E, f(x)\in F^\perp$.}
\item Montrer que $f$ est une projection orthogonale et préciser ses caractéristiques géométriques. 
    \solution{Après un rapide calcul on trouve que $\forall x \in E, ~ f(f(x)) = f(x)$ donc $f$ est bien une projection ;\\
    \sout{Montrons maintenant que $f$ est un endomorphisme orthogonal : \ie montrons que $\forall x \in E, \|f(x)\| = \|x\|$, on a $\|f(x)\|^2 = \bps f(x) \mid f(x) \eps = $}\\
    \textcolor{red}{Et non une projection orthogonale n'est pas une isométrie: $f$ n'est pas bijective !! \textit{(conférer la matrice diagonale d'une projection $\ne id$).}}\\
    Reprenons correctement, on a donc montré que $f$ était une projection, il nous reste à montrer que c'est une projection orthogonale ;\\
    On a montré dans $1.$ que pour tout $x$ on avait $f(x) \in F^\perp$ autrement dit l'image de $f$ est $F^\perp$, comme on a $F \oplus F^\perp = E$ on a une projection sur $F^\perp$ parallèlement à $F$, c'est donc une projection orthogonale par définition.
    }
\item Dans $\R^3$, on choisit $u=(1,1,1)$ et $v=(1,-1,1)$. En adaptant le résultat précédent, exprimer à l'aide de $u$ et $v$  la projection orthogonale de mêmes caractéristiques que dans la question précédente.
\solution{Donc dans $\R^3$, $f$ nous donne $p_{F^\perp} = (-x + 2z , -y , -2x - z) $}
\een
\ee

\medskip

%%%%%%%%%%
%Exo 6
%%%%%%%%%%
\be
Soit $y \in \R^n$ tel que $\|y\| = 1$. Pour tout $x$ de $\R^n$, on pose
\[f(x) = x - 2 \langle x,y \rangle y.\]
\ben
\item Montrer que $f$ est une isométrie et préciser sa nature.
    \solution{Pour montrer que $f$ est une isométrie, le cours nous donne une définition et une propriété, soit on montre que $\forall x,y \in E = \R^n, \bps f(x) \mid f(y) \eps = \bps x \mid y \eps$, soit, et c'est ce que nous allons faire ici, on montre que $\forall x \in E, \| f(x) \| = \|x\|$.\\
    Soit donc $x \in E$, on a $\|f(x) \|^2 = \bps f(x) \mid f(x) \eps = \bps x - 2 \bps x,y \eps y \mid x - 2 \bps x,y \eps y \eps = \|x\|^2 + 4 \bps x\mid y \eps^2 -4 \bps x \mid y \eps \bps x \mid y \eps = \|x\|^2$, ainsi, $f$ est bien une isométrie.\\
    Pour sa nature, calculons $f^2$ : $f(f(x)) = x$, ainsi $f$ est une symétrie.\\
    On peut préciser ses caractéristiques : notons $s$ la symétrie, on voit que $s$ est déterminée comme suit : $s = id - 2p_F$, avec $p_F$ la projection orthogonale sur $F = vect(y)$, donc $s$ est la symétrie par rapport à $F^\perp$, c'est donc une réflexion, car $dim(F) = 1$, (symétrie par rapport à un hyperplan).
    }
\item On choisit $n=3$ et $y = \left({\frac 1 {\sqrt 2}, 0, -\frac 1 {\sqrt 2}}\right)$. Existe-t-il une base dans laquelle la matrice de $f$ s'écrit simplement?
\solution{il nous suffit de prendre une base adaptée à la décomposition $F \oplus F^\perp$, avec $F = vect(y)$.}\\ 
Comment obtenir alors la matrice de $f$ dans la base canonique?
\solution{Cherchons la base dans laquelle la matrice de $f$ sera diagonale, on a déjà un vecteur normé pour $F$, cherchons une BON de $F^\perp$, il nous suffit pour cela de chercher un vecteur orthogonal à $y$ :  $v_1 = (0,1,0)$ est bien orthogonal à $y$, pour le dernier on peut prendre $v_2 =  y \wedge v_1 = \left({\frac 1 {\sqrt 2}, 0, \frac 1 {\sqrt 2}}\right)$ qui est bien normé, dans la base $\mathcal{B} = \{v_1, v_2, y\}$, la matrice de $f$ sera diagonale, de plus si on note $\mathcal{P} = P_{\mathcal{C}an, \mathcal{B}}$ la matrice de passage de la base canonique vers $\mathcal{B}$ alors la matrice de $f$ dans la base canonique sera donnée par : $\mathcal{P}D\mathcal{P}^{-1}$.
}
\een
\ee

\medskip
%%%%%%%%%%
%Exo 7
%%%%%%%
\be
Soit $(u,v)$ une famille orthonormée de vecteurs de $\R^n$. Pour tout $x$ de $\R^n$, on pose
\[f(x) = x - \langle x,u+v \rangle u - \langle x,v-u \rangle v.\] %\afaire{(rotation)}. 
%\afaire{(Alternative : symétrie)} \[f(x) = x + \langle x,v-u \rangle u + \langle x,u-v \rangle v.\] 
Montrer que $f$ est une isométrie et préciser sa nature. 
\solution{$f$ est une isométrie \ssi $f$ préserve la norme : après un long calcul fastidieux on trouve bien que $\|f(x)\|^2 = \|x\|^2$ donc $f$ est une isométrie
}\\
Donner sa matrice dans une base bien choisie.
\solution{\textcolor{red}{Je trouve $f^2(x) = x -2(\bps x \mid u \eps u + \bps x \mid v \eps v)$, y-a-t'il une erreur dans l'énoncé ?}}
\ee

\medskip
%%%%%%%%%%
%Exo 8
%%%%%%%%%%
\be
Soit $p$ un projecteur d'un espace vectoriel euclidien $E$. Montrer que $p$ est orthogonal si et seulement si pour tout $x \in E$, $\|p(x)\| \leq \|x\|$.
\solution{Donc déjà $p$ n'est pas une isométrie, ne pas dire que $\|p(x)\| = \|x\|$.\\
\star ~ Montrons \Rightarrow : Supposons $p$ projecteur orthogonal ; donc $p$ est la projection sur un \sev $F$ parallèlement à $F^\perp$, on a la décomposition suivante de l'espace : $E = F \oplus F^\perp$, \ie que $\forall x \in E$ on a $x_1 \in F$ et $x_2 \in F^\perp$ : $x = x_1 + x_2$, mais $p(x) = p(x_1 + x_2) = p(x_1) + p(x_2) = p(x_1) = x_1$ donc en élevant au carré pour utiliser le théorème de Pythagore on a $\|p(x) \|^2 = \| p(x_1) + p(x_2) \|^2 = \|p(x_1) \|^2 = \| x_1 \|^2 \leq \| x \|^2 = \| x_1 + x_2 \|^2 = \|x_1\|^2 + \| x_2 \|^2$. C'est ce que nous voulions montrer.\\
\star ~ Montrons \Leftarrow : supposons que $\forall x \in E$, $\|p(x)\| \leq \|x\|$, montrons que $p$ est un projecteur orthogonal ; \textcolor{orange}{A FINIR !}
}
\ee

\medskip
%%%%%%%%%%
%Exo 9
%%%%%%%%%%
\be
\ben
\item Soit $\alpha \in \R$. Montrer que pour $P, Q \in \R_n[X]$, 
$$\varphi(P,Q) = \sum_{k=0}^n P^{(k)}(\alpha) Q^{(k)} (\alpha)$$
où  $P^{(k)}$ désigne la dérivée $k$-ième de $P$, définit un produit scalaire sur $\R_n[X]$.
\item Montrer qu'il existe une unique base $(P_0, \ldots, P_n)$ orthonormale pour le produit scalaire $\varphi$ telle que chaque $P_i$ soit de degré $i$ et de terme de degré maximal positif.
\item Calculer $P_i^{(k)}(\alpha)$ pour tout $k \in \N$.
\een
\ee

\newpage
\section{Syst\`emes d'\'equations affines}

\medskip
%%%%%%%%%%%
%Exo 10
%%%%%%%%%%%
\be
On note $F$ l'ensemble des  $(x,y,z)\in\R^3$ v\'erifiant le syst\`eme d'\'equations suivant:
$$
\left\{\begin{array}{lrlrlrlrlcl} 
2x &-& y  &+ & 3 z &=& 1,\\
x  &+& 4y& -&6z   &=& -2 .
\end{array}\right.
$$
\begin{enumerate}
  \item Montrer que $F$ est un sous-espace affine de $\R^3$ et préciser sa direction $\vect{F}$. 
Quelle est la nature de $F$? Donner une équation paramétrique de $F$.
\solution{Soit $u = (x,y,z) \in \R^3$, posons $l_1(u) = 2x -y +3z$ et $l_2(u) = x +4y -6y$, on soit que $\forall \alpha \in \R$, $l_1(u) \ne \alpha l_2(u)$, ainsi les deux lignes du système d'équations sont indépendantes, donc le système est surjectif, \ie que le point $(1,-2)$ est atteint, ainsi l'ensemble des solutions du système est un espace affine de dimension : nombre de variable - nombre de lignes indépendantes = 1, la direction du \sea est l'ensemble $\mathcal{S}_0$ des solutions au système d'équation homogène associé : 
$$
\left\{\begin{array}{lrlrlrlrlcl} 
2x &-& y  &+ & 3 z &=& 0,\\
x  &+& 4y& -&6z   &=& 0 .
\end{array}\right.
$$
en utilisant la méthode du pivot de Gauss on trouve que 
$\mathcal{S}_0 =\{(-2, 5, 3)\}$\\
On peut aussi comprendre ce que l'on fait, on veut trouver $(x,y,z)$ tel qu'il soit perpendiculaire à $v_1 = (2,-1,3)$ et à $v_2 = (1,4,-6)$, mais on connaît un tel vecteur, c'est $v_1 \pv v_2$ ! \\
Pour chercher une solution particulière au système initiale on peut poser $z = 0$ \textit{(en effet le système restant est bien surjectif donc on pourra bien trouver une solution)} on a donc :
$$
\left\{\begin{array}{lrlrlrlrlcl} 
2x &-& y &=& 1,\\
x  &+& 4y   &=& -2 
\end{array}\right.
$$
ce qui nous donne finalement : 
$$
\left\{\begin{array}{lrlrlrlrlcl} 
2x &-& y &=& 1,\\
&& 9y   &=& -5 .
\end{array}\right.
$$
On a donc comme équation paramétrique de $F = (\frac{2}{9},-\frac{5}{9}, 0) + vect((-2, 5, 3))$
}
 \item \'Ecrire ce système sous forme matricielle, puis à l'aide d'une application linéaire $f$. Identifier 
 le sous-espace affine $F$ à l'aide de $f$.
 \solution{on a donc $A = \mat{
2&-1&3\\
1&4&-6}$
et $B = \mat{
1\\
-2}$
Telle que le système s'écrit sous la forme $AX = B$ 
On peut considérer l'application $f \in \Ll(\R^3, \R^2)$ telle que $(x,y,z) \mapsto (l_1(x,y,z), l_2(x,y,z))$, on a donc dans ces conditions $F = (\frac{2}{9},-\frac{5}{9}, 0) + ker(f)$
 }
 \item Interpréter le système d'équations à l'aide d'hyperplans.
 \solution{$F = (\frac{2}{9},-\frac{5}{9}, 0) + ker(l_1) \cap ker(l_2)$. \textbf{Rappel :} le noyau d'une forme linéaire non nulle est un hyperplan.}
\end{enumerate}
\ee

\medskip

%%%%%%%%
%Exo 11
%%%%%%%%
\be
Soient $\lambda\in \mathbb R$ un param\`etre. 
\begin{enumerate}
  \item \`A quelles conditions sur $\lambda$ les deux syst\`emes d'\'equations
 $$ \left\{
  \begin{array}{lll}
  -x+\lambda y-3z&=&\lambda-1\\
  x-3y+\lambda z&=&-2
  \end{array}\right.
  \qquad \left\{
  \begin{array}{lll}
  y+z&=&-\lambda+2\\
  \lambda x-2z&=&0
  \end{array}\right.
 $$
  d\'ecrivent-ils des droites affines de $\mathbb R^3$ ?
  \solution{Une droite affine est de dimension 1, donc il faut que les deux lignes des deux systèmes soient indépendantes, pour le premier système, cette condition est remplie quand $\lambda \ne 3$ pour le second système quelque soit la valeur de $\lambda$ les deux lignes sont indépendantes.}
  \item On suppose les conditions du 1. satisfaites. Trouver pour chaque droite son \'equation param\'etrique.
  \solution{Soit donc $\lambda \ne 3$ pour le premier système ; Pour trouver la direction de la droite affines du système 1 on cherche l'ensemble des solutions du système homogène associé : 
   $$ \left\{
  \begin{array}{lll}
  -x+\lambda y-3z&=&0\\
  x-3y+\lambda z&=&0
  \end{array}\right.
  $$
  cela revient a chercher un vecteur $(x,y,z)$ orthogonal aux vecteurs $v_1 = (-1, \lambda, -3)$ et $v_2 = (1, -3, \lambda)$, il nous suffit donc de calculer le produit vectoriel $v_1 \pv v_2 = (\lambda^2 - 9, \lambda - 3, 3- \lambda) = (\lambda +3, 1, -1)$ car $\lambda \ne 3$. Donc $D_1 = (1, 1, 0) + vect(\lambda +3, 1, -1)$.\\
  Pour le système 2 n'importe quelle valeur de $\lambda$ nous donne une droite affine, donc une fois de plus on calcul le produit vectoriel qui nous donne la direction de la droite affine $D_2$ : $(0,1,1) \pv (\lambda, 0, -2) = (-2, \lambda , -\lambda)$, d'où $D_2 = (-2, 2, -\lambda) + vect(-2, \lambda , -\lambda)$.
  }
  \item Etudier selon la valeur de $\lambda$ les positions relatives de ces 2 droites. 
 On pr\'ecisera lorsqu'elles sont parall\`eles,  confondues, s\'ecantes.
 \solution{\star \textbf{Parallélisme :} Deux \sea sont parallèles s'ils ont mêmes directions,  \ie si $(\lambda +3, 1, -1) = \alpha (-2, \lambda , -\lambda)$, ce qui nous amène à résoudre un système qui nous donne les valeurs de $\lambda$ qui conviennent : $\lambda \in \{-1, -2\}$\\
 \star \textbf{Confondues :} Deux droites sont confondues si leurs directions sont égales et si les points de l'une appartiennent également à l'autre, \ie existe-t-il un $t \in \R$ tel que $$\mat{-2 \\ 2 \\ -\lambda} + t \mat{-2 \\ \lambda \\ - \lambda} = \mat{1 \\ 1 \\ 0}$$
 }
\end{enumerate}
\ee

\medskip
%%%%%%%%%%%
%Exo 12
%%%%%%%%%%%
\be
Soient $\alpha,\beta,a,b,c$ des r\'eels. On consid\`ere trois plans $P_{1}$, $P_{2}$ et $P_{3}$ de $\mathbb R^3$, d'\'equations respectives :
$x+2y+\beta z=a$ , $2x+4y=b$ et $\alpha x+(\alpha+1)y=c$.
D\'eterminer, suivant les valeurs de $\alpha,\beta,a,b,c$, la dimension du
sous-espace affine $P_1\cap P_2\cap P_3$ (si cette intersection est
non vide). 
\ee

\medskip
%%%%%%%%%%%%%
%Exo 13
%%%%%%%%%%%%%%
\be
On note $F$ l'ensemble des quintuplets $(x_1,x_2,x_3,x_4,x_5)\in\R^5$ v\'erifiant le syst\`eme d'\'equations affine suivant:
$$
\left\{\begin{array}{lrlrlrlrlcl} 
x_1 &+& x_2 &+ & x_3 &+& x_4 & & &=& 1,\\
x_1 &-& x_2& & &+& x_4 &-& x_5 &=& 1 ,\\
-x_1 &+& 3x_2 &+& x_3 &-& x_4 &+& 2x_5 &=& -1.
\end{array}\right.
$$
Montrer que $F$ est un sous-espace affine de $\R^5$, donner sa dimension, sa direction $\vect{F}$ et une base de celle-ci.
\solution{\textcolor{gray}{\textbf{Rappel} : $F$ est un \sea de $\R^5$ s'il existe un \sev $\vec{F} \subset \R^5$.}\\
Si on note $L_1, L_2, L_3$ les trois lignes du système, on voit que $L_1 = 2L_2 + L_3$, on peut donc éliminer $L_3$ et ne garder que les deux premières, ce qui nous donne le système suivant :
$$\mathcal{S} :
\left\{\begin{array}{lrlrlrlrlcl} 
x_1 &+& x_2 &+ & x_3 &+& x_4 & & &=& 1,\\
x_1 &-& x_2& & &+& x_4 &-& x_5 &=& 1 ,\\
\end{array}\right.
$$
On voit que $L_1$ et $L_2$ sont indépendantes, donc le système est surjectif, donc  $F$ est bien un \sea de dimension $3$. La direction est donnée par l'ensemble de solutions du système homogène associé : 
$$\mathcal{S}_h :
\left\{\begin{array}{lrlrlrlrlcl} 
x_1 &+& x_2 &+ & x_3 &+& x_4 & & &=& 1,\\
x_1 &-& x_2& & &+& x_4 &-& x_5 &=& 1 ,\\
\end{array}\right.
$$
qui nous donne : 
$$
\left\{\begin{array}{lrlrlrlrlcl} 
x_2 &=& -x_1 &- & x_3 &-& x_4,\\
x_5 &=& 2x_1& & &+& x_3 &+& 2x_4,\\
\end{array}\right.
$$
D'où la direction $\vec{F} = vect((1, -1,0,0,2), (0,-1,1,0,1), (0,-1,0,1,2))$
}
\ee

\newpage
\section{Droites et plans dans $\R^{3}$}
\medskip

%%%%%%%%%%ù
%Exo 14
%%%%%%%%%%%
\be
On se place dans $\mathbb R^3$.
\begin{enumerate}
  \item D\'eterminer une \'equation du plan $V$ engendr\'e par les vecteurs $(1,2,1)$ et $(0,1,1)$ et passant par l'origine.
  \solution{On a donc par définition $V = (0,0,0) + vect((1,2,1),(0,1,1))$\\
  On a $\vec{V} = vect((1,2,1),(0,1,1))$, on veut une représentation cartésienne de ce \sev , on veut donc trouver une forme linéaire $l$ telle que $\vec{V} = \{\vec{u} \in \R^3 \mid l(\vec{u}) = 0 \}$, mais le lemme de représentation nous ramène à chercher un vecteur $\vec{w} \in \R^3$ tel que $\vec{V} = \{\vec{u} \in \R^3 \mid \bps \vec{w} \mid \vec{u} \eps = 0 \} = vect(\vec{w})^\perp$, étant en dimension finie, nous avons l'égalité suivante : $\vec{V}^\perp = (vect(\vec{w})^\perp)^\perp = vect(\vec{w})$, nous sommes donc ramené à chercher une base de l'orthogonal de $\vec{V}$. Le vecteur $\vec{w}$ est perpendiculaire à $(1,2,1)$ et à $(0,1,1)$, donc $\vec{w} = (1,2,1) \pv (0,1,1) = (1, -1, 1)$, d'où $\vec{V} = \{\vec{u} \in \R^3 \mid \bps (1, -1, 1) \mid \vec{u} \eps = 0 \} = \{\vec{u} \in \R^3 \mid x-y+z = 0 \}= V$
  }
  \item D\'eterminer une \'equation du plan $V'$ parall\`ele \`a $V$ et passant par le point $(0,0,1)$. Quelle est son \'equation param\'etrique ?
  \solution{On a $V$ et $V'$ parallèle donc $V'$ admet une équation cartésienne de la forme $x - y + z = \alpha, ~ \alpha \in \R$ mais $V'$ passe par $(0,0,1)$ \ie $(0,0,1) \in V'$, \ie $\alpha$ est nécessairement égale à $1$, on a donc $V' = \{\vec{u} \in \R^3 \mid x-y+z = 1 \}$\\
  Pour l'équation paramétrique, c'est simple ; $V$ et $V'$ sont parallèles donc par définition $\vec{V} = \vec{V'} = vect((1,2,1),(0,1,1))$, d'où $V' = (0,0,1) + vect((1,2,1),(0,1,1))$.
  }
  \item Soit $D$ la droite passant par $(1,0,0)$ et dirig\'ee par le vecteur $(1,0,1)$. D\'eterminer les points d'intersection de $V'$ et de $D$.
  \solution{On a une équation paramétrique de $D = (1, 0,0) + vect((1,0,1))$, autrement dit $(x,y,z) \in D \iff (x,y,z) = (1, 0,0) + t(1,0,1)$ ce qui nous donne le système suivant : 
$$
\left\{\begin{array}{lrlrlrlrlcl} 
x &-& 1 &=& t,\\
y & & & =& 0 ,\\
z & & & =& t.
\end{array}\right.
$$
On a donc $D = \{\vec{u} \in \R^3 \mid x- z = 1 \text{ et } y = 0 \}$, l'intersection est donnée par le système suivant :
$$D \cap V' = 
\left\{\begin{array}{lrlrlrlrlcl} 
x &-& & &z&=& 1,\\
y & & & & &=& 0 ,\\
x &-&y&+&z&=& 1.
\end{array}\right.
$$
Le point d'intersection est $(1,0,0)$
}
\end{enumerate}
\ee

\medskip
%%%%%%%
%Exo 15
%%%%%%%%%
\be
D\'eterminer une \'equation de la droite de $\mathbb R^3$ passant par les points $A = (1,1,1)$ et $B = (1,0,2)$.
\solution{$D = A + vect(\vec{AB}) = \{ (x,y,z) \in \R^3 \mid x = 1 \text{ et } y + z = 2\}$
}
\ee

\medskip
%%%%%%%%%%%
%Exo 16
%%%%%%%%%%%%
\be
Dans l'espace $\mathbb R^3$, on consid\`ere le plan $P$ d'\'equation $x+y+z=1$.
\begin{enumerate}
\item D\'eterminer une \'equation de plan $P'$ passant par les points $A = (2,-1,0)$, $B = (0,0,2)$ et $C = (-1,1,2)$.
\solution{Le plan est dirigé par les vecteurs $\vec{AB}$ et $\vec{AC}$, on a donc $P = A + vect(\vec{AB},\vec{AC}) = \{(x,y,z) \in \R^3 \mid 2x+2y+z = 2 \}.$
}
\item D\'eterminer la nature de  $\vect P\cap \vect{P'}$.
\solution{$x+y+z=0$ et $2x+2y+z = 0$ ne sont pas proportionnelles donc les deux plans ne sont pas parallèles, ainsi la dimension de l'intersection de $\vec{P}$ avec $\vec{P'}$ est de dimension $1$, c'est donc une droite vectorielle.
}
\item D\'eduire de la question pr\'ec\'edente que $P\cap P'$ est non vide, et pr\'eciser sa nature. 
\solution{On a $P \cap P'$ donnée par le système suivant : 
$$ 
\left\{\begin{array}{lrlrlrlrlcl} 
x &+&y&+&z&=& 1,\\
2x &+&2y&+&z&=& 2.
\end{array}\right.
$$
On voit que $A \in P \cap P'$ donc $P \cap P' \ne \emptyset$, c'est donc une \sea de dimension 1, c'est donc une droite affine. 
}
\item D\'eterminer les caract\'eristiques g\'eom\'etriques de $P\cap P'$ (point et base de sa direction).
\solution{$P \cap P' = A + vect((-1,1,0))$
} 
\end{enumerate}
\ee

\section{Autres exemples d'espaces affines}
\medskip

%%%%%%%
%Exo 17
%%%%%%%
\be
D\'eterminer parmi les sous-ensembles suivants ceux qui sont des sous-espaces affines de $\mathbb R^3$  et pr\'eciser alors leurs directions et leurs dimensions.
\begin{enumerate}
  \item $V_1=\{(x,y,z)\in\mathbb R^3, x+2y+z=1\}$
  \solution{$V_1$ est un \sea de direction $\vec{V_1}=\{(x,y,z)\in\mathbb R^3, x+2y+z=0\}$, d'où  $\vec{V_1}=\{(x,y,z)\in\mathbb R^3, z=-x-2y \} = vect((1,0,-1),(0,1,-2))$. $V$ est de dimension $1$.
  }
  \item $V_2=\{(x,y,z)\in\mathbb R^3, x+2y+z=1 \text{ et } x=y=0\}$
  \solution{$V_2$ est de dimension $3-1 = 2$, sa direction est donnée par $\vec{V_2}=\{(x,y,z)\in\mathbb R^3, z=0 \} = vect((1,0,0),(0,1,0))$
  }
  \item  $V_3=\{(x,y,z)\in\mathbb R^3, x^2+y^2=1\}$
  \solution{Pas espace affine car $\vec{V_3}=\{(x,y,z)\in\mathbb R^3, x^2+y^2=1\}$ n'est pas un \sev
  }
  \item  $V_4=\{(x,y,z)\in\mathbb R^3, x^2+2xy+y^2=0\}$ 
  \solution{$V_4=\{(x,y,z)\in\mathbb R^3, x^2+2xy+y^2=0\} = \{(x,y,z)\in\mathbb R^3, (x+y)^2=0\}$ \ie c'est l'ensemble des vecteurs $(x,y,z) \in \R^3$ tels que $x+y = 0$ donc $V_4 = vect((1,-1,0), (0,0,1))$, $V_4$ est de dimension $2$.
  }
  \item $V_5=\{(x,y,z)\in\mathbb R^3, x^2+2xy+y^2=1\}$ 
\end{enumerate}
\ee

\medskip
%exo 18
\be
Soit $n\in\mathbb N^*$. On note $E_n$ l'espace vectoriel des fonctions de $\mathbb R$ dans $\mathbb R$ qui sont polynomiales de degr\'e inf\'erieur ou \'egal \`a $n$. Soit 
$F_0=\{f\in E_n,\int_0^1 f(t) d t=0 \}$ et $F_1=\{f\in E_n,\int_0^1 f(t) d t=1 \}$.
\begin{enumerate}
  \item Montrer que $F_0$ est un $\mathbb R$-espace vectoriel.
  \item Montrer que $F_1$ est un espace affine dont l'espace vectoriel sous-jacent est $F_0$. Quelle est la dimension de $F_1$ ?
  \item On suppose $n=4$. Montrer que la partie $V$ de $F_1$ form\'ee des polynômes divisibles par $\left(x-\frac{1}{2}\right)^2$ est un plan affine de $F_1$.
\end{enumerate}
\ee
\medskip
%exo19
\be
Soit $a$ et $b$ deux r\'eels. Montrer que les suites de r\'eels $(u_n)_{n\geq
0}$ v\'erifiant $u_{n+1}=au_n+b$ pour tout $n\geq 0$ est un sous-espace
affine de l'espace vectoriel des suites r\'eelles. Pr\'eciser la dimension de
ce sous-espace affine. 
\ee
\medskip
%exo20
\be
Soit $E$ un \sev de $\R[X]$, on note $F=\{P\in E, P'(0)=1\}$. 
\ben
\item Montrer que $F$ est un sous-espace affine de $E$. 
\item On suppose que $E=\R_5[X]$. D\'eterminer la nature de $F$ ainsi qu'une base de sa direction.
\item On suppose ici que $E=\R[X]$. Montrer que $F$ est un hyperplan affine. 
\een
\ee
\medskip
%exo21
\newpage
\section{Exercices th\'eoriques}
\be
Soit $E$ un espace affine.
\begin{enumerate}
  \item Soit $F$ une partie non vide de $E$. Montrer que $F$ est un sous-espace affine de $E$ si et seulement si toute droite passant par deux points distincts de $F$ est contenue dans $F$.
  \item D\'ecrire le sous-espace affine engendr\'e par deux droites affines non coplanaires dans un espace affine.
  \item Soient $F_1$ et $F_2$ deux sous-espaces affines de $E$. Montrer que $F_1\cup F_2$ est un sous-espace affine de $E$ si et seulement si $F_1\subset F_2$ ou $F_2\subset F_1$.
\end{enumerate}
\ee
\medskip
%exo22
\be
On consid\`ere deux sous-espaces affines $V$ et $W$ d'un espace affine $E$ et on note $T$ le sous-espace affine
   engendr\'e par $V\cup W$.
   \begin{enumerate}
    \item Pour tout $a\in V$ et tout $b\in W$, montrer qu'on a $\overrightarrow{T}=\overrightarrow{V}+\overrightarrow{W}+\mathrm{Vect}(\overrightarrow{ab})$.
    \item Pour tout $a\in V$ et tout $b\in W$, montrer que $V$ rencontre $W$ si et seulement si le vecteur $\overrightarrow{ab}$ est dans
    $\overrightarrow{V}+\overrightarrow{W}$.
    \item En d\'eduire que $\dim T=\dim(\overrightarrow{V} + \overrightarrow{W})+1$ si $V$ ne rencontre pas $W$, et que $\dim T=\dim(\overrightarrow{V}+\overrightarrow{W})$ sinon.
   \end{enumerate}
\ee
\newpage

\section{Transformations affines-D\'efinitions}
\medskip
%exo23
\be
Dans $\R^{2}$, on note $a=(0,0)$, $b=(1,0)$, $c=(1,1)$ et $d=(0,1)$. Repr\'esenter l'image de $abcd$ par les applications affines suivantes : 
\begin{enumerate}
\item l'application $f$ telle que $f(a)=b$ et 
$\mat{3&0\\0&\frac 1 2}$ est la matrice de $\overrightarrow{f}$ dans la base canonique.   
\item l'application $g$ telle que $g(a)=c$ et 
$\mat{0&-1\\1&0}$  est la matrice de $\overrightarrow{g}$ dans la base canonique;
\solution{Par définition d'une application affine on a $\vv{f}(\vv{ab}) = \vv{f(a)f(b)} = f(b) - f(a)$ \ie dans notre cas : $g(x) = g(a) + \vec{g}(\vec{ax})$. \\
Donc dans notre cas on a $g(x) = \mat{1\\1} + \mat{0&-1\\1&0}\mat{x-0\\y-0}$\\
On a donc les images suivantes : $g(b) = (1,2)$, $g(c) = (0,2)$, $g(d) = (0,1)$
}
  \item l'application $h$ telle que $h(a)=d$ et 
$\mat{0&1\\1&0}$ est la matrice de $\overrightarrow{h}$ dans la base canonique. 
\solution{On a $h(x) = h(a) + \vec{h}(\vec{ax})$. \\
Donc dans notre cas on a $h(x) = \mat{1\\1} + \mat{0&1\\1&0}\mat{x-0\\y-0}$\\
On a donc les images suivantes : $h(b) = (0,2)$, $h(c) = (1,2)$, $h(d) = (2,1)$
} 
\item $h$ et $g$ sont-elles \'egales ? Donner une application affine envoyant $g(a), g(b), g(c)$ sur $h(a), h(b), h(c)$. Ecrire la matrice de son application lin\'eraire associ\'ee. Que constate-t-on?
\solution{Les deux applications ne sont clairement pas égales.\\
Une application affine est entièrement déterminée par sont action sur un repère donc on sait qu'il existe une unique application affine $k$ telle que $k(g(a)) = h(a)$, $k(g(b)) = h(b)$, $k(g(c)) = h(c)$.\\
\medskip
Déterminons la matrice : de $\vv{g(a)g(b)}$ = \\
\medskip
$\vv{k}(\vv{g(a)g(b)}) = \vv{k}((0,1)) = \vv{h(a)h(b)} = (0,1)$\\
$\vv{k}(\vv{g(a)g(c)}) = \vv{h(a)h(c)} = (1,1)$\\
Et on a
}
\end{enumerate}
\ee
\medskip
%exo24
\be
Soit $f$ une application affine qui envoie $abcd$ sur $a'b'c'd'$, comme indiqu\'e sur l'une des figures suivantes.
\begin{enumerate}
\item Justifier que $f$ ne d\'efinit  une application affine  que dans un seul des cas repr\'esent\'es.  Montrer qu'elle est alors unique. 
\item $f$ est-elle bijective ?
\item Donner la matrice de l'application lin\'eaire associ\'ee dans la base $(\vect{ab},\vect{ad})$ puis dans la base $(\vect{ab},\vect{ac})$. En d\'eduire l'expression matricielle de $f$ dans le r\'ep\`ere $(a,b,c)$. \end{enumerate} 
\ee
\medskip
%exo25
\be
D\'eterminer toutes les applications affines d'un espace affine de dimension 1.
\ee
\newpage
 \section{Translations-Homoth\'eties}
 %exo26
 \be
 D\'emontrer qu'une application affine qui commute avec toutes les translations est elle-même une translation.
 \ee
 \medskip
 %exo27
 \be
 On d\'efinit quatre points  $a=(1,1)$, $a'=(-2,2)$, $b=(1,3)$ et $b'=(-2,1)$. Montrer qu'il existe une homoth\'etie $h$ transformant $a$ en $a'$ et $b$ en $b'$. Pr\'eciser son centre et son rapport.
 \solution{Supposons que $h$ existe et notons $c$ sont centre et $k$ sont rapport. On a par définition des applications affines $\vect{h}(\vect{ab}) = \vect{a'b'}$, \ie que l'on a $k\vect{ab} = \vect{a'b'}$. De là on a la valeur de $k = -\frac 12$. Posons $c := \mat{x\\y}$ pour $x,y \in \R$ on a donc un système d'équation donné par $h(a) = c + k \vect{ca}$, de là on tire $x = -1$ et $y = \frac 53$. 
}
 \ee
\medskip
%exo28
\be
Soit $f$ une transformation affine du plan. Soient $a$, $b$ et $c$ trois points non align\'es. 
On note $a'$, $b'$ et $c'$ les images respectives de $a$,
$b$ et $c$ par $f$. On suppose que $(a'b')$ est parall\`ele \`a
$(ab)$, $(a'c')$ \`a $(ac)$ et $(b'c')$ \`a $(bc)$. Montrer que $f$ est une
homoth\'etie ou une translation.
\solution{On a donc $\vect{ab} = \lambda_1 \vect{a'b'}$, $\vect{ac} = \lambda_2~\vect{a'c'}$ et $ \vect{bc} = \lambda_3 ~ \vect{b'c'}$. Si on écrit la matrice de $f$ dans la base $\mathcal B = (\vect{ab},\vect{ac})  $, on a $\mat{\lambda_1 & 0 \\ 0 & \lambda_2}$
}
\ee
\medskip
%exo29
\be[\textit{Theor\`eme de Desargues.}]
Soient deux triangles non aplatis $abc$ et $a'b'c'$ sans sommets communs. 
On suppose
que $(ab)$ est parall\`ele \`a $(a'b')$, que $(bc)$ est parall\`ele \`a $(b'c')$ et
que $(ac)$ est parall\`ele \`a $(a'c')$. Montrer que les droites $(aa')$,
$(bb')$ et $(cc')$ sont concourantes ou parall\`eles.
\ee
\medskip
\be
Soit $E$ un espace affine, $a$ et $b$ deux points (non n\'ecessairement distincts) de $E$ et $\lambda, \mu$ deux r\'eels non nuls et diff\'erents de 1. On note $h$ l'homoth\'etie de centre $a$ et de rapport $\lambda$ et $h'$ celle de centre $b$ et de rapport $\mu$. 
\begin{enumerate}
\item On suppose $\lambda\mu=1$. D\'eterminer la nature de $h'\circ h$  et $h\circ h'$.
\item On suppose $\lambda=1/3$ et $\mu=2$. D\'eterminer  $h'\circ h$  et $h\circ h'$.
\end{enumerate}
\ee
\medskip
\be
Montrer que 2 homoth\'eties commutent si et seulement si elles ont le m\^eme centre.
\ee
\medskip
\be
Soient $A=(2,1)$ et $B=(-1,1)$ deux points du plan affine $\R^2$. Déterminer les caractéristiques
de la composée des deux homothéties $h=h_{A,1/2}\circ h_{B,3}$.
{{Solution|contenu=
$\vec h=\frac12\mathrm{id}\circ3\mathrm{id}=\frac32\mathrm{id}$ donc $h$ est une homothétie de rapport $\frac32$. Son centre $C$ est déterminé par :
:$\frac32\overrightarrow{CB}=\overrightarrow{Ch(B)}=\overrightarrow{Ch_{A,1/2}(B)}=\overrightarrow{CA}+\frac12\overrightarrow{AB}=\overrightarrow{CB}+\frac12\overrightarrow{BA}$, soit $\overrightarrow{CB}=\overrightarrow{BA}$.
$C$ est donc le symétrique de $A$ par rapport à $B$. Ou algébriquement : $C=B+\overrightarrow{AB}=(-1,1)+(-3,0)=(-4,1)$.
}}
\ee


\section{Applications affines}
\medskip

\be
Soit $s : \R^3 \to \R^3$ l'application définie par : $s(x,y,z)=(-2y+z-2,-x-y+z-2,-x-2y+2z-2)$. Déterminer la nature de cette application affine.
\ee

\medskip

\be
Soit $A=(2,1)$ et $B=(-1,1)$ deux points du plan affine $\R^2$. Déterminer les caractéristiques de la composée des deux homothéties $h=h_{A,1/2}\circ h_{B,3}$
\solution{$\vect{h_{A, \frac 12}(h_{B, 3})} = \frac 12 \vect{id} 3\vect{id} = \frac 32 \vect{id}$, c'est donc une homothétie de rapport $\frac 32$, cherchons le centre \ie le point fixe : \\
Si on note $h:=h_{M, \frac 32} =h_{A,1/2}\circ h_{B,3}$ = , on cherche donc le point $M$ tel que $h(M) = M$, \ie $h_{A,1/2}\circ h_{B,3}(M) = M$.\\
On pose $M' := h_{B,3}(M)$, comme $h_{B,3}(M) = B + 3\vect{BM}$ on a $\vect{BM'} = 3\vect{BM}$ on réserve l'expression pour plus tard.\\
On a d'autre part $h_{A,1/2}(M') = A + \frac 12 \vect{AM'}$, (on veut $h(M)=M$ donc $h_{A,1/2}(M') = M$), d'où $\vect{AM} = \frac 12 \vect{AM'}$, par la relation de chasles on se ramène à $\vect{AM} = \frac 12 (\vect{AB} + \vect{BM'})$ on remplace $\vect{BM'}$ ce qui nous donne  $\vect{AM} = \frac 12 (\vect{AB} + 3\vect{BM}) = \frac 12 (\vect{AB} + 3(\vect{BA} + \vect{AM}))$, on obtient finalement $\vect{AM} = 2\vect{AB}$ et de là on tire $M = (-4,1)$.
}
\ee

\medskip

%Exo34
\be
Dans $\R^3$, calculez la projection orthogonale du point $A = (5,-5,5)$ sur la droite définie par le système 
$$
\left\{
\begin{array}{lrlrlrlcl} 
-4x &-&7y&=& 178,\\
-8x &-&7y&=& 398.
\end{array}\right.
$$
\solution{Soit $\mathcal D$ la droite définie par le système ci dessus. On veut la projection orthogonale $p$ du point $A$ sur la droite $\mathcal D$.Posons $A' = p(A)$. On sait que $p(A) \in \mathcal D$ et que $\vect{AA'}$ est orthogonale à $\mathcal D$. \\
\doigt Du système ce dessus on tire qu'un point $(x,y,z) \in \R^3$ appartient à la droite $\mathcal D$ si et seulement si $(x,y,z) = (-55,6,z)$, \ie $\mathcal D = (-55,6,0) + \R(0,0,1)$.\\
\doigt En posant $A' := (x',y',z')$, on a $AA' = (x'-5,y'+5,z-5)$ et en exprimant le produit scalaire $\bps AA' \mid (0,0,1) \eps = 0$, on en tire que $z' = 5$. \\
Ainsi, comme $A' \in \mathcal D$, $A'$ s'écrit sous la forme $(-55,6,z)$ avec $z \in \R$ et que $AA' \perp (0,0,1)$ on en tire que $A' = (-55,6,5)$.
}
\ee

\medskip

%Exo35
\be
Dans $\R^3$, calculez la projection orthogonale du point $M = (-1,-5,-10)$ sur la droite déterminée par les points $A = (10,8,1)$ et $B = (11,3,-6)$.
\solution{Un rapide dessin permet de visualiser la situation, si on note $\mathcal D$ la droite et $M'$ le projeté de $M$ sur $\mathcal D$ on voit que $\vect{MM'} \perp \vect{AB}$ et que évidemment $M' \in \mathcal D$, le premier point se traduit par le fait que $\bps \vect{AB} \mid \vect{MM'} \eps = 0$, le second point qu'il existe $t\in \R$ tel que $M' = A + t\vect{AB}$.}
\ee

\medskip

\be
Dans $\R^3$, calculez la projection othogonale du point $M = (5,-5,5)$ sur le plan définie par l'équation $24x+37y+22z=-76$.
\solution{Si on note $\vect{n} = (24,37,22)$ le vecteur normal au plan, et le point $M'$ le projeté du point $M$ sur le plan, on a $\vect{MM'}$ parallèle au vecteur normal $\vect{n}$, autrement dit $\exists t \in \R$ tel que $\vect{MM'} = t\vect{n}$, de plus le point $M' \in \mathcal P$, on trouve $t$ et on remplace pour trouver les coordonnées de $M'$.}
\ee

\medskip

\be
Dans le plan affine $\R^2$ muni du repère cartésien $(O,e_1,e_2)$, on considère la droite $\mathcal D$ d'équation $2x+y-2=0$. Donner l'expression analytique de la symétrie par rapport à $\mathcal D$ de direction $e_1+e_2$.
\solution{Un rapide dessin nous permet une fois de plus de considérer la situation : on voit que le milieu (que l'on appelera $m$) de $ [MM']$ appartient à $ \mathcal P$ et que $\vect{MM'} // (e_1+e_2)$, si on pose $M' = (x,y,z)$ le symétrique de $M = (x,y,z)$ par la symétrie on a $m = \Big(\frac{(x+x')}{2} ,\frac{(y+y')}{2} \Big)$, comme $m$ appartient à la droite, il vérifie l'équation donc on a $2\times \frac{(x+x')}{2} +  \frac{(y+y')}{2} - 2 = 0$, il nous reste à exploiter le fait que $\vect{MM'}// (e_1+e_2)$, information que l'on peut interprété par le fait qu'il existe un $t\in \R$ tel que $\vect{MM'} = t(e_1+e_2)$ }
\ee

\medskip

\be
Soit $s:\R^3\to\R^3$ l'application définie par : $s(x,y,z)=(-2y+z-2,-x-y+z-2,-x-2y+2z-2)$. Déterminer la nature de cette application affine ainsi que ces caractéristiques.
\ee

\medskip

\be
Soit $p:\R^2\to\R^2$ l'application définie par : $p(x,y)=\left(\frac{2x+y}3+2,\frac{2x+y}3-4\right)$\\
Montrer que $p^2 = p$ et déterminer $p$ géométriquement (points fixes, etc.).
\ee

\medskip

\be
Notons $s_A$ la symétrie centrale de centre $A$ et $t_u$ la translation de vecteur $u$, montrer que $s_B\circ s_A=t_{2\overrightarrow{AB}}$, en déduire que pour tous $A,B,C,D \in \mathcal E$, $ABCD$ est un parallélogramme si et seulement si $s_D\circ s_C\circ s_B\circ s_A=\mathrm{id}_{\mathcal E}$.
\ee

\medskip

\be %Exo 41
Identifier l'application affine $f$ du plan qui envoie respectivement les points $A = (1,0)$, $B = (2,-1)$ et $C = (1,1)$ sur les points $A' = (1,-1)$, $B' = (-1,-3)$ et $C' = (3,-1)$
\solution{
}
\ee

\medskip

\be %Exo42
On considère une translation $\tau$ et une homothétie h d'un espace affine $E$. Identifier les applications :
\ben
\item $f_1:=\tau\circ h\circ\tau^{-1}$;
\solution{On a $\vect{f_1} = \vect{id} \circ \vect{\lambda id} \circ \vect{id} = \vect{\lambda id}$. Montrons que $\vect{f_1}$ est une homothétie, cherchons le centre : Soit $c$ le centre de $h$ \ie $h(c) = c$, on a $f_1(\tau(c)) = \tau(c)$, ainsi $\tau(c)$ est le centre de l'homothétie $f_1$.
}
\item $f_2:=h^{-1}\circ\tau\circ h$ ;
\solution{On a $f_2 = \frac 1\lambda \vect{id} \circ \vect{id} \circ \lambda \vect{id} = \vect{id}$, cela ressemble à une translation, cherchons le vecteur $\vect{u}$ tel que $f_2(A) = A + \vect{u}$ : On note $A''' := f_2(A)$ d'où $\vect{AA'''} = \vect{u}$, si on avance petit à petit on a en notant $c$ le centre de l'homothétie $h$ et $\vect{v}$ le vecteur de la translation $\tau$ on a : $h(A) = A' = c + \lambda \vect{cA}$ ce qui nous donne $\vect{cA'} = \lambda \vect{cA}$, ensuite on a $\tau(A') = A'' = A' + \vect{v}$ \ie $\vect{A'A''}  = \vect{v}$ et enfin $h^{-1}(A'') = A''' = c + \frac 1\lambda \vect{cA''}$ d'où $\vect{cA'''} = \frac 1\lambda \vect{cA''}$.\\
Par la relation de Chasles on a $\vect{AA'''} = \vect{AA'}+ \vect{A'A''}+\vect{A''A'''} = \vect{Ac}+\vect{cA'}+\vect{v}+\vect{A''c}+\vect{cA'''} = \vect{Ac}+\vect{\lambda cA}+\vect{v}-\vect{\lambda cA'''}+\vect{cA'''}$, on a donc $\vect{AA'''} = (1-\lambda)\vect{Ac} + \vect{v} + (1- \lambda)\vect{cA'''}$ \ie $\vect{AA'''} = (1-\lambda)\vect{AA'''} + \vect{v}$, d'où $\vect{AA'''} = \frac{\vect{v}}{\lambda} = \vect{u}$
}
\item $f_3:=\tau\circ h\circ\tau$.
\een
\ee

\medskip

\be
\textbf{Questions de cours...}
\ben
\item Quelle est la nature de l'ensemble d'équation $\mat{1&1&1\\2&1&0}X = \mat{1\\0}$ ? Retrouver l'application linéaire associée à la matrice, identifiez dans quel espace sont les solutions à ce système, puis donnez les solutions de ce système.
\item Déterminez les matrices associées à l'intersection de deux sous-espaces affines de $\R^3$ d'équation $x+ y+z = 2$ et $x-2y-z = 3$. Donnez une représentation paramétrique de cette intersection.
\item Déterminez une représentation cartésienne de $V = \{(3+t, 2+t,1+2t), t \in \R\}$ et de $W = \{(3s+t-1, 2s+t, s+2t+3), s,t \in R\}$
\item Montrez que $\vect{aa}=0$, puis que $\vect{ab} = - \vect{ba}$.
\item Montrez que s'il existe $a_0 \in E$ pour lequel l'application $b \in E \mapsto \vect{a_0b} \in \vect{E}$ est bijective, alors pour tout $a$, l'application $b \in E  \mapsto \vect{ab} \in \vect{E}$ est encore bijective.
\item Justifiez le fait que le milieu de $(a,b)$ est aussi le milieu de $(b,a)$.
\item Soit $(a,b,c,d)$ un quadruplet, montrer que $\vect{ab} = \vect{dc} \iff \vect{ad} = \vect{bc} \iff$ "le milieu de $(a,c)$ est égal au milieu de $(b,d)$".
\item Une réunion de sous espaces affines est-elle un sous espace affine ? Donnez des exemples.
\item Montrez qu'une application affine est une translation si et seulement si son application linéaire associée est l'identité.
\item Donnez la nature de l'image d'une droite affine. Que peut-on dire si $f$ est bijective ?
\item Montrez que les applications affines préservent les barycentres.
\item Soit $t_{\vect{u}}$ une translation de vecteur $\vect{u} \ne \vect{0}.$ Donnez un exemple de partie invariante mais pas fixe par $t_{\vect{u}}$, et un exemple de partie stable mais pas invariante par $t_{\vect{u}}$.
\item Soit $a,b$ deux points distincts du plan. Donnez une condition sur $a',b'$ pour qu'il existe une homothétie $h$ telle que $h(a) = a'$ et $h(b) = b'$. COnstruire un centre de $h$ dans ce cas.
\item Montrez qu'une transformation affine du plan qui préserve trois directions deux à deux distinctes est une homothétie ou une translation.
\item Expliquez la distinction entre "préserver les directions" et "préserver le parallélisme".
\item Construire des exemples de symétries glissées qui ne sont pas des symétries.
\item Est-ce que une application affine préserve les milieux ? Justifier.
\item Est-ce qu'il existe des sous espaces affines dont l'intersection n'est pas un sous espaces affines ? 
\item Est-ce que $D = \{(2+t,-t,3t+1) \mid t \in \R \}$ est faiblement parallèle à $P = \{(x,y,z) \in \R^3 \mid 2x+2y+z = 1\}$ ?
\item Montrer qu'une application affine $f : E \to F$ est surjective si et seulement si l'application linéaire associée est surjective.
\item Montrer qu'une application affine $f : E \to F$ est injective si et seulement si l'application linéaire associée est injective.
\een
\ee



\newpage
\section{formes quadratiques}
\be[Vrai-Faux]
Répondre par vrai ou faux en justifiant.
\ben
\item Soit $A = (1,0,0), ~ B = (0,1,0), ~ C = (0,0,1)$ tel que $(O,A,B,C)$ forme un repère. Il existe une unique transformation affine qui envoie le triangle $OAB$ sur $ABC$.
\item Si dans $\R^2$, $q$ et $q'$ sont deux formes quadratiques de signe $(1,1)$, alors $q+q'$ est de signe $(1,1)$ aussi.
\item Soit $\varphi$ une forme bilinéaire sur $E$, et $q$ la forme quadratique associée : soit $u \in E$, alors $q(u) = 0 \iff \varphi(u,v) = 0, ~ \forall v \in E$.
\een
\ee

\medskip

\be
Soit $q(x,y) = x^2 -2y^2$ une application de $\R^2$. 
\ben
\item Est-ce que $q$ est une forme quadratique ?
\item Est-ce que le cône isotrope forme un sous espace affine ?
\item Expliciter le cône isotrope.
\een
\ee














\end{document}
