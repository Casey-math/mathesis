\documentclass[11pt,a4paper]{article}
\usepackage{amssymb,amsmath,amsthm,amscd} %\input{epsf}
\textwidth=18cm
\textheight=26cm
\oddsidemargin = -1truecm
\topmargin =-2truecm
\headsep =1truecm
\reversemarginpar
\marginparsep = -2cm
\parindent = 0truecm

\usepackage{xcolor}
\usepackage[utf8]{inputenc}
\usepackage{float}

\usepackage{multicol}

\usepackage{pgf,tikz,tkz-tab}
\usetikzlibrary{arrows}
\usepackage{pgfplots}
% \pgfplotsset{compat=1.15}
\usepackage{mathrsfs}
%\usepackage{psfrag}
%\usepackage{graphicx}

%\usepackage{epsfig}
\usepackage{amsmath}

\newcommand{\N}{\mathbb{N}}
\newcommand{\Z}{\mathbb{Z}}
\newcommand{\Q}{\mathbb{Q}}
\newcommand{\R}{\mathbb{R}}
\newcommand{\C}{\mathbb{C}}
\newcommand{\T}{\mathbb{T}}
\newcommand{\ms}{\medskip}
\newcommand{\vfe}{\vfill\eject}

\newcommand{\A}{\mathcal{A}}
\newcommand{\fou}{\mathcal{F}}
\newcommand{\dist} {\rm dist}

\newcommand{\im}{\mbox{Im}}
\newcommand{\re}{\mbox{Re}}

\newcommand{\abs}[1]{\vert #1 \vert}
\newcommand{\absb}[1]{\bigl\vert #1 \bigr\vert}
\newcommand{\absbig}[1]{\biggl\vert #1 \biggr\vert}
\newcommand{\norm}[1]{\| #1 \|}
\newcommand{\normb}[1]{\bigl\| #1 \bigr\|}
\newcommand{\normbig}[1]{\biggl\| #1 \biggr\|}
\newcommand{\parent}[1]{\bigl(#1\bigr)}
\newcommand{\parentbig}[1]{\biggl(#1\biggr)}
\newcommand{\set}[1]{\bigl\{#1\mathclose{}\bigr\}}
\newcommand{\setbig}[1]{\biggl\{#1\mathclose{}\biggr\}}


\newcommand{\bigo}[1]{O\mathopen{}\left(#1\right)}
\newcommand{\japbrak}[1]{\langle#1\rangle}

\newcommand{\interval}[4]{\mathopen{#1}#2\mathclose{}\mathpunct{},#3\mathclose{#4}}
\newcommand{\intervaloo}[2]{\interval{]}{#1}{#2}{[}}
\newcommand{\intervaloc}[2]{\interval{]}{#1}{#2}{]}}
\newcommand{\intervalco}[2]{\interval{[}{#1}{#2}{[}}
\newcommand{\intervalcc}[2]{\interval{[}{#1}{#2}{]}}


\newcounter{ex}
\setcounter{ex}{0}

% \newcommand{\exo}{\stepcounter{ex} \vspace{0.2cm} \noindent
% {\bf Exercice \arabic{ex}.} $\,$}

% \newtheorem{exo}[ex]{\noindent \textbf{Exercice}}

\newtheorem{exs}[ex]{Exercice}
\renewcommand{\theexs}{}
\newenvironment{exo}{\begin{exs}\rm}{\end{exs}\vspace{.1cm}}

\newtheorem{exos}[ex]{\color{gray} Exercice}
\renewcommand{\theexos}{}
\newenvironment{exosup}{\begin{exos}\rm \color{gray}}{\end{exos}}
 

\usepackage{enumitem}
 \setenumerate{leftmargin=.6cm, topsep=.2cm, itemsep=.1cm}
 \setlist{leftmargin=0cm}

%Solution
\newcommand{\solution}[1]{\par\noindent\textbf{\color{olive}Solution :} \textcolor{olive}{#1}}

\begin{document}


\setcounter{section}{0}


\begin{tabular*}{\textwidth}{l @{\extracolsep{\fill}} cr}
Mathématiques & \textbf{\Large{Feuille de TD 1}} & MEU353\\
Université Paris-Saclay & & 2023-2024
\end{tabular*}

\ms
\begin{center}
\textbf{Rappels de Topologie}
\end{center}
\ms

\begin{exo} \textbf{Convergence dans $\R^2$.}
On consid\`ere $\R^2$, muni de la distance euclidienne. Donc si 
$X = (x,y)$ et $X' = (x',y')$, $\dist(X,X') = \sqrt{|x-x'|^2 + |y-y'|^2}$.

On se donne une suite $\{ X_k \}$ dans $\R^2$, avec $X_k = (x_k,y_k)$ pour $k \geq 0$, et
un point $X = (x,y) \in \R^2$. 

D\'emonter que $\lim_{k \to +\infty} X_k= X$ (dans $\R^2$ muni de la distance euclidienne) si et seulement si
$\lim_{k \to +\infty} x_k = x$ et $\lim_{k \to +\infty} y_k= y$.
\end{exo}


\begin{exo}\textbf{Normes usuelles sur $\R^n$} Rappelons la définition des normes usuelles sur l'espace vectoriel $\R^n$, comme applications de $\R^n$ dans $\R^+$ : pour tout $x=(x_1,\dots,x_n)\in\R^n$,
% \begin{align}
% &d_{\infty}(x,y)=\max\{|x_1-y_1|,\dots,|x_n-y_n|\} ;\\
% &d_1(x,y)=|x_1-y_1|+\dots+|x_n-y_n| ;\\
% &d_2(x,y)=\sqrt{|x_1-y_1|^2+\dots+|x_n-y_n|^2},\\
% \end{align}
\begin{align*}
&\|x\|_{\infty}=\max\{|x_1|,\dots,|x_n|\}, \\
&\|x\|_{1}=|x_1|+\dots+|x_n|,\\
&\|x\|_{2}=\sqrt{|x_1|^2+\dots+|x_n|^2},\\
\end{align*}
%,\ y=(y_1,\dots,y_n)\in\R^n.$
\begin{enumerate}
\item Montrer que $\|\cdot\|_{\infty},\ \|\cdot\|_{1}$ et $\|\cdot\|_{2}$ sont des normes dans $\R^n.$
\item Montrer que ces trois normes sont équivalentes. 
\item Donner les trois distances dans l'ensemble $\R^n$ induites par les normes $\|\cdot\|_{\infty},\ \|\cdot\|_{1}$ 
et $\|\cdot\|_{2}$. On notera ces distances $d_{\infty},\ d_1$ et $d_2$.
\item Dessiner deux points distincts dans $\R^2$ et représenter leur distance au sens de $d_1$, $d_2$ et $d_\infty$. 
\item Dans $\R^2,$ représenter les boules fermées $\overline{B}_{d_1}(0,1)$, $\overline{B}_{d_2}(0,1)$ et $\overline{B}_{d_\infty}(0,1)$. 
\end{enumerate}
\end{exo}
\begin{exo}\textbf{Représentation des formes linéaires.}
Notons $(\cdot|\cdot)$ le produit scalaire euclidien dans $\R^n$. 
\begin{enumerate}
    \item Soit $L:\R^n\longrightarrow\R.$ Montrer que $L$ est linéaire \textit{si et seulement si} il existe un vecteur $v$ dans $\R^n$ tel que pour tout $x\in\R^n$, $L(x)=(v|x)$.
     \item On munit $\R^n$ de la norme euclidienne. Quelle est alors, dans la question ci-dessus, la norme de $L$?
\end{enumerate}
\end{exo}

\begin{exo}\textbf{Exemples concrets.} Considérons les sous-ensembles suivants de $\R^2$ :
\begin{align*}
A&=\{(x,x^3)\,:\,x< 1\},\\
B&=\{(n,\frac{1}{n+1})\,:\,n\in\N\},\\
%C&=\{(x,y)\in\R^2\,:\,x^2+y^2\le 1,\ x\ge0,\ y>0\},\\
C&=\{(x,y)\in\R^2\,:\,y\ge x^2,\ y>x+1\}.
%E&=\{(x,y)\in\R^2\,:\,x\ge y^2-4,\ |x|\le 2\}.
\end{align*}
\begin{enumerate}
    \item Représenter dans le plan chacun de ces ensembles.
    \item Lesquels de ces ensembles sont ferm\'es? ouverts?
    \item Déterminer leur intérieur, leur frontière et leur adhérence. 
    \item Lesquels de ces ensembles sont compacts ?
\end{enumerate}  
\end{exo}

\vfe 

\begin{exo}\textbf{Graphes.} On se donne une fonction $f : \R \to \R$
et on note $G_f \subset \R^2$ le graphe de $f$. 
\begin{enumerate}
    \item Rappeler la d\'efinition de $G_f$.
	    \solution{$G_f = \{(x,f(x)) | x \in \mathcal{D}_f\} \subset \R^2$}
    \item Montrer que $G_f$ est ferm\'e si $f$ est continue. 
	    \solution{Supposons $f$ continue, montrons que $G_f$ est fermé, utilisons la caractérisation séquentielle des fermés : Soit donc $(x_n,y_n)$ une suite de $G_f$ qui converge vers $(x,y)\in \R^2$, montrons que $(x,y)\in G_f$.\\
		On a donc $(x_n,y_n)_n \in G_f,~ \forall n \in \N$, on a donc $y_n = f(x_n)$, donc $f(x_n) \to y$ quand $n \to + \infty$ et comme $f$ est continue, $f(x_n) \to f(x)$ quand $n \to \infty$, donc par unicité de a limite on a $f(x) = y \Rightarrow (x,y) \in G_f$.
		}
    \item La r\'eciproque est-elle vraie?
	    \solution{faux, pourquoi ??}
\end{enumerate}  
\end{exo}


\begin{exo}\textbf{Limites.} Soient $f:\R^2\longrightarrow\R$ la fonction définie par
\begin{equation*}
f(x,y)=
\begin{cases}
\sin(\frac{y^2}{x}),&\ x\neq0,\\
0,&\ x=0,
\end{cases}    
\end{equation*}
\begin{enumerate}
\item Montrer que $f$ admet la même limite selon toutes les directions en $(0,0)$ mais que $f$ n'est pas continue en $(0,0)$. 
\item Montrer que les fonctions suivantes, notées $g$ et $h$, sont continues au point $(0,0)$.
\begin{equation*}
g(x,y)=
\begin{cases}
\frac{x|y|}{\sqrt{x^2+y^2}},&\ (x,y)\neq(0,0),\\
0,&\ (x,y)=(0,0),
\end{cases}
\quad 
h(x,y)=
\begin{cases}
(x-y)\frac{x^2}{x^2+y^2},&\ (x,y)\neq(0,0),\\
0,&\ (x,y)=(0,0).
\end{cases}
\end{equation*}
\end{enumerate}
\end{exo}


\begin{exo} \textbf{Exemples de distances}\\
\begin{enumerate}
\item Rappeler la d\'efinition d'une norme et d'une distance. 
V\'erifier que toute norme sur un espace vectoriel $E$ d\'efinit une distance
sur $E \times E$. La suite donne des exemples de distances qui ne viennent pas 
directement de normes.
    \item V\'erifier que les applications suivantes sont bien des distances.
    \begin{align*}
    d_1&:\ (x,y)\in\R\times\R\mapsto \min(1,\abs{x-y})\,,\\
    d_2&:\ (x,y)\in\R\times\R\mapsto\sqrt{\abs{x-y}}
    \  \text{ (et que se passse-t-il avec $(x,y)\in\R\times\R\mapsto \abs{x-y}^2$?)}
    \,,\\
    d_3&:\ ((x_1,x_2),(y_1,y_2))\in\R^2\times\R^2 \mapsto \sqrt{\abs{x_1-y_1}}+\min(1,\abs{x_2-y_2})\,. 
    \end{align*}
    \item Vérifier que les ouverts pour la distance $d_1$ et $d_2$ (resp. $d_3$) sont les mêmes que les ouverts pour la distance usuelle sur $\R$ (resp. $\R^2$).
\end{enumerate}
\end{exo}



\begin{exo} \textbf{Produit d'espaces métriques I}\\

Soient $(E_1,d_1)$ et $(E_2,d_2)$ deux espaces métriques. 
\begin{enumerate}
\item Proposer (au moins) une distance $d$ sur l'espace produit $E_1\times E_2$.
	\solution{On a par exemple pour $(x_1,x_2),(y_1,y_2) \in (E_1,d_1)\times(E_2,d_2)$ la distance définie comme le max des distance sur chacuns des espaces métriques : $\max \big( d_1(x_1,y_1), d_2(x_2,y_2) \big)$, on peut faire de même pour les deux autres distances classiques.}
\item Montrer que pour toute suite $(x_k,y_k)\in (E_1\times E_2)^\N$, et pour tout $(x,y)\in E_1\times E_2$,
\[
(x_k,y_k) \underset{k\to\infty}{\longrightarrow} (x,y)\  \text{dans}\ (E_1\times E_2, d)
\quad \textit{ssi}\quad 
x_k \underset{k\to\infty}{\longrightarrow} x\  \text{dans}\ (E_1,d_1)\,,\ \text{et}\ y_k \underset{k\to\infty}{\longrightarrow} y\  \text{dans}\ (E_2,d_2)\,.
\]
		\solution{On peut raisonner ici par équivalence en prenant la norme 1 définie sur l'ensemble produit.}
\end{enumerate}
\end{exo}

\begin{exo}\textbf{Fonction continue dans un espace métrique}\\

Soit $f: (E_1,d_1)\to (E_2,d_2)$, et $x\in E_1$. 
\begin{enumerate}
    \item Montrer que $f$ est continue au point $x$ \textit{ssi} pour tout $\epsilon>0$, il existe $\delta>0$ tel que 
\[
f\parent{B_1(x,\delta)}\subset B_2\parent{f(x),\epsilon}\,.
\]
\item Montrer que $f$ est continue au point $x$ \textit{ssi} pour toute suite 
$\{ x_k \}$ dans $E_1$ qui converge vers $x$ dans $E_1$, la suite $\{ f(x_k) \}$ converge vers $f(x)$ dans $E_2$.

\end{enumerate}
\end{exo}

\begin{exo}\textbf{Équivalence des normes dans $\R^n$}\\

Soit $N: \R^n\to \R^+$ une norme, et $\norm{\ }$ la norme Euclidienne sur $\R^n$, définie on le rappelle par 
\[
\norm{x} = \sqrt{\sum_{i=1}^n x_i^2}\,,\quad \text{pour}\ x=(x_i)_{1\leq i\leq n}\,.
\]
\begin{enumerate}
\item Montrer qu'il existe une constante $M>0$ telle que pour tout $x\in\R^n$,
\[
N(x)\leq M\norm{x}\,.
\]
\item Montrer alors que $N: \R^n\longrightarrow \R_+$ est une fonction continue. \textit{Remarque: On déduit de cette question que toute norme sur $\R^n$ est une application continue}.
\end{enumerate}
 Soit maintenant $S=\set{x\in\R^n\mid \norm{x}=1}$.
\begin{enumerate}[resume]
    \item Montrer qu'il existe $m>0$ tel que $m\leq N(x)$ pour tout $x\in S$.
    \item En déduire que pour tout $x\in\R^n$,
    \[
    m\norm{x}\leq N(x)\,.
    \]
    \item Soient $N$ et $N'$ deux normes sur $\R^n$. Montrer que $N$ et $N'$ sont \'equivalentes.
\end{enumerate}
\end{exo}

\begin{exo}\textbf{Caractérisation des ouverts de $\R$.}
Munissons $\R$ de sa topologie la plus naturelle, \`a savoir celle qui vient de la valeur absolue $\abs{\ }$. 
\begin{enumerate}
    \item Justifier le fait que toute réunion dénombrable d'intervalles ouverts est encore un ouvert de $\R$.
    \item Réciproquement, on va montrer que que tout ouvert de $\R$ peut s'écrire comme une réunion dénombrable d'intervalles ouverts. Considérons pour cela $\mathcal{U}$ un ouvert de $\R$.
    \begin{enumerate}
        \item Montrer que $\mathcal{U}$ est une réunion (pas forcément dénombrable) d'intervalles ouverts.
        \item Utiliser la densité de $\Q$ dans $\R$ pour avoir une réunion dénombrable. 
 Indication: On suppose $U \neq \R$. Pour $y\in \Q \cap U$, consid\'erer l'intervalle $I_y = ]y-d(y)/2, y+d(y)/2$, o\`u $d(y) = \rm{dist}(y, \R \setminus U)$.
      
    \end{enumerate}
\end{enumerate}
\
\end{exo}

\begin{exo}\textit{(Bonus)} \textbf{Produit d'espaces métriques II}\\

Soit $(E_j,d_j)_{j\geq0}$ une famille dénombrable d'espaces métriques. Supposons que pour tout $j\geq0$,
\begin{equation} \label{1}
d_j(x_j,y_j)\leq1\quad \text{lorsque}\ (x_j,y_j)\in E_j\times E_j\,.
\end{equation}
%\[
%d_j(x_j,y_j)\leq1\quad \text{lorsque}\ (x_j,y_j)\in E_j\times E_j\,.
%\]
\begin{enumerate}
    \item Montrer que 
    \[
    d: \parent{(x_j),(y_j)}_{j\geq0}\in  \parent{\prod_{j\geq0}E_j}\times\parent{\prod_{j\geq0}E_j}\longmapsto \sum_{j\geq0}2^{-j}d_j(x_j,y_j)
    \]
est une distance sur $\displaystyle\prod_{j\geq0}E_j$. 
\item Montrer que la suite $\set{(x_j^{(k)})}_k$ converge dans l'espace métrique produit $\parent{\prod_j E_j, d}$ \textit{ssi} chaque suite coordonnée $(x_j^{(k)})_k$ converge dans l'espace $(E_j,d_j)$.
\item
Que faire si l'on n'a pas \eqref{1} d'embl\'ee
\end{enumerate}
\end{exo}

\begin{exo} \textit{(Bonus)} \textbf{Caractérisation des compacts de $\intervaloo{0}{1}$} \\
\begin{enumerate}
    \item Quels sont les sous-ensembles compacts de 
    $[0,1]$? Ceux de $\intervaloo{0}{1}$ ? 
\end{enumerate}

\end{exo}

\end{document}











