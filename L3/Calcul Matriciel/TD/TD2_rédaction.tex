\documentclass[12pt,a4paper]{article}
\usepackage[left=2cm,right=2cm,top=2cm,bottom=2cm]{geometry}
\usepackage{amsmath}
\usepackage{amsthm}
\usepackage{thmtools}
\usepackage{amsfonts}
\usepackage{amssymb}
\usepackage[utf8]{inputenc}
\usepackage[T1]{fontenc}
\usepackage[francais]{babel}
\usepackage{mathtools}   % loads »amsmath
\usepackage[dvipsnames]{xcolor}
\usepackage{ulem}
\usepackage{enumitem}
\usepackage{pifont}
\usepackage[most]{tcolorbox}
\usepackage[scr=rsfs]{mathalpha}
\usepackage{esvect} %vecteur avec \vv
\usepackage{dsfont}


\usepackage{calc}
\usepackage[nothm]{thmbox}



%Raccourcies
\newcommand{\N}{\mathbb{N}}
\newcommand{\1}{\mathds{1}}
\newcommand{\PP}{\mathds{P}}
\newcommand{\Z}{\mathbb{Z}}
\newcommand{\E}{\mathds{E}}
\newcommand{\K}{\mathbb{K}}
\newcommand{\Q}{\mathbb{Q}}
\newcommand{\R}{\mathbb{R}}
\newcommand{\C}{\mathbb{C}}
\newcommand{\ie}{\textit{i.e }}
\newcommand{\A}{\mathscr{A}}
\newcommand{\Aa}{\mathcal{A}}
\newcommand{\Mm}{\mathcal{M}}
\newcommand{\Chi}{\mathcal{X}}
\newcommand{\Ll}{\mathcal{L}}
\newcommand{\Tup}{\bigtriangleup}
\newcommand{\Tdn}{\bigtriangledown}
\newcommand{\bps}{\langle}
\newcommand{\eps}{\rangle}
\newcommand{\pv}{\wedge}
\newcommand{\doigt}{\ding{43} ~}
\newcommand{\cray}{\ding{46} ~}


% Définir une commande pour la boîte encadrée
\newcommand{\todo}[1]{%
    \begin{tcolorbox}[colback=red!10!white, colframe=red!75!black, title=To-do :]
        #1
    \end{tcolorbox}%
}
%\begin{tcolorbox}[colback=red!10!white, colframe=red!75!black, title=Ma Boîte Encadrée]
%    Ceci est un exemple de texte à l'intérieur de la boîte.
%\end{tcolorbox}





\def\BC{base canonique }
\def\ev{espace vectoriel }
\def\evs{espaces vectoriels }
\def\sevs{sous-espaces vectoriels }
\def\sev{sous-espace vectoriel }
\def\sep{sous-espace propre }
\def\seps{sous-espaces propres }
\def\sea{sous-espace affine }
\def\seas{sous-espaces affines }
\def\AL{application lin\'eaire }
\def\ALs{applications lin\'eaires }
\def\AA{application affine }
\def\AAs{applications affines }
\def\Aff{ \hbox{\it Aff}}
\def\vep{vecteur propre }
\def\vap{valeur propre }
\def\ssi{si et seulement si }

\newcommand{\dessous}[2]{\underset{#1}{#2}}

\def\bar#1{\overline{#1}}
\def\e#1{{\hbox{e}^{#1}}}
\def\mat#1{\begin{pmatrix}#1\end{pmatrix}}
\def\vect#1{\overrightarrow{\kern-1pt#1\kern 2pt}}
\def\card#1{{\hbox{Card}(#1)}}
\def\bin(#1,#2){ \left(\!\begin{smallmatrix} #2 \\ #1 \end{smallmatrix}\!\right)}
\def\Aff{ \hbox{\it Aff}}
\def\lims{\,\overline{\lim}\;}
\def\limi{\,\underline{\lim}\;}
\def\vide{\emptyset}
\def\lims{\,\overline{\lim}\;}
\def\limi{\,\underline{\lim}\;}
\def\vide{\emptyset}
\def\vfi{\varphi}
\def\dim#1{\text{dim }(#1)}

\def\ben{\begin{enumerate}}
\def\een{\end{enumerate}}
\def\bi{\begin{itemize}}
\def\ei{\end{itemize}}

\DeclareMathOperator{\Id}{Id}
\DeclareMathOperator{\Tr}{Tr}
\DeclareMathOperator{\rg}{rg}

\definecolor{rltred}{rgb}{0.75,0,0}
	\definecolor{rltgreen}{rgb}{0,0.5,0}
	\definecolor{oneblue}{rgb}{0,0,0.75}
	\definecolor{marron}{rgb}{0.64,0.16,0.16}
	\definecolor{forestgreen}{rgb}{0.13,0.54,0.13}
	\definecolor{purple}{rgb}{0.62,0.12,0.94}
	\definecolor{dockerblue}{rgb}{0.11,0.56,0.98}
	\definecolor{freeblue}{rgb}{0.25,0.41,0.88}
	\definecolor{myblue}{rgb}{0,0.2,0.4}

%Note
\newenvironment{note}{\par\medskip\noindent\begin{tabular}{l|p{\linewidth-8cm}}
\ding{46}& \color{black}}
{\end{tabular}\par\medskip}
\def\bn{\begin{note}}
\def\en{\end{note}}


%Attention
\def\war{\color{red}\medskip\begin{thmbox}[M]{\ding{39}\textbf{Attention :}}\noindent\color{black}}
\def\endwar{\end{thmbox}}
\def\bw{\begin{war}}
\def\ew{\end{war}}


%Style définition
\declaretheoremstyle[
  headfont=\color{RoyalBlue}\normalfont\bfseries,
  bodyfont=\color{NavyBlue}\normalfont,
]{colored}
\declaretheorem[
  style=colored,
  name=Définition,
]{df}
\def\bd{\begin{df}}
\def\ed{\end{df}}

%Style Proposition
\declaretheoremstyle[
  headfont=\color{BrickRed}\normalfont\bfseries,
  bodyfont=\color{black}\normalfont,
]{coloredProp}
\declaretheorem[
  style=coloredProp,
  name=Proposition,
]{prop}
\def\bp{\begin{prop}}
\def\ep{\end{prop}}

%Style exo
\declaretheoremstyle[
  headfont=\color{orange}\normalfont\bfseries,
  bodyfont=\color{black}\normalfont,
]{coloredexo}
\declaretheorem[
  style=coloredexo,
  name=Exercice,
]{exo}
\def\be{\begin{exo}}
\def\ee{\end{exo}}

%Règle soulignée
\def\regle{\color{blue}\begin{thmbox}[M]{\textbf{Règle :}}\noindent\color{blue}}
\def\endregle{\end{thmbox}}
\def\br{\begin{regle}}
\def\er{\end{regle}}

%Démonstration
\let\oldproof\proof
\renewcommand{\proof}{\color{olive}\oldproof}
\def\bpf{\begin{proof}}
\def\epf{\end{proof}}

%\theoremstyle{definition}
%\newtheorem{prop}{Proposition}
%\newtheorem{exo}{Exercice}
%\newtheorem{df}{Définition}

\newcommand{\bpropo}{
    \begin{tcolorbox}[colback=Yellow!10!Yellow, colframe=red!75!black]
    \bp
}
\newcommand{\epropo}{
    \ep
    \end{tcolorbox}}

%Solution
\newcommand{\solution}[1]{\par\noindent\textbf{\color{OliveGreen}Solution :} \textcolor{OliveGreen}{#1}}

%Titre
%\title{Espaces affines, notes   }
\author{$\mathcal{F.J}$}
%\date{2023-2024}


\newcommand{\tn}{\mathrm}
\newcommand{\eref}[1]{(\ref{#1})}
\newcommand{\dep}[2]{\partial_{#2}{#1}}
\newcommand{\dis}{\displaystyle}
\newcommand{\MmnK}{\mathcal{M}_{m,n}(\K)}
\newcommand{\MnpK}{\mathcal{M}_{n,p}(\K)}
\newcommand{\MnK}{\mathcal{M}_{n}(\K)}
\newcommand{\MmK}{\mathcal{M}_{m}(\K)}
\newcommand{\MmnR}{\mathcal{M}_{m,n}(\R)}
\newcommand{\MnR}{\mathcal{M}_{n}(\R)}
\newcommand{\MnC}{\mathcal{M}_{n}(\C)}
\newcommand{\MmR}{\mathcal{M}_{m}(\R)}
\newcommand{\nmt}[1]{|\mskip -2mu|\mskip -2mu|{#1}|\mskip -2mu |\mskip -2mu|}


\title{TD2 rédaction}

\begin{document}
\maketitle
\begin{exo} {\bf Théorème de Gerschgörin - localisation des valeurs propres}.\ \\
  Si $n\in\N,\ n\geq1,$ et $A\in\MnK$, on note $spec(A)$ le spectre de la matrice $A$, c'est-à-dire l'ensemble des valeurs propres de $A$ et, pour chaque $i\in\{1,\dots,n\}$, on note $D_i(A)$ l'ensemble
  $$
  D_i(A)=\{z\in\C\, :\, |z-A_{ii}|\leq \sum_{\substack{j=1\\j\neq i}}^n|A_{ij}|\}.
  $$
  $D_i$ s'appelle le $i$-ème disque de Gerschgörin de la matrice $A$.
  Soit $n\in\N,\ n\geq1,$ et $A\in\MnK$.
\ben
  \item Montrer que pour tout $\lambda\in spec(A)$,
  $$
\lambda\in\displaystyle{\bigcup_{i=1}^n }D_i(A).
  $$
	 \solution{Soit donc $\lambda \in Sp(A)$, montrons que $\lambda$ est au moins dans l'un des $D_i(A)$.\\
	On sait que $\lambda \in Sp(A)$, \ie $\exists u \in \K^n$ tel que $Au = \lambda u$ avec $u \ne 0$. \\
	Définissons maintenant $k_0$ tel que $|u_{k_0} | = \displaystyle \max_{i \in \{1,...,n\}} |u_i|$. Comme $u \ne 0$, $u_{k_0} \ne 0$. \\
	On a donc $\displaystyle \lambda u_{k_0} = \sum_{k=1}^n A_{k_0k} u_k$ qui nous donne $\lambda u_{k_0} - A_{k_0 k_0} u_{k_0} = \displaystyle \sum_{k=1,k\ne k_0}^n A_{k_0 k}u_{k_0}$\\
	Ce qui nous donne : $| \lambda - A_{k_0 k_0}|| u_{k_0}| = \displaystyle \sum_{k=1,k\ne k_0}^n | A_{k_0 k}||u_{k_0}|$\\
	Et par suite : $|\lambda - A_{k_0 k_0}| \leq \displaystyle \sum_{k=1,k\ne k_0}^n |A_{k_0 k}|$ car $\frac{|u_k|}{|u_{k_0}|} \leq 1$
	Donc on a bien $\lambda$ dans l'un des $D_i(A)$.
	}	
 \item En déduire que le rayon spectral $\rho(A):=\displaystyle{\max_{\lambda\in spec(A)}|\lambda|}$ de $A$ vérifie
  $$
  \rho(A)\leq\displaystyle{\max_{1\leq i\leq n}\sum_{j=1}^n|A_{ij}|}.
  $$
\een

\end{exo}
\newpage

\be
\begin{enumerate}
    \item[]
\item   Soit $A\in\MnR$ orthogonale ({\it i.e.} vérifiant $AA^T=A^TA=I_n$). Montrer que si $\lambda\in\C$ est une valeur propre de $A$, alors $|\lambda|=1$.
	\solution{Supposons donc $A \in \Mm_n(\R)$ orthogonale et $\lambda \in Sp(A)$, on sait donc qu'il existe un vecteur $u$ non nul tel que $Au = \lambda u$. En prenant la norme de $Au$ on obtient : $\|Au\| = \|\lambda u \| = |\lambda| \|u\|$. D'autre part on a aussi $\|Au\|^2 = \bps Au | Au \eps = \bps u | A^tAu \eps = \bps u | u \eps$ car $AA^t = I_n$. Donc on a $|\lambda | \|u \| = \|u\| \Rightarrow |\lambda | = 1$. 
		}
    \item On considère ici le produit scalaire euclidien et la norme euclidienne sur $\R^n$. Soit $A\in\MnR$.
      \begin{enumerate}
    \item Montrer que $A$ est orthogonale si et seulement si les colonnes de $A$ forment une base o. n. de $\R^n$ si et seulement si $A$ préserve le produit scalaire, autrement dit si et seulement si pour tous $x,\ y\in\R^n$,
      $$
(Ax|Ay)=(x|y).
      $$
		      \solution{Pour $A$ orthogonale $\iff$ les colonnes $C_1,...,C_n$ de $A$ forment une $BON$ déjà fait en cours.\\
		      Montrons que $A$ orthogonale $\iff$ $A$ préserve le produit scalaire : 
		      \bi
	      \item[ $\Rightarrow$] Supposons $A$ orthogonale, \ie $AA^t = I_n$ et montrons que $A$ préserve le produit scalaire. \\ 
		      On a donc $\bps Ax | Ay \eps = \bps x | A^t A y \eps = \bps x | y \eps$ car $A^t A = A A^t = I_n$
	      \item[$\Leftarrow$] Supposons que $A$ préserve le produit scalaire et montrons que $A$ est orthogonale : \\
		      $\bps Ax | Ay \eps = \bps x | A^t A y \eps = \bps x | y \eps$ et cela quelques soient $x,y \in \R^n$. \\
		      On a donc $\bps Ax | Ay \eps - \bps x | y \eps = 0, ~ \forall x,y \in \R^n$, donc $\bps x | (A^tA)y \eps = 0, ~ \forall x,y \in \R^n$ \ie $(A^tA)y = 0, ~ \forall y \in \R^n$, d'où $AA^t = I_n$. 
		      \ei
		      }
      \item Montrer que $A$ préserve le produit scalaire si et seulement si $A$ préserve la norme, autrement dit si et seulement si pour tout $x\in\R^n$,
      $$
\|Ax\|=\|x\|,
      $$ 
et en déduire que $A$ est orthogonale si et seulement si $A$ préserve la norme.
	\solution{\bi
\item[\Rightarrow] évident. 
\item[\Leftarrow] Supposons que $A$ préserve la norme, et en utilisant l'identité de polarisation on a $\bps Ax | Ay \eps = \frac 12(\|Ax + Ay \|^2 - \|Ax\|^2 - \|Ay\|^2) = \frac 12 (\|x+y\|^2 - \|x \| - \|y \| ^2) = \bps x | y \eps$.
	\ei}
      \end{enumerate}
\end{enumerate}
\ee

\be[Caractérisation des valeurs propres pour des matrices hermitiennes - théorème de Courant-Fischer]
Soit $n\in\N,\ n\geq1,$ et $A\in\MnC$ une matrice hermitienne (auto-adjointe). Soient $\lambda_1\leq\dots\leq\lambda_n$ les $n$ valeurs propres réelles de $A$. \\
\ben
	\item Montrer que : 
  $$
\lambda_1=\min_{x\in\C^n,\ x\neq0}\frac{(Ax|x)_\C}{\|x\|_2^2}=\min_{x\in\C^n,\ \|x\|_2=1}(Ax|x)_\C
$$
et que
  $$
\lambda_n=\max_{x\in\C^n,\ x\neq0}\frac{(Ax|x)_\C}{\|x\|_2^2}=\max_{x\in\C^n,\ \|x\|_2=1}(Ax|x)_\C
$$
  \item Montrer de même si $A \in \Mm_n(\R)$ :
	  $$
\lambda_1=\min_{x\in\R^n,\ x\neq0}\frac{(Ax|x)}{\|x\|_2^2}=\min_{x\in\R^n,\ \|x\|_2=1}(Ax|x)
$$
\solution{On a $A \in \Mm_n(\R)$ symétrique (\ie $A^t = A$) donc $A$ est diagonalisable dans une $BON$ de vecteurs propres, \ie qu'il existe une base orthonormée de vecteurs propres $(u_1,...,u_n)$, ordonnons les valeurs propres $\lambda_1 \leq \lambda_2 \leq ... \leq \lambda_n$ ({\it il peut y en avoir qui se répète vu que l'on va jusqu'à $n$}). On se place dans la base orthonormée $(u_1,...,u_n)$, dans cette base $x = \displaystyle \sum_{k=1}^n x_k e_k$. Donc en prenant la norme on a $\|x\| = \displaystyle \sum_{k=1}^n \|x_k\|$
}\\
et que
  $$
\lambda_n=\max_{x\in\R^n,\ x\neq0}\frac{(Ax|x)}{\|x\|_2^2}=\max_{x\in\R^n,\ \|x\|_2=1}(Ax|x).
$$
\een
\ee
\end{document}
