\documentclass[12pt,a4paper]{article}
\usepackage[left=2cm,right=2cm,top=2cm,bottom=2cm]{geometry}
\usepackage{amsmath}
\usepackage{amsthm}
\usepackage{thmtools}
\usepackage{amsfonts}
\usepackage{amssymb}
\usepackage[utf8]{inputenc}
\usepackage[T1]{fontenc}
\usepackage[francais]{babel}
\usepackage{mathtools}   % loads »amsmath
\usepackage[dvipsnames]{xcolor}
\usepackage{ulem}
\usepackage{enumitem}
\usepackage{pifont}
\usepackage[most]{tcolorbox}
\usepackage[scr=rsfs]{mathalpha}
\usepackage{esvect} %vecteur avec \vv
\usepackage{dsfont}


\usepackage{calc}
\usepackage[nothm]{thmbox}



%Raccourcies
\newcommand{\N}{\mathbb{N}}
\newcommand{\1}{\mathds{1}}
\newcommand{\PP}{\mathds{P}}
\newcommand{\Z}{\mathbb{Z}}
\newcommand{\E}{\mathds{E}}
\newcommand{\K}{\mathbb{K}}
\newcommand{\Q}{\mathbb{Q}}
\newcommand{\R}{\mathbb{R}}
\newcommand{\C}{\mathbb{C}}
\newcommand{\ie}{\textit{i.e }}
\newcommand{\A}{\mathscr{A}}
\newcommand{\Aa}{\mathcal{A}}
\newcommand{\Mm}{\mathcal{M}}
\newcommand{\Chi}{\mathcal{X}}
\newcommand{\Ll}{\mathcal{L}}
\newcommand{\Tup}{\bigtriangleup}
\newcommand{\Tdn}{\bigtriangledown}
\newcommand{\bps}{\langle}
\newcommand{\eps}{\rangle}
\newcommand{\pv}{\wedge}
\newcommand{\doigt}{\ding{43} ~}
\newcommand{\cray}{\ding{46} ~}


% Définir une commande pour la boîte encadrée
\newcommand{\todo}[1]{%
    \begin{tcolorbox}[colback=red!10!white, colframe=red!75!black, title=To-do :]
        #1
    \end{tcolorbox}%
}
%\begin{tcolorbox}[colback=red!10!white, colframe=red!75!black, title=Ma Boîte Encadrée]
%    Ceci est un exemple de texte à l'intérieur de la boîte.
%\end{tcolorbox}





\def\BC{base canonique }
\def\ev{espace vectoriel }
\def\evs{espaces vectoriels }
\def\sevs{sous-espaces vectoriels }
\def\sev{sous-espace vectoriel }
\def\sep{sous-espace propre }
\def\seps{sous-espaces propres }
\def\sea{sous-espace affine }
\def\seas{sous-espaces affines }
\def\AL{application lin\'eaire }
\def\ALs{applications lin\'eaires }
\def\AA{application affine }
\def\AAs{applications affines }
\def\Aff{ \hbox{\it Aff}}
\def\vep{vecteur propre }
\def\vap{valeur propre }
\def\ssi{si et seulement si }

\newcommand{\dessous}[2]{\underset{#1}{#2}}

\def\bar#1{\overline{#1}}
\def\e#1{{\hbox{e}^{#1}}}
\def\mat#1{\begin{pmatrix}#1\end{pmatrix}}
\def\vect#1{\overrightarrow{\kern-1pt#1\kern 2pt}}
\def\card#1{{\hbox{Card}(#1)}}
\def\bin(#1,#2){ \left(\!\begin{smallmatrix} #2 \\ #1 \end{smallmatrix}\!\right)}
\def\Aff{ \hbox{\it Aff}}
\def\lims{\,\overline{\lim}\;}
\def\limi{\,\underline{\lim}\;}
\def\vide{\emptyset}
\def\lims{\,\overline{\lim}\;}
\def\limi{\,\underline{\lim}\;}
\def\vide{\emptyset}
\def\vfi{\varphi}
\def\dim#1{\text{dim }(#1)}

\def\ben{\begin{enumerate}}
\def\een{\end{enumerate}}
\def\bi{\begin{itemize}}
\def\ei{\end{itemize}}

\DeclareMathOperator{\Id}{Id}
\DeclareMathOperator{\Tr}{Tr}
\DeclareMathOperator{\rg}{rg}

\definecolor{rltred}{rgb}{0.75,0,0}
	\definecolor{rltgreen}{rgb}{0,0.5,0}
	\definecolor{oneblue}{rgb}{0,0,0.75}
	\definecolor{marron}{rgb}{0.64,0.16,0.16}
	\definecolor{forestgreen}{rgb}{0.13,0.54,0.13}
	\definecolor{purple}{rgb}{0.62,0.12,0.94}
	\definecolor{dockerblue}{rgb}{0.11,0.56,0.98}
	\definecolor{freeblue}{rgb}{0.25,0.41,0.88}
	\definecolor{myblue}{rgb}{0,0.2,0.4}

%Note
\newenvironment{note}{\par\medskip\noindent\begin{tabular}{l|p{\linewidth-8cm}}
\ding{46}& \color{black}}
{\end{tabular}\par\medskip}
\def\bn{\begin{note}}
\def\en{\end{note}}


%Attention
\def\war{\color{red}\medskip\begin{thmbox}[M]{\ding{39}\textbf{Attention :}}\noindent\color{black}}
\def\endwar{\end{thmbox}}
\def\bw{\begin{war}}
\def\ew{\end{war}}


%Style définition
\declaretheoremstyle[
  headfont=\color{RoyalBlue}\normalfont\bfseries,
  bodyfont=\color{NavyBlue}\normalfont,
]{colored}
\declaretheorem[
  style=colored,
  name=Définition,
]{df}
\def\bd{\begin{df}}
\def\ed{\end{df}}

%Style Proposition
\declaretheoremstyle[
  headfont=\color{BrickRed}\normalfont\bfseries,
  bodyfont=\color{black}\normalfont,
]{coloredProp}
\declaretheorem[
  style=coloredProp,
  name=Proposition,
]{prop}
\def\bp{\begin{prop}}
\def\ep{\end{prop}}

%Style exo
\declaretheoremstyle[
  headfont=\color{orange}\normalfont\bfseries,
  bodyfont=\color{black}\normalfont,
]{coloredexo}
\declaretheorem[
  style=coloredexo,
  name=Exercice,
]{exo}
\def\be{\begin{exo}}
\def\ee{\end{exo}}

%Règle soulignée
\def\regle{\color{blue}\begin{thmbox}[M]{\textbf{Règle :}}\noindent\color{blue}}
\def\endregle{\end{thmbox}}
\def\br{\begin{regle}}
\def\er{\end{regle}}

%Démonstration
\let\oldproof\proof
\renewcommand{\proof}{\color{olive}\oldproof}
\def\bpf{\begin{proof}}
\def\epf{\end{proof}}

%\theoremstyle{definition}
%\newtheorem{prop}{Proposition}
%\newtheorem{exo}{Exercice}
%\newtheorem{df}{Définition}

\newcommand{\bpropo}{
    \begin{tcolorbox}[colback=Yellow!10!Yellow, colframe=red!75!black]
    \bp
}
\newcommand{\epropo}{
    \ep
    \end{tcolorbox}}

%Solution
\newcommand{\solution}[1]{\par\noindent\textbf{\color{OliveGreen}Solution :} \textcolor{OliveGreen}{#1}}

%Titre
%\title{Espaces affines, notes   }
\author{$\mathcal{F.J}$}
%\date{2023-2024}


\usepackage{amsmath, amssymb}
\author{Anito Kodama}
\title{Théorème de Convergence monotone}
\begin{document}
\maketitle

\bp[Théorème de Beppo-Levi - (Convergence monotone)]
Si les variables aléatoires $X_n$ sont positives et si la suite $(X_n)_n$ converge simplement presque sûrement vers $X$, alors
$$ \lim_{n \to \infty} \mathbb{E}[X_n] = \mathbb{E}[X], $$ 
et cela même si $\mathbb{E}[X] = \infty$.
\ep

\bpf

\subsubsection*{Montrons la convergence de l'espérance par étapes.}

\begin{enumerate}
    \item $\mathbb{E}[X_n] \geq 0$, et la suite $\mathbb{E}[X_n]$ est croissante. Ainsi, il existe un $\alpha \geq 0$ tel que
    $$ \lim_{n \to \infty} \mathbb{E}[X_n] = \alpha. $$

    \item Comme $X_n \leq X$ pour tout $n$, on a également :
    $$ \mathbb{E}[X_n] \leq \mathbb{E}[X]. $$

    \item On en déduit alors que :
    $$ \alpha \leq \mathbb{E}[X]. $$
\end{enumerate}

\subsubsection*{Montrons que $\alpha \geq \mathbb{E}[X]$}

Pour cela, nous devons prouver que pour toute variable aléatoire étagée \( S \) telle que \( 0 \leq S \leq X \), on a :
$$ \alpha \geq \mathbb{E}[S]. $$

Prenons donc une variable aléatoire étagée \( S \geq 0 \) telle que \( S \leq X \). Montrons que pour tout \( c \in [0, 1[ \), on a :
$$ \alpha \geq c \cdot \mathbb{E}[S] = \mathbb{E}[cS]. $$

\subsubsection*{Construction des événements}

Considérons les événements \( E_{n,c} = \{X_n \geq cS\} \). Ces événements forment une suite croissante, car \( (X_n) \) est croissante. De plus, on a :
$$ \bigcup_{n \geq 0} E_{n,c} = \Omega. $$

Ensuite, on utilise les inégalités suivantes :
\[
\mathbb{E}[X_n] \geq \mathbb{E}[X_n \cdot \mathbf{1}_{E_{n,c}}] = \mathbb{E}[X_n \cdot \mathbf{1}_{X_n \geq cS}] \geq \mathbb{E}[cS \cdot \mathbf{1}_{X_n \geq cS}].
\]

\subsubsection*{Calculs d'espérance}

Nous avons :
\[
\mathbb{E}[cS \cdot \mathbf{1}_{E_{n,c}}] = c \cdot \mathbb{E}\left[\sum_{i=1}^K a_i \mathbf{1}_{A_i} \cdot \mathbf{1}_{E_{n,c}}\right] 
= c \cdot \sum_{i=1}^K a_i \mathbb{P}(A_i \cap E_{n,c}).
\]

Comme \( A_i \cap E_{n,c} \) est croissant, on a :
\[
\lim_{n \to \infty} \mathbb{P}(A_i \cap E_{n,c}) = \mathbb{P}(A_i).
\]

D'où :
\[
\lim_{n \to \infty} \mathbb{E}[cS \cdot \mathbf{1}_{E_{n,c}}] = c \cdot \sum_{i=1}^K a_i \mathbb{P}(A_i) = c \cdot \mathbb{E}[S].
\]

\subsubsection*{Conclusion}

Ainsi, on a :
\[
\lim_{n \to \infty} \mathbb{E}[X_n] \geq c \cdot \mathbb{E}[S], \quad \forall c \in [0, 1[.
\]

En faisant tendre \( c \) vers 1, on en déduit que \( \alpha \geq \mathbb{E}[S] \) pour toute variable étagée \( S \) telle que \( 0 \leq S \leq X \). Finalement, comme \( X \) peut être approchée par des variables étagées \( S \), on conclut que :
$$ \alpha \geq \mathbb{E}[X]. $$

Étant donné que \( \alpha \leq \mathbb{E}[X] \) (voir l'étape 2), on en conclut que :
$$ \alpha = \mathbb{E}[X]. $$

Ainsi, on a montré que :
$$ \lim_{n \to \infty} \mathbb{E}[X_n] = \mathbb{E}[X]. $$

\epf

\section*{Explications de la preuve}

La preuve repose sur des propriétés de croissance, d'inégalités, et sur l'usage de variables aléatoires étagées qui permettent d'approximer $X$.

\subsection*{1. Premier point : $\mathbb{E}[X_n] \to \alpha$}

La première étape consiste à remarquer que les espérances $\mathbb{E}[X_n]$ sont croissantes car $X_n \leq X_{n+1}$ pour tout $n$, et que les $X_n$ sont positives. Par conséquent, $\mathbb{E}[X_n] \geq 0$ et forme une suite croissante. 

Une suite croissante bornée (elle est bornée par $X$) converge nécessairement, et donc il existe une limite, notée $\alpha$, telle que
\[
\lim_{n \to \infty} \mathbb{E}[X_n] = \alpha.
\]

\subsection*{2. Inégalité avec $\mathbb{E}[X]$}

Puisque $X_n \leq X$ presque sûrement (p.s.) pour tout $n$, on a également l'inégalité
\[
\mathbb{E}[X_n] \leq \mathbb{E}[X].
\]
Cela signifie que la limite $\alpha$ est inférieure ou égale à $\mathbb{E}[X]$, c'est-à-dire :
\[
\alpha \leq \mathbb{E}[X].
\]
Nous avons donc déjà une borne supérieure sur $\alpha$.

\subsection*{3. Inégalité avec $\mathbb{E}[S]$ pour $S \leq X$}

Pour montrer que $\alpha = \mathbb{E}[X]$, on introduit une variable aléatoire étagée $S$, qui est une approximation simple de $X$. Par construction, $S \leq X$, et il suffit de montrer que pour toute telle variable $S$, on a
\[
\alpha \geq \mathbb{E}[S].
\]
Cela entraînera alors que $\alpha \geq \mathbb{E}[X]$ (car $X$ peut être vu comme la limite de telles approximations).

\subsection*{4. Utilisation des indicatrices $E_{n,c}$}

Prenons un facteur $c \in [0,1[$ et considérons l'ensemble des événements $E_{n,c} = \{X_n \geq cS\}$, qui sont croissants avec $n$ (puisque $X_n$ est croissant). En prenant la limite, on obtient :
\[
\bigcup_{n \geq 0} E_{n,c} = \Omega
\]
par convergence simple presque sûrement de $X_n \to X$.

Ensuite, on montre que :
\[
\mathbb{E}[X_n] \geq \mathbb{E}[X_n \times \mathbf{1}_{E_{n,c}}] \geq \mathbb{E}[cS \times \mathbf{1}_{E_{n,c}}],
\]
ce qui nous donne une borne inférieure pour $\mathbb{E}[X_n]$.

\subsection*{5. Passage à la limite}

Lorsque $n \to \infty$, l'événement $E_{n,c}$ converge presque sûrement vers $\{S \leq X\}$, et donc l'espérance $\mathbb{E}[cS \times \mathbf{1}_{E_{n,c}}]$ converge vers $c \times \mathbb{E}[S]$. Ceci implique que :
\[
\lim_{n \to \infty} \mathbb{E}[X_n] \geq c \times \mathbb{E}[S].
\]
Comme cela est vrai pour tout $c \in [0,1[$, on obtient finalement que :
\[
\alpha \geq \mathbb{E}[S].
\]

\subsection*{6. Conclusion : $\alpha = \mathbb{E}[X]$}

Comme cela est vrai pour toute variable $S$ étagée telle que $0 \leq S \leq X$, on peut conclure que $\alpha \geq \mathbb{E}[X]$. Mais, comme on avait déjà $\alpha \leq \mathbb{E}[X]$, on conclut que :
\[
\alpha = \mathbb{E}[X].
\]

\subsection*{Résumé final}

Ainsi, la preuve montre que la suite croissante d'espérances $\mathbb{E}[X_n]$ converge vers $\mathbb{E}[X]$, même lorsque $\mathbb{E}[X] = \infty$, en utilisant l'approximation par des variables étagées et des inégalités sur des espérances conditionnelles.

\end{document}


