\documentclass[a4paper,12pt]{article}
\usepackage[utf8]{inputenc}
\usepackage[T1]{fontenc}
\usepackage[french]{babel}

\title{Exercices de Méditation et Pleine Conscience pour l'évaluation}
\author{Floryan JOURDAN}
\date{\today}

\begin{document}

\maketitle

\section*{Exercice 1 : Méditation}

J'ai choisi de pratiquer la méditation sur l'acceptation, guidée par l'enregistrement disponible dans "Ressources - Enregistrements" sur e-campus. Pendant la séance, j'ai trouvé que c'était à la fois une expérience enrichissante et stimulante - comme à chaque fois que j'ai eu l'occasion de pratiquer depuis le début d'année. 

Au début, il m'a été un peu difficile de me détacher des pensées tourbillonnantes de la journée qui abondent en cette periode d'examens, mais à mesure que la méditation progressait, j'ai réussi à me concentrer davantage sur le guide vocal et les instructions. La méditation sur l'acceptation m'a amené à explorer mes émotions présentes sans jugement, et j'ai ressenti une certaine résistance initiale.

Cependant, au fil du temps, une sensation de calme s'est installée. J'ai remarqué des pensées qui surgissaient, certaines positives et d'autres négatives, mais l'accent mis sur l'acceptation m'a aidé à ne pas m'attarder sur celles-ci bien que cela ne soit pas chose facile. Des émotions variées ont émergé, de la frustration, de le colère même, de la joie aussi, mais en les accueillant sans jugement, j'ai ressenti un relâchement progressif.

Physiquement, j'ai ressenti une détente au niveau de la respiration, et la tension musculaire que je ressentais au début s'est atténuée, par contre j'e n'arrive jamais à me débarasser de la tension au niveau du dos, je trouve cela dommage car parfois je ne parviens pas à me concentrer pleinement sur une médiation... Mais en revenant à cette méditation, celle ci m'a laissé avec une sensation de légèreté et un état d'esprit plus équilibré - j'avais commencé cette méditation par un exercice que Hugues nous avait montré, celui ou il faut secouer les mains et s'arrêter puis ne rien faire un instant, couplé à la méditation j'ai vraiment senti cet état d'esprit apaisé.

\section*{Exercice 2 : Bilan de l’UE}

Ne connaissant pratiquement rien à la méditation mais ayant toujours été attiré par la pratique, ce cours de méditation a été une expérience très intéressante, marquée par une exploration approfondie des pratiques méditatives et de leurs effets sur différents aspects de ma vie. 

Parmi les pratiques abordées, la méditation sur l'acceptation a particulièrement captivé mon intérêt - d'où mon choix pour l'exercice précédent. Cette approche m'a permis de développer une meilleure compréhension de mes émotions et de cultiver une attitude de non-jugement envers mes pensées ou envers autruit. Cependant, certaines pratiques en particulier les exercices debout ont été plus difficiles à faire et à intégrer dans ma routine, soulignant la nécessité d'une adaptabilité dans ma démarche méditative. Par ailleurs, j'ai également pris l'habitude de faire un scan corporel chaque soir avant de dormir, ce qui m'aide à m'endormir par la suite.

Le cours a été instructif sur le plan théorique, présentant divers mécanismes biologiques sous-tendant les pratiques méditatives et leurs bienfaits, les types de resirations - par exemple que la respiration abdominale vient stimuler le système parasympathique ce qui permet de se relaxer rapidement. Les enseignements sur l'attention aux émotions, le lâcher prise et la bienveillance envers soi-même et les autres ont eu un impact notable sur ma vie quotidienne. Des prises de conscience significatives ont émergé quant à mon mode de fonctionnement émotionnel, offrant des perspectives nouvelles sur la gestion du stress et des situations difficiles, la prise en compte de l'actualité pour les méditations au dojo était à mon goût très bien pour la mise en context de la méditation dans la vie quotidienne.

Intégrer la méditation dans ma routine quotidienne a renforcé les bienfaits ressentis pendant les cours. Les enseignements ont également trouvé leur place dans des situations de la vie quotidienne, agissant comme des outils pratiques pour maintenir la présence et la clarté mentale - tout du moins c'est vers quoi j'essaye de tendre.

Au-delà du contenu du cours, j'ai exploré d'autres ressources, écoutant et lisant des méditations et des textes d'auteurs proposé en cours. Cette diversification m'a offert une perspective élargie sur les différentes approches méditatives.

En somme, ce cours de méditation a été un voyage introspectif éclairant. Les pratiques ont façonné ma compréhension de moi-même, des autres et ont eu des répercussions positives dans ma vie quotidienne. Je suis reconnaissant pour cette expérience qui a ouvert la porte à un chemin continu d'exploration et de croissance personnelle à travers la méditation, je tiens à vous remercier tous les deux pour cela.

\section*{Exercice 3 : Définition de la méditation/pleine conscience}

C'est une pratique de présence totale, accueillant sans jugement toutes les pensées, sensations et émotions, permettant ainsi d'éviter d'être emporté par le flot des pensées négatives, favorisant le lâcher-prise et la gestion du stress.

\end{document}

