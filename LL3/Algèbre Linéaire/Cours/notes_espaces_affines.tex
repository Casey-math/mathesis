\documentclass{../../tete}
\title{Espaces affines, notes   }

\begin{document}
\maketitle
\section{Espace affine}
\subsection{Définition}
\bd
Soit $E$ un ensemble non vide. 
On dit que $E$ est un \textbf{espace affine} sur $\vec{E}$ losqu'il existe une \textit{loi de composition externe} (notée +) de $E \times \vec{E}$ dans $E$ qui vérifie :
\ben
\item $\forall A \in E, A + \vec{0} = A$ ;
\item $\forall A \in E, \forall \vec{u}, \vec{v} \in \vec{E}$, on a $(A + \vec{u}) + \vec{v} = A + (\vec{u} + \vec{v})$;
\item $\forall A \in E$, l'application $\vec{u} \mapsto A + \vec{u}$ est une \textbf{bijection} de $\vec{E}$ sur $E$.
\een
\ed
\bn
\star ~ Les éléments de $E$ sont souvent appelés \textbf{points}, alors que les éléments de $\vec{E}$ sont appelés \textbf{vecteurs}.

\star ~ On dit que $\vec{E}$ est l'espace vectoriel associé à $E$, ou encore que c'est la \textbf{direction} de $E$.
\en

\bw
\star ~ Le symbole $+$ entre deux vecteurs désigne l'addition des vecteurs, celle qui fait de $(\vec{E}, +)$ un groupe abélien.\\
\star ~ Entre un point et un vecteur en revanche, il s'agit de la loi de composition externe dont on a besoin pour définir un espace affine.\\
\ding{43} ~ Bien remarquer de quelle loi on parle dans le 2. de la définition.
\ew

\bn
Un espace affine n'est pas seulement un ensemble : c'est une structure; 
\ben
\item l'ensemble ; 
\item l'espace vectoriel ;
\item la L.C.E \textit{(loi de composition externe)}.
\een
\doigt En pratique on confond souvent l'espace affince $(E, \vec{E}, +)$ avec l'ensemble sous-jacent $E$.
\en

\medskip

\noindent \dag ~ En utilisant le vocabulaire des actions de groupe, on peut traduire la définition par : Le groupe additif $(\vec{E}, +)$ opère librement et transitivement sur $\vec{E}$.

\noindent \dag ~ Lorsque $\vec{u}$ est l'unique vecteur tel que $B = A + \vec{u}$, on notera $\vec{u} = \vec{AB} = B - A$.

\subsection{Premières propriétés}
\bp
Pour tout $A,B,C \in E$, on a la relation de Chasles : $\vec{AB}+\vec{BC} = \vec{AC}$.
\ep
\begin{proof}
On peut écrire cette relation sous la forme : $(B-A) + (C - B) = C -A$.\\
En appliquant 2. puis la définition de $\vv{XY}$, on a :
\ben
\item $A + (\vec{AB} + \vec{BC}) = (A + \vv{AB}) + \vv{BC} = B + \vv{BC} = C = A+ \vec{AC}.$
\item On peut conclure grâce au fait que l'application $\vec{x} \mapsto A + \vec{x}$ est injective, que $\vec{AB}+\vec{BC} = \vec{AC}$.
\een
\end{proof}

\bp
Pour tout $X \in E$ on a $\vv{XX} = \vec{0} = X - X$.
\ep
\bpf
C'est une conséquence de $X+\vv{0} = X$ et de la définition de $\vv{XX}$.
\epf

\bp
Pour tout $x,y \in E$, on a $\vv{XY} = - \vv{YX}$ ; ("Trivialement : $Y-X = -(X-Y)$.)
\ep

\bpf
Il suffit d'appliquer la relation de Chasles : $\vv{0} = \vv{XX} = \vv{XY} + \vv{YX}.$
\epf

\bp
Pour tout $X, Y, Z \in E$, on a $\vv{XY} - \vv{XZ} = \vv{ZY} ; ("trivialement" : $(Y-X) - (Z-X) = Y-Z.)
\ep
\bpf
Conséquence immédiate de la relation de Chasles.
\epf
\bn
 On peut retenir que toutes les simplifications "triviales" dans les soustractions de points sont légitimes, c'est pourquoi il n'est pas gênant d'utiliser les notations + et -.
\en

\subsection{Translations}
\bp
Pour tout $\vv{u} \in \vv{E}$, l'application $A \mapsto A + \vv{u}$ est une bijection de $E$ dans lui même appelée translation de vecteur $\vv{u}$. On la note $t_{\vv{u}}$.
\ep
\bw
Bien faire la différence avec l'application du point 3. de la définition de l'espace affine : \ding{125}~$\forall A \in E$, l'application $\vec{u} \mapsto A + \vec{u}$ est une \textbf{bijection} de $\vec{E}$ sur $E$.~\ding{126}
\ew
\bpf
Pour tout $B \in E$, on peut considérer le point $A = B + (-\vv{u})$; en utilisant 2. puis 1. on a $A+ \vv{u} = (B + (-\vv{u})) + \vv{u} = B + (-\vv{u} + \vv{u} = B + \vv{0} = B,$ donc $t_{\vv{u}}$ est surjective.\\
SI $t_{\vv{u}}(A) = t_{\vv{u}}(A') = B$, on a $A + \vv{u} = A' + \vv{u}$, donc $(A + \vv{u}) + (-\vv{u}) + (-\vv{u})$ et on en déduit $A = A + \vv{0} = A + (\vv{u} - \vv{u}) = A' - (\vv{u} - \vv{u}) = A' + \vv{0} = A'$, et $t_{\vv{u}}$ est injective.
\epf

























\end{document}
