\documentclass[11pt,a4paper]{article}
\usepackage[]{amsmath,amsthm,amssymb,amscd}
\usepackage[T1]{fontenc}
%\usepackage[utf8]{inputenc}
\usepackage[francais]{babel}



 \setlength{\topmargin}{-40pt}
 \setlength{\oddsidemargin}{-24pt}
 %\addtolength{\evensidemargin}{-78pt}
 \setlength{\marginparsep}{0pt}
 \setlength{\marginparwidth}{0pt}
 \setlength{\textwidth}{500pt}
 \setlength{\headsep}{18pt}
 \setlength{\headheight}{12pt}
 \setlength{\textheight}{730pt}



\usepackage{fancyhdr}
 
\pagestyle{fancy}
\fancyhf{}
\rhead{Université Paris Saclay\\2023-2024}
\lhead{L3MATH\\MEU302}
\cfoot{\thepage}

%               Theorem Environments
\theoremstyle{definition}
\newtheorem{exo}{Exercice}


%               Subquestions numbered with letters
\renewcommand{\theenumi}{\textbf{\alph{enumi}}}

%               Maths definitions
\renewcommand{\emptyset}{\varnothing}
\newcommand{\Sph}{\mathbb{S}}
\newcommand{\N}{\mathbb{N}}
\newcommand{\R}{\mathbb{R}}
\newcommand{\Z}{\mathbb{Z}}
\newcommand{\C}{\mathbb{C}}
\newcommand{\CP}{\mathbb{C}P}
\newcommand{\lieG}{\mathfrak{G}}
\newcommand{\liem}{\mathfrak{m}}
\newcommand{\hol}{\mathcal{H}}
\newcommand{\proj}{\mathbb{P}}
\newcommand{\calo}{\mathcal{O}}
\newcommand{\dbar}{\bar{\partial}}
\renewcommand{\d}{\partial}
\newcommand{\D}{\mathrm{D}}
\newcommand{\into}{\hookrightarrow}
\newcommand{\abs}[1]{\lvert#1\rvert}
\newcommand{\norm}[1]{\lVert#1\rVert}
\newcommand{\lie}{\mathfrak}
\newcommand{\half}{\tfrac{1}{2}}
\newcommand{\third}{\tfrac{1}{3}}
\newcommand{\esc}[1]{\langle #1 \rangle}
\newcommand{\rot}{\nabla\times}
\renewcommand{\div}{\nabla\cdot}
\newcommand{\grad}{\nabla}
\newcommand{\co}[1]{(#1_1,#1_2,#1_3)}
\newcommand{\Det}[3]{
        \begin{vmatrix}
                #1_1 & #1_2 & #1_3 \\
                #2_1 & #2_2 & #2_3 \\
                #3_1 & #3_2 & #3_3 
        \end{vmatrix}}
\newcommand{\x}{\times}
%\newcommand{\pdif}[3][]{\frac{\partial #1}{\partial #2}}
\newcommand{\pdif}[3][]{\frac{{\partial}^{#1}#2}{\partial #3}}
\newcommand{\diff}[3][]{\frac{{\partial}^{#1}#2}{\partial x_{#3}^{#1}}}
\newcommand{\pd}[2]{\frac{\partial #1}{\partial #2}}
\newcommand{\suchthat}{\;:\;}
\newcommand{\st}{\;:\;}
\newcommand{\imp}{\im}
\newcommand{\ii}{\mathbf{i}}
\newcommand{\dxdy}{dx\,dy}
\newcommand{\dxdydz}{dx\,dy\,dz}
\newcommand{\dydx}{\dfrac{dy}{dx}}
\newcommand{\n}{\mathbf{n}}
%\newcommand{\ans}[1]{[#1]}
\newcommand{\ans}[1]{}
\newcommand{\dx}{\,dx}
\newcommand{\dy}{\,dy}
\newcommand{\dz}{\,dz}
\newcommand{\du}{\,du}
\newcommand{\dv}{\,dv}
\newcommand{\dS}{\,dS}
\newcommand{\ds}{d\mathbf{s}}
\newcommand{\inner}[1]{\langle #1 \rangle}
\newcommand{\vect}{\boldsymbol{\mathfrak{X}}}
\newcommand{\SL}{\mathrm{SL}}
\newcommand{\GL}{\mathrm{GL}}
\newcommand{\SO}{\mathrm{SO}}
\newcommand{\SU}{\mathrm{SU}}
\newcommand{\U}{\mathrm{U}}
\newcommand{\OO}{\mathrm{O}}
%\renewcommand{\sl}{\mathfrak{sl}}
\newcommand{\gl}{\mathfrak{gl}}
\newcommand{\su}{\mathfrak{su}}
%\renewcommand{\u}{\mathfrak{u}}
\newcommand{\Ad}{\mathrm{Ad}}
\newcommand{\Id}{\mathrm{Id}}
%\newcommand{\vect}{\boldsymbol{\mathfrak{X}}}
\newcommand{\xra}{\xrightarrow}
\newcommand{\RPn}{\mathbb{R}\mathrm{P}^n}
\newcommand{\Sp}{\mathrm{Sp}}

\DeclareMathOperator{\gradient}{grad}
\DeclareMathOperator{\rotation}{rot}
\DeclareMathOperator{\divergence}{div}
\DeclareMathOperator{\tr}{tr}
\DeclareMathOperator{\rk}{rk}
%\DeclareMathOperator{\im}{im}
\DeclareMathOperator{\Ker}{Ker}
\DeclareMathOperator{\coker}{coker}
\DeclareMathOperator{\Hom}{Hom}
\DeclareMathOperator{\End}{End}
\DeclareMathOperator{\Div}{div}
\DeclareMathOperator{\Sign}{Sign}
\DeclareMathOperator{\sen}{sen}
\DeclareMathOperator{\senh}{senh}
\DeclareMathOperator{\tg}{tg}
\DeclareMathOperator{\tgh}{tgh}
\DeclareMathOperator{\arctg}{arctg}
\DeclareMathOperator*{\Coeff}{Coeff}
\DeclareMathOperator*{\Res}{Res}
\DeclareMathOperator{\re}{Re}
\DeclareMathOperator{\im}{Im}
\DeclareMathOperator{\intt}{int}
\DeclareMathOperator{\Alt}{Alt}
\DeclareMathOperator{\Graph}{Graph}
\DeclareMathOperator{\Card}{Card}

%\DeclareMathOperator{\exp}{exp}


\parindent 0mm
\parskip 2mm



\begin{document}
\vspace*{1pt}
\begin{center}{\bfseries
Feuille d'exercices 1: Matrices, bases et applications linéaires.
}
\end{center}


{\bfseries
Applications linéaires
}

\begin{exo}
Existe-t-il une application linéaire entre $K$-espaces vectoriels, avec $K=\R$, telle que
\begin{itemize}
\item[(a)] l'image d'une droite vectorielle soit une demi-droite (ouverte/fermée)?
\item[(b)] l'image d'un plan vectoriel soit un plan vectoriel privé de l'origine?
\item[(c)] l'image d'un plan vectoriel privé de l'origine soit une droite vectorielle?
\item[(d)] (*) que pouvez-vous dire de (b) et (c) pour $K$ un corps quelconque?
   \end{itemize}
\end{exo}

\begin{exo}
Pour les familles de vecteurs $\mathcal{F}=(e_1, e_2, \ldots e_n)$ et $\mathcal{F}'=(e_1', e_2', \ldots, e_n')$ suivantes, existe-t-il une application linéaire permettant de passer de $\mathcal{F}$ à $\mathcal{F}'$? Si oui est-elle unique?
\begin{itemize}
\item[(a)] Les vecteurs de $\R^2$ :  $e_1=(1,0), e_2=(3,2), e_3=(1,-2)$  \\
et les vecteurs de $\R^2$ : $\quad e_1'=(0,1), e_2'=(0,-2), e_3'=(-2,4)$ (dessiner les vecteurs dans ce cas);\\
\item[(b)] Les vecteurs de $\C^3$ :   $e_1=(1,0,0), e_2=(0,1,0), e_3=(0,0,1)$ \\ 
et les vecteurs de $\C^3$ : $ e_1'=(0,0,0), e_2'=(0,0,1), e_3'=(1,0,0)$;\\
\item[(c)] Les vecteurs de $\R^3$ :  $e_1=(1,0,0), e_2=(0,1,0), e_3=(0,0,1), e_4=(1,1,1)$  \\ 
 et les vecteurs de $\R_3[X]$ : $e_1'=X^3, e_2'=X^2, e_3'=X, e_4'=1$;\\
\item[(d)] Les vecteurs de $\R_4[X]$ :  $e_1=1+X^4, e_2=X^2$  \\
et les vecteurs de $\R^2$  $e_1'=(2,-1), e_2'=(3,0)$;\\
\item[(e)] Les vecteurs de $\R_3[X]$ : $e_1=2X+1, e_2=X^3+X^2+X, e_2= 2X^3+2X^2+1$  \\et les vecteurs de $\R_2[X]$ : $e_1'=1+X, e_2'=2+2X, e_3'=3+3X$;\\
\item[(f)] Les vecteurs de $\C_4[X]$, $e_1=(1,1,1, 0), e_2=(0,i,i, i), e_3=(-1,1,1,2)$ \\
et les matrices de $M_2(\C)$ $
e_1' = \begin{pmatrix}
1 & -2 \\
0 & 3
\end{pmatrix}
$, 
$e_2' = \begin{pmatrix}
0 & 2 \\
2 & 4
\end{pmatrix}
$,
$e_3' = \begin{pmatrix}
-1 & 0 \\
i & 1
\end{pmatrix}
$.
   \end{itemize}
\end{exo}



\begin{exo}
Soit $f: \R^2 \rightarrow \R^2$ l'application définie par $f(e_1)=e_1'$ et $f(e_2)=e_2'$. Dans chacun des cas suivants, déterminer graphiquement l'image des vecteurs $v_i$, donner la matrice de $f$ dans la base canonique, puis l'expression de $f(x,y)$ pour tout $(x,y)\in \R^2$.  Identifier la transformation du plan définie par $f$.
\begin{itemize}
\item[(a)] pour $e_1 = (1,0)$, $e_2=(0,1)$,  $\quad e_1' = (1,1)$, $e_2'=(-1,1)$ et $\quad v_1=(2,0)$, $v_2=(2,1)$;
\item[(b)] pour $e_1 = (1,1)$, $e_2=(0,1)$,  $\quad e_1' = (4,1)$, $e_2'=(3,1)$ et $\qquad v_1=(1,0)$, $v_2=(2,-1)$;
\item[(c)] pour $e_1 = (3,3)$, $ e_2=(0,-3)$, $\quad e_1' = (1,2)$, $e_2'=(-2,-4)$ et $\quad v_1=(1,2)$, $v_2=(2,1)$.

\end{itemize}
\end{exo}


\begin{exo}
On considère les applications linéaires $f_i : \R^2 \rightarrow \R^2$ définies pour tout $(x,y)\in \R^2$ par
$$f_1(x,y) = (y,x), \quad f_2(x,y) = \left({\frac{2x+y}{4},\frac{2x+y}{2}}\right), \quad f_3(x,y) = \left({\frac{x-\sqrt{3}y}{2},\frac{\sqrt{3}x+y}{2}}\right).$$
\begin{itemize}
\item[(a)] Représenter l'image de la base canonique pour chacune de ces applications. Préciser leur nature.
\item[(b)] Donner les matrices associées à ces applications. Ces matrices sont-elles équivalentes?
\end{itemize}
\end{exo}



\begin{exo}
Si $A$ est la matrice associée à une application linéaire $f$ dans une base quelconque, montrer que le rang de $f$ est la dimension de l'espace engendré par les colonnes de $A$.
\end{exo}



\begin{exo}
Quelles sont les classes d'équivalences des matrices de $M_{n,1}(\R)$?
\end{exo}

\begin{exo}
Parmi les matrices suivantes, lesquelles sont équivalentes?

$
A = \begin{pmatrix}
1 & 0 \\
0 & 1
\end{pmatrix}
$,
$B = \begin{pmatrix}
1 & 2 \\
2 & 1
\end{pmatrix}$,
$C = \begin{pmatrix}
1 & 2 \\
2 & 4
\end{pmatrix}$,
$D =  \begin{pmatrix}
0 & 1 \\
0 & 1  \\
\end{pmatrix}$,
$E= \begin{pmatrix}
i & 0 \\
0 & -i  \\
\end{pmatrix}$,
$
F =  \begin{pmatrix}
1 & 0 & 0 \\
0 & 1 & 0 \\
\end{pmatrix}$,
$G =  \begin{pmatrix}
1 & 2 & 3\\
1 & 2 & 3 \\
\end{pmatrix}$, \smallskip

\noindent
$H =  \begin{pmatrix}
1 & 0 &  1\\
0 & 0 & 1 \\
\end{pmatrix}$,
$I =  \begin{pmatrix}
1 & 0 \\
0 & 1 \\
0 & 0
\end{pmatrix}$,
$J =  \begin{pmatrix}
i & -1 \\
-i & 1 \\
0 & 0
\end{pmatrix}$,
$K = \begin{pmatrix}
1 & 2 & 0 \\
0 & 1 & 0 \\
0 & 0 & 0
\end{pmatrix}$,
$L =  \begin{pmatrix}
1 & 0 & 0 \\
0 & 0 & 1 \\
0 & 0 & 0
\end{pmatrix}$,
$M =  \begin{pmatrix}
1 & 0 & 0 \\
0 & 2 & 1 \\
0 & 0 & 0
\end{pmatrix}$,\smallskip

\noindent
$N =  \begin{pmatrix}
0 & 0 & 0 \\
i & -1 & 0 \\
-1 & -i & 0
\end{pmatrix}$,
$O =  \begin{pmatrix}
1 & 0 & 0 & 0\\
0 & 1 & 0 & 0 \\
0 & 0 & 0 & 0 \\
0 & 0 & 0 & 0 
\end{pmatrix}$,
$P =  \begin{pmatrix}
1 & 5 & 9 & 4\\
2 & 6 & 0 & 5 \\
3 & 7 & 0 & 6 \\
4 & 8 & 0 & 7 
\end{pmatrix}$,
$Q =  \begin{pmatrix}
1 & 5 & -5 & 4\\
2 & 6 & -6& 4 \\
3 & 7 & -7& 4 \\
4 & 8 & -8 & 4 
\end{pmatrix}$.

\end{exo}



{\bfseries
Bases et espaces vectoriels
}
\begin{exo}
Les familles de fonctions suivantes sont-elles libres?
\begin{itemize}
\item[(a)] $\{\R \rightarrow \R, x \mapsto \exp(kx)\}_{k=1, \ldots n}$ pour $n \in \N^*$;
\item[(b)] $\{\R \rightarrow \R, x \mapsto x^k\}_{k=0, \ldots n}$ pour $n \in \N^*$;
   \end{itemize}
\end{exo}

\begin{exo}
On considère l'ensemble $M_2(\C)$ des matrices $2\times2$ à coefficients dans $\C$.
\begin{itemize}
\item[(a)] Rappelez pourquoi $M_2(\C)$ est un $\C$-espace vectoriel. Quelle est sa dimension?
\item[(b)] Montrer que $M_2(\C)$ est un $\R$-espace vectoriel. Quelle est sa dimension?
\item[(c)] Réciproquement, est-ce que $M_2(\R)$ est un $\C$-espace vectoriel?
   \end{itemize}
\end{exo}



\begin{exo}
\begin{itemize}
\item[(a)] Pour $x, y, z \in \R$ calculer le déterminant 
$\begin{vmatrix}
1 & 2 & x\\
2 & 3 & y\\
3 & 4 & z
\end{vmatrix}$. 
\item[(b)] En déduire une équation cartésienne du sous-espace vectoriel $H$ de $\R^3$ engendré par les vecteurs de coordonnées 
$(1,2,3)$ et $(2,3,4)$.
\item[(c)] Montrer que $H$ est le noyau d'une forme linéaire $f \in L(\R^3,\R)$ et donner sa matrice dans les bases canoniques de $\R^3$ et $\R$.
   \end{itemize}
\end{exo}

\begin{exo}($\star$)
Soit $\{v_1, v_2, \ldots v_{n-1}\}$ une famille libre de vecteurs d'un espace vectoriel $E$ de dimension~$n$.
Montrer qu'un vecteur $u$ de $E$ appartient au sous-espace vectoriel engendré par ces vecteurs si et seulement si $\det(v_1, v_2, \ldots v_{n-1},u) = 0$.
\end{exo}

\begin{exo}
Soit $E$ un espace vectoriel de dimension $n$. On désigne par $E^*=L(E,\R)$ l'ensemble des formes linéaires sur $E$. Remarquer que $E^*$ est un espace vectoriel, préciser sa dimension et donner une base.
\end{exo}


\begin{exo}($\star$)
Montrer qu'un sous-espace vectoriel est un hyperplan si et seulement si il est le noyau d'une forme linéaire non nulle.
\end{exo}


{\bfseries
Algèbre
}
\begin{exo}
Montrer que la relation ``\textit{$A$ est équivalente à $B$}'' est une relation d'équivalence sur l'ensemble des matrices
$M_{n,p}(\R)$ 
\end{exo}

\begin{exo}


\begin{itemize}
\item[\phantom{()}]
\item[(a)] Montrer qu'une matrice de $M_n(\R)$ est une matrice de passage si et seulement si elle est inversible.
\item[(b)] Montrer que $GL_n(\R)$ est un groupe multiplicatif. Est-il commutatif?
   \end{itemize}
\end{exo}


\begin{exo}($\star$)
L'ensemble $M_{n,p}(\R)$ est-il un groupe multiplicatif ?
\end{exo}


\begin{exo}($\star$)
\begin{itemize}
\item[(a)] Montrer que la famille de matrices de $M_2(\C)$ 
$\begin{pmatrix}
1 & 0 \\
0 & 1
\end{pmatrix}$, 
$I = \begin{pmatrix}
i & 0 \\
0 & -i
\end{pmatrix}$, 
$J =\begin{pmatrix}
0 & -1\\
1 & 0
\end{pmatrix}$, 
$K =\begin{pmatrix}
0 & -i\\
-i & 0
\end{pmatrix}$
est libre.
\item[(b)] Montrer que le sous-espace vectoriel engendré par cette famille forme une algèbre.
\item[(c)] Montrer qu'il forme un corps.
   \end{itemize}
\end{exo}



\begin{exo}
Soit $P \in K[X]$ un polynôme sur un corps $K$ et $\lambda \in K$. Montrer que $P(\lambda) = 0$ si et seulement si $(X-\lambda)$ divise le polynôme $P$.
\end{exo}

\end{document}
