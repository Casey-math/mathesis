 \documentclass[10pt,a4paper]{article}
\usepackage{titlesec}
\usepackage{fullpage}
\usepackage{amsmath}
\usepackage{amssymb}
\usepackage{amsfonts}
\usepackage{amsthm}
%\usepackage{theorem}
\usepackage[english,francais]{babel}
%\usepackage[latin1]{inputenc}
\usepackage[T1]{fontenc}
\usepackage[utf8]{inputenc}
\pagestyle{empty}
\usepackage[pdftex]{graphicx}

\usepackage{tcolorbox}

%\topmargin -1cm
\textheight = 62\baselineskip
\advance\textheight by \topskip
\textwidth 480pt

\hoffset-1cm
\voffset -1cm

\def\bf#1{\textbf{#1}}
\def\it#1{\textit{#1}}
 
\newcounter{nexo}
\setcounter{nexo}{1}
\newcommand{\exo}{\medskip\noindent \bf{ Exercice \thenexo. \stepcounter{nexo}}}

\newtheorem{theo}{Th\'eor\`eme}
\newtheorem{prop}{Proposition}
\newtheorem{lemme}{Lemme}
\newtheorem{coro}{Corollaire}

\def\afaire#1{\textit{\textcolor{magenta}{#1}}}

\newcommand {\csq}{\smallskip\noindent \bf{Cons\'equence : } \rm }
\newcommand {\rem}{\smallskip\noindent \bf{Remarque : }  }
\newcommand {\defi}{\smallskip\noindent \bf{D\'efinition : } }
\newcommand {\defs}{\smallskip\noindent \bf{D\'efinitions : } }
\newcommand {\note}{\smallskip\noindent \bf{Notation : } \rm }
\newcommand {\nots}{\smallskip\noindent \bf{Notations : } \rm }
\newcommand {\dem}{\smallskip\noindent {\sc Preuve : } \rm }
\newcommand{\findem}{\hfill$\square$\par\medskip}


\def\propr#1{\smallskip\noindent \bf{Propri\'et\'e : } \it {#1} \par}
\def\props#1{\smallskip\noindent \bf{Propri\'et\'es : }\it{#1}\par}
\def\Question#1{\smallskip\noindent \bf{Question : }{\it {#1}}\par\medskip}
\def\Rappel#1#2{\smallskip\noindent \bf{Rappel : #1}\par \it{#2}}


\def\itbf{\itshape \bfseries}
\def\noi{\noindent}

\def\be{\begin{enumerate}}
\def\ee{\end{enumerate}}
\def\bi{\begin{itemize}}
\def\ei{\end{itemize}}



\def\BC{base canonique }
\def\ev{espace vectoriel }
\def\evs{espaces vectoriels }
\def\sevs{sous-espaces vectoriels }
\def\sev{sous-espace vectoriel }
\def\sep{sous-espace propre }
\def\seps{sous-espaces propres }
\def\sea{sous-espace affine }
\def\seas{sous-espaces affines }
\def\AL{application lin\'eaire }
\def\ALs{applications lin\'eaires }
\def\AA{application affine }
\def\AAs{applications affines }
\def\Aff{ \hbox{\it Aff}}
\def\vep{vecteur propre }
\def\vap{valeur propre }

\def\ssi{si et seulement si }

\def\lims{\,\overline{\lim}\;}
\def\limi{\,\underline{\lim}\;}
\def\vide{\emptyset}
\def\lims{\,\overline{\lim}\;}
\def\limi{\,\underline{\lim}\;}
\def\vide{\emptyset}
\def\vfi{\varphi}
\def\dim#1{\text{dim }(#1)}

\def\T{\mathbb{T}}
\def\N{\mathbb{N}}
\def\Q{\mathbb{Q}}
\def\R{\mathbb{R}}
\def\C{\mathbb{C}}
\def\Z{\mathbb{Z}}
\def\Bc{\mathcal{B}}
\def\Mc{\mathcal{M}}
\def\Lc{\mathcal{L}}
\def\rond{{\scriptstyle\circ}}


\def\ol#1{\overline{#1}}
\def\e#1{{\hbox{e}^{#1}}}
\def\mat#1{\begin{pmatrix}#1\end{pmatrix}}
\def\vect#1{\overrightarrow{\kern-1pt#1\kern 2pt}}
\def\card#1{{\hbox{Card}(#1)}}
 
 \DeclareMathOperator{\Id}{Id}
\DeclareMathOperator{\Tr}{Tr}
\DeclareMathOperator{\rg}{rg}

\titleformat{\section}{\normalfont\bfseries}
{\thesection.}{0.5em}{}
 

\title{Feuille 2 : Etudes des endomorphismes}

\begin{document}
\noindent L3MATH \hfill Universit\'e Paris Saclay\\
MEU302  \hfill  2023-2024\\

\begin{center}
\bf{Feuille d'exercices 2 : \'Etudes et réduction d'endomorphismes}
\end{center}
\vspace{0.2cm}


\section{Entrainement}

\exo On considère l'endomorphisme $f$ de $\R^3$ dont la matrice dans la base canonique est 
%
\begin{equation*}
 A=
 \begin{pmatrix}
    3 & 2 & -4 \\ 
    1 & 2 & -2 \\ 
    2 & 2 & -3
 \end{pmatrix}
\end{equation*}
%
\be
\item Déterminer la nature de $f$ et ses éléments caractéristiques.  
\item Déterminer une base de $\R^{3}$ dans laquelle la matrice de $f$ a une écriture plus simple.
\ee

\exo
Soit $A= \mat{5&4&2\\-6&-5&-2\\0&0&-1}$
\be
\item
Montrer qu'il existe une matrice $D$ diagonale telle que $A=PDP^{-1}$, où $P=\mat{ 1&2&1\\-1&-3&-1\\0&0&-1}$.
\item En déduire l'expression de $A^n$ lorsque $n\in \N$.
\item Quelle est la nature de l'endomorphisme associé à $A$ ?
\ee

\exo
Soit $\mathcal{B}=(e_1,e_2,e_3)$ une base de $\R^{3}$ et $f$ un endomorphisme de $\R^3$ tel que 
%
\begin{equation*}
 f(e_1) = e_1+e_3, \quad 
 f(e_1+e_2) = e_1+e_2+e_3 \quad 
 \mbox{ et } \quad 
 f(e_1+e_2+2 e_3) = e_1+e_2+e_3.
\end{equation*}
%
\be
\item Justifiez pourquoi un tel endormorphisme existe. Est-il unique?
\item Placer les vecteurs de $\mathcal{B}$  et leur image sur un dessin. Pouvez-vous deviner la nature de $f$?
\item Donner les images par $f$ des vecteurs de la base $\mathcal{B}$ à l'aide du dessin. Déduisez-en sa matrice dans la base $\mathcal{B}$.
\item Déterminer une base de $\R^{3}$ dans laquelle la matrice de $f$ soit diagonale et caractériser $f$.
\ee

\exo
Soit $(e_1,e_2,e_3)$ une base de $\R^{3}$, 

\noindent
\begin{tabular}{cc}
\begin{minipage}{10cm}
\be
\item Montrer qu'il existe une unique  $f\in L(\R^3)$ qui fixe $e_2$ et qui envoie $e_1$ sur $f_1$ et $e_3$ sur $f_3$ comme l'indique la figure ci-contre. \\ Pourquoi $f$ est-elle bijective ?
\item On note $e'_1 = \frac 1 2(e_1-e_3)$  Placer ce vecteur sur le schéma puis calculer son image. 
\item Montrer que l'ensemble des vecteurs fixes de $f$ est un plan vectoriel not\'e $\cal P$. Préciser sa base. Que peut-on en déduire sur les valeurs propres de $f$ ?
\item
Déterminer l'image de $e'_3=\frac 1 2  (e_1+e_3)$. En déduire que  $f^{2}=id$ ($f^{2}=f\circ f$). 
\item Montrer que $f$ est diagonalisable et donner dans une base qu'on choisira la matrice $A$ associ\'ee \`a $ f$. 
\ee
\end{minipage} &
\begin{minipage}{5cm}
\includegraphics[scale=0.3]{images/td2-symetrie-3d.png}
\end{minipage}
\end{tabular} 

\exo Montrer que l'application $f$ de $\mathbb R^2$ dans lui-m\^eme qui \`a $(x,y)$ associe $(2x-2y, x-y)$ est une projection. Faire une figure et pr\'eciser ses \'el\'ements caract\'eristiques (sur quel espace projette-t-on, parallèlement à quoi ?).  


\exo Soit $f : \R^3\rightarrow\R^3$ l'endomorphisme défini par
%
\begin{equation*}
 f(x,y,z) = (7x-4y-4z,6x-3y-4z,6x-4y-3z).
\end{equation*}
%
On note $\Bc=(e_1,e_2,e_3)$ la base canonique de $\R^3$.
\be
\item On pose $P=X^2-1$. Calculer $P(f)(e_i)$ pour tout $1\leq i\leq3$. En déduire que $f \rond f=id_{\R^3}$.
\item En déduire que les valeurs propres de $f$ sont dans $\{1,-1\}$.
\item Soit $A=M_{\Bc}(f)$. Calculer $P(A)$ et retrouver le résultat de la question.
\ee


\exo
Parmi les matrices suivantes, lesquelles sont semblables ? Préciser dans quel ensemble.
%
\begin{equation*}
 A=\mat{1&1&0\\
0&1&0\\
0&0&2}, \quad
B = \mat{1&0&0\\
0&1&0\\
0&0&2},\quad
C = \mat{1&0&0\\
0&2&1\\
0&0&2},\quad
D = \mat{1&0&0\\
1&1&0\\
0&0&2},\quad
E = \mat{2&1&0\\
0&2&0\\
0&0&1},
\end{equation*}
\begin{equation*}
F = \mat{1&0&i\\
0&2&0\\
0&0&1},\quad
G = \mat{1&0&i\\
0&0&1\\
0&2&0},\quad
H = \mat{1&0&i\\
0&1&0\\
0&0&2},\quad
I =  \mat{1&1&0\\
0&2&0\\
0&0&2},\quad
J = \mat{1&0&0\\
0&2&0\\
0&0&2}.
\end{equation*}


\exo Décrire une méthode qui permettrait de décider si les matrices \\
%
\begin{equation*}
 A = \mat{-2&-3&-4\\
-1&0&0\\
2&2&3}
\mbox{ et }
B=\mat{-9&72&52\\
1&-5&-4\\
-3&20&15}
\end{equation*}
%
sont semblables dans $M_3(\R)$. La mettre en {\oe}uvre dans cet exemple.


\exo Soit $A\in M_4(\R)$ définie par 
$A=\mat{-1&1&1&0\\
0&-1&0&1\\
-1&2&1&-1\\
0&-1&0&1}$. \be
\item Calculer $A^2$.  
\item Déterminer la forme réduite de Jordan $J$ de $A$ et une matrice inversible $P$ telle que $A=P J P^{-1}$.
\ee


\exo Soit $f\in L(\R^5)$ nilpotent tel que $dim Ker f = 2$ et $dim Ker f^2 = 4$. 
\be
\item Démontrer que l'indice de nilpotence de $f$ vaut 3.
\item Dessiner le tableau de Young associé à $f$
\item En déduire la forme de Jordan associée à $f$ en précisant dans quelle base on se place.
\ee

\exo
On considère $f\in L(\R^n)$  un endomorphisme nilpotent  associé à chacun des tableaux de Young suivants. \\
\begin{minipage}{8cm}
Donner dans  chaque cas : 
\be
\item La valeur de $n$ et l'indice de nilpotence de $f$.
\item La matrice de Jordan associée à $f$.
\item Le rang de $f$ et la dimension de $Im f \cap Ker f $. 
\ee
\end{minipage}
\begin{minipage}{8cm}\quad
\includegraphics[scale=0.17]{images/tableau-young.png}
\end{minipage}



\exo Soit $A\in M_n(\R)$ définie par 
$A=\mat{1&\cdots&1\\\vdots& \ddots& \vdots\\1&\cdots&1}$. \be
\item Déterminer le rang de cette matrice. 
\item En déduire l'ensemble de ses valeurs propres (son spectre).
\item Montrer que $A$ est diagonalisable. 
\item Montrer que $A^2-tr(A)A=0$ sans calculer $A^2$. 
\ee


\exo 
Pour quelles valeurs de $\theta$ les endomorphismes canoniquement associés aux matrices
%
\begin{equation*}
 A_{\theta}= \mat{\cos(\theta)&-\sin(\theta)\\ \sin(\theta)&\cos(\theta)}
 \mbox{ et }
B_{\theta}= \mat{\cos(\theta)&-\sin(\theta)&0\\ \sin(\theta)&\cos(\theta)& 0\\0 & 0 & 1}
\end{equation*}
%
admettent-ils des sous-espaces stables? Commenter.


\section{Approfondissement}

\exo Soit $f\in L(E)$ et $E$ un $\R$-\ev de dimension $n\geq 2$. 
\be
\item Montrer que $f$ admet toujours des plans stables. 
\item Montrer que si $n$ est impair, alors $f$ admet au moins une direction stable. 
\item Si $E$ un $\C$-\ev, que peut-on dire de plus ?
\ee

\exo Soit $n \in \N^*$, et $f$ l'endomorphisme de $\R^n$ associé à la matrice
%
\begin{equation*}
 J = \mat{0 & 1 & 0 & \ldots & 0\\ 
0 & 0 & 1 & \ldots & \\
   &     & 0 & \\
 \vdots   &     &  &\ddots & 1 \\  
 0 &\ldots & & & 0 }.
\end{equation*}

\be
\item Calculer $J^k$ pour $k \in \N^*$.
\item ($\star$) Déterminer tous les sous-espaces vectoriels stables par $f$.
\ee

\exo Soit $A= \mat{2&-1&2\\1&0&2\\1&-1&3}$
\be 
\item Vérifier que $P = (X-1)(X-3)$ est un polynôme annulateur pour $A$ (c'est à dire $P(A)=0$).
\item Soit $R_n = a_n X + b_n$ le reste de la division euclidienne de $X^n$ par $P$. Déterminer $a_n$ et $b_n$.
\item En déduire une expression de $A^n$.
\ee

\exo (Polynômes)
\be
\item Effectuer la division euclidienne de $P=X^4+2X^3-X+6$ par $Q=X^3-6X^2+X+4$ dans~$\R[X]$.
\item Effectuer la division euclidienne de $P=iX^3-X^2+(1-i)$ par $Q=(1+i)X^2-iX+3$ dans~$\C[X]$.
\item ($\star$) Trouver deux polynômes $U$ et $V$ de $\R[X]$ tels que $AU+BV=1$, où $A=X^7-X-1$ et $B=X^5-1$.
\ee


\exo ($\star$)
On considère l’espace vectoriel $\mathcal{F}(\R,\R)$ des fonctions de $\R$ dans $\R$ et soit $T: \mathcal{F}(\R,\R) \rightarrow \mathcal{F}(\R,\R)$ l'application 
définie par : si $f \in \mathcal{F}(\R,\R)$, alors $T(f) \in \mathcal{F}(\R,\R)$ est la fonction définie par  
%
\begin{equation*}
 \forall x \in \R, \quad T(f)(x) = f(-x).
\end{equation*}
%
Montrer que $T$ est une symétrie de $\mathcal{F}(\R,\R)$ et donner ses éléments caractéristiques.


\section{Propriétés générales}

\exo  Soit $f$ un endomorphisme de $E$. Montrer que les sous-espaces propres de $f$ associés à des valeurs propres distinctes sont en somme directe.

\exo (Cours) Montrer que $\lambda$ est valeur propre de $f$ \ssi $\lambda$ est racine du polynôme caractéristique de $f$.

\exo (Cours)
 Existe-t-il des endomorphismes nilpotents vérifiant $Im f \bigoplus Ker f =E$ ?

\exo
Soit $E$ un espace vectoriel de dimension $n$. Quelle est la dimension de $L(E)$? Montrer que pour tout endomorphisme $f$ de $E$ il existe un polynôme $P$ tel que 
tel que $P(f)=0$ (sans le théorème de Cayley--Hamilton). Montrer que son degré est toujours inférieur à $n^2$.


\exo Soit $A,B$ des matrices de $\Mc(\R)$. Montrer qu'elles sont semblables sur $\R$ \ssi elles sont semblables sur $\C$.


\exo (Cours)
Soient $p$ et $q$ deux projections équivalentes. Sont-elles nécessairement semblables?


\exo (Cours) Montrer qu'une \AL est une homothétie \ssi sa matrice dans une base est de la forme $\lambda I_n$.


\exo Montrer qu'une \AL du plan qui pr\'eserve 3 directions 2 \`a 2 distinctes est une homoth\'etie .


\exo Donner un exemple de deux matrices $A$ et $B$ telles que $AB = 0$ mais $B A$ est non nulle. Que pouvez-vous dire des noyaux et images des endomorphismes associés?




%\exo($\star$)
%Soit $E$ un $\R$-espace vectoriel de dimension $n \in \N^*$ et $f \in L(E)$ un endomorphisme nilpotent d'indice de nilpotence $k$.
%Montrer que 
%\[k = n \iff \rg(f) = n-1.\]


\exo
\'Etant donnée une matrice $A \in M_n(\R)$, on veut définir l'exponentielle \[e^{A} = \lim_{n \rightarrow +\infty} \sum_{k=0}^{n} \frac{1}{k!} A^k.\]
\be
\item Montrer que si $A = D  =  \mat{
\lambda_1 & 0               & \ldots & 0 \\
      0         & \lambda_2& \ddots & \vdots \\
    \vdots  & \ddots       &  \ddots  & 0 \\
0              & \ldots         & 0         &\lambda_{n}}$ est diagonale, alors 
$e^{A} =  \mat{
e^{\lambda_1}& 0               & \ldots & 0 \\
      0         & e^{\lambda_2}& \ddots & \vdots \\
    \vdots  & \ddots       &  \ddots  & 0 \\
0              & \ldots         & 0         &e^{\lambda_n}}$. 
En déduire l'existence de $e^{A}$ lorsque $A$ est diagonalisable.

\item Montrer que si $A=N$ est nilpotente, alors il existe un polynôme $P = \R[X]$ tel que $e^{A} =P(A)$.

\item On suppose que $A = D+N$ avec $DN = ND$. Prouvez que dans ce cas $e^{A}$ est bien définie et que l'on a $e^{A} = e^{D} e^{N}  = e^{N} e^{D}$.

\item  En déduire l'expression de $e^A$ dans le cas général.
\ee


\section{Pour réfléchir un peu plus...}

\exo(Cours) Soient $f, g \in L(E)$ deux endomorphismes d'un espace vectoriel de dimension finie. Montrer que si $f$ et $g$ commutent et si ils sont tous deux diagonalisables, alors ils admettent une base co-diagonalisante (c'est à dire une base dans laquelle leurs matrices sont toutes deux diagonales).


\exo Montrer que si un endomorphisme $f$ est diagonalisable, alors $f^k$ est aussi diagonalisable. Que pensez-vous de la réciproque ?

\exo ($\star$) Montrer que l'ensemble $H(E)$ des homothéties non nulles est un sous-groupe commutatif de $GL(E)$ pour la composition. 
Montrer qu'il est le centre de $GL(E)$, c'est à dire l'ensemble des endomorphismes qui commutent avec toutes les éléments de $GL(E)$.

\exo($\star \star$)
Soit $f \in L(E)$ un endomorphisme d'un $\R$-espace vectoriel $E$ de dimension finie tel que $f^2 = -\Id$. Montrer qu'il existe une base dans laquelle la matrice de $f$ 
s'écrit par blocs
\[ \mat{R& 0 & \ldots & 0\\
0 & R& \\
   &    &\ddots & 0 \\
0&\ldots&0&R} \mbox{ avec } 
R=\mat{0&-1\\
1&0} \in M_2(\R).\]
(\emph{Indication: on pourra d'abord montrer que la dimension de $E$ est paire}).


\exo($\star \star$) \emph{Polynôme compagnon.}
Soit $P \in \R[X]$ un polynôme unitaire de la forme $P(X) = X^n+ \sum_{i=0}^{n-1} a_i X^i$. 
On appelle matrice compagnon associée à $P$ la matrice 
\[C =  \mat{
0 & 0 & \ldots & 0 & -a_0\\
1 & 0&   \ldots & 0&-a_1\\
0 & 1&  \ddots  &  \vdots  &\vdots\\
\vdots & \ddots & \ddots & 0 & -a_{n-2}\\
0& \ldots & 0 &1&-a_{n-1}}.\]
\be
\item
Montrer que l'application linéaire associée à $C$ dans la base canonique est déterminée par $f(e_i) = e_{i+1}$ pour $i<n$ et $f(e_n) = - \sum_{i=0}^{n-1} a_i e_{i+1}$.
\item 
Montrer que $(-1)^n P$ est le polynôme caractéristique associé à $f$.
\item
Déterminer le polynôme minimal associé à $f$. En déduire que $f$ est diagonalisable si et seulement si $P$ n'a que des racines simples.
\item 
Soit $\lambda$ une racine de $P$. Montrer que $|\lambda| \leq R$ où $R = \max\{|a_0|,|a_1|+\ldots+|a_{n-1}|+1\}$ (\emph{indication: on pourra considérer un 
vecteur propre associé à $\lambda$}).
\ee

\exo($\star \star$)
Soit$f \in L(E)$ un endomorphisme d'un $\R$-espace vectoriel $E$ de dimension finie dont le polynôme minimal est $\mu_f(X) = \prod_{i=1}^s(X-\lambda_i)^{m_i}$. 
\'Etant donné un polynôme $P \in \R[X]$, on cherche à montrer que l'endomorphisme $P(f)$ est diagonalisable si et seulement si 
\[\forall i\in \{1, \ldots s\}, \forall k\in \{1, \ldots m_i-1\}, \; P^{(k)}(\lambda_i) = 0.\]
 \be
\item Montrer cette équivalence pour un endomorphisme $f$ nilpotent d'indice $m$ 
(\emph{indication: on pourra d'abord montrer que la famille $f, f^2, \ldots, f^{m-1}$ est libre}).
\item En déduire le résultat dans le cas où $f- \lambda \Id$ est nilpotent.
\item Conclure dans le cas général (\emph{indication: on pourra utiliser le théorème des noyaux}).
\ee

\end{document}




\begin{exercice}~(avec geogebra)
Pour tout $m=(x,y)$ de $\R^{2}$, on d\'efinit  $f(m)=(5x+4y-4, -6x-5y+6)$. On note $(O,\vect{i},\vect{j})$ le rep\`ere canonique de $\R^{2}$.
 \begin{enumerate}
 \item Montrer que $f$ est une sym\'etrie affine et d\'eterminer ses \'el\'ements caract\'eristiques.
 \item Soit $t$ une translation de vecteur $u=\overrightarrow i-\overrightarrow j$.  D\'eterminer la nature de $t\circ f$ et donner ses caract\'eristiques g\'eom\'etriques. Illustrer ces transformations sur une figure Geogebra.
\end{enumerate}
\end{exercice}

\end{document}
\begin{exercice}
On suppose que $f$ une sym\'etrie affine par rapport \`a $F$ parall\`element \`a  $G$. 
\begin{enumerate}  
\item À quelle condition sur $u$, l'application $t\circ f$ est-elle encore une sym\'etrie affine ? A quelle condition $t\circ f=f\circ t$ ?
 \item Soit $g$ une application affine dont l'application lin\'eaire associ\'ee est une sym\'etrie vectorielle. Montrer qu'il existe une unique sym\'etrie affine $s$ et une unique translation $t$ telles que : $g=s\circ t=t\circ s$.
\end{enumerate}
\end{exercice}


\end{document}
