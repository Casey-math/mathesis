\documentclass[12pt,a4paper]{article}
\usepackage[left=2cm,right=2cm,top=2cm,bottom=2cm]{geometry}
\usepackage{amsmath}
\usepackage{amsthm}
\usepackage{thmtools}
\usepackage{amsfonts}
\usepackage{amssymb}
\usepackage[utf8]{inputenc}
\usepackage[T1]{fontenc}
\usepackage[francais]{babel}
\usepackage{mathtools}   % loads »amsmath
\usepackage[dvipsnames]{xcolor}
\usepackage{ulem}
\usepackage{enumitem}
\usepackage{pifont}
\usepackage[most]{tcolorbox}
\usepackage[scr=rsfs]{mathalpha}
\usepackage{esvect} %vecteur avec \vv
\usepackage{dsfont}


\usepackage{calc}
\usepackage[nothm]{thmbox}



%Raccourcies
\newcommand{\N}{\mathbb{N}}
\newcommand{\1}{\mathds{1}}
\newcommand{\PP}{\mathds{P}}
\newcommand{\Z}{\mathbb{Z}}
\newcommand{\E}{\mathds{E}}
\newcommand{\K}{\mathbb{K}}
\newcommand{\Q}{\mathbb{Q}}
\newcommand{\R}{\mathbb{R}}
\newcommand{\C}{\mathbb{C}}
\newcommand{\ie}{\textit{i.e }}
\newcommand{\A}{\mathscr{A}}
\newcommand{\Aa}{\mathcal{A}}
\newcommand{\Mm}{\mathcal{M}}
\newcommand{\Chi}{\mathcal{X}}
\newcommand{\Ll}{\mathcal{L}}
\newcommand{\Tup}{\bigtriangleup}
\newcommand{\Tdn}{\bigtriangledown}
\newcommand{\bps}{\langle}
\newcommand{\eps}{\rangle}
\newcommand{\pv}{\wedge}
\newcommand{\doigt}{\ding{43} ~}
\newcommand{\cray}{\ding{46} ~}


% Définir une commande pour la boîte encadrée
\newcommand{\todo}[1]{%
    \begin{tcolorbox}[colback=red!10!white, colframe=red!75!black, title=To-do :]
        #1
    \end{tcolorbox}%
}
%\begin{tcolorbox}[colback=red!10!white, colframe=red!75!black, title=Ma Boîte Encadrée]
%    Ceci est un exemple de texte à l'intérieur de la boîte.
%\end{tcolorbox}





\def\BC{base canonique }
\def\ev{espace vectoriel }
\def\evs{espaces vectoriels }
\def\sevs{sous-espaces vectoriels }
\def\sev{sous-espace vectoriel }
\def\sep{sous-espace propre }
\def\seps{sous-espaces propres }
\def\sea{sous-espace affine }
\def\seas{sous-espaces affines }
\def\AL{application lin\'eaire }
\def\ALs{applications lin\'eaires }
\def\AA{application affine }
\def\AAs{applications affines }
\def\Aff{ \hbox{\it Aff}}
\def\vep{vecteur propre }
\def\vap{valeur propre }
\def\ssi{si et seulement si }

\newcommand{\dessous}[2]{\underset{#1}{#2}}

\def\bar#1{\overline{#1}}
\def\e#1{{\hbox{e}^{#1}}}
\def\mat#1{\begin{pmatrix}#1\end{pmatrix}}
\def\vect#1{\overrightarrow{\kern-1pt#1\kern 2pt}}
\def\card#1{{\hbox{Card}(#1)}}
\def\bin(#1,#2){ \left(\!\begin{smallmatrix} #2 \\ #1 \end{smallmatrix}\!\right)}
\def\Aff{ \hbox{\it Aff}}
\def\lims{\,\overline{\lim}\;}
\def\limi{\,\underline{\lim}\;}
\def\vide{\emptyset}
\def\lims{\,\overline{\lim}\;}
\def\limi{\,\underline{\lim}\;}
\def\vide{\emptyset}
\def\vfi{\varphi}
\def\dim#1{\text{dim }(#1)}

\def\ben{\begin{enumerate}}
\def\een{\end{enumerate}}
\def\bi{\begin{itemize}}
\def\ei{\end{itemize}}

\DeclareMathOperator{\Id}{Id}
\DeclareMathOperator{\Tr}{Tr}
\DeclareMathOperator{\rg}{rg}

\definecolor{rltred}{rgb}{0.75,0,0}
	\definecolor{rltgreen}{rgb}{0,0.5,0}
	\definecolor{oneblue}{rgb}{0,0,0.75}
	\definecolor{marron}{rgb}{0.64,0.16,0.16}
	\definecolor{forestgreen}{rgb}{0.13,0.54,0.13}
	\definecolor{purple}{rgb}{0.62,0.12,0.94}
	\definecolor{dockerblue}{rgb}{0.11,0.56,0.98}
	\definecolor{freeblue}{rgb}{0.25,0.41,0.88}
	\definecolor{myblue}{rgb}{0,0.2,0.4}

%Note
\newenvironment{note}{\par\medskip\noindent\begin{tabular}{l|p{\linewidth-8cm}}
\ding{46}& \color{black}}
{\end{tabular}\par\medskip}
\def\bn{\begin{note}}
\def\en{\end{note}}


%Attention
\def\war{\color{red}\medskip\begin{thmbox}[M]{\ding{39}\textbf{Attention :}}\noindent\color{black}}
\def\endwar{\end{thmbox}}
\def\bw{\begin{war}}
\def\ew{\end{war}}


%Style définition
\declaretheoremstyle[
  headfont=\color{RoyalBlue}\normalfont\bfseries,
  bodyfont=\color{NavyBlue}\normalfont,
]{colored}
\declaretheorem[
  style=colored,
  name=Définition,
]{df}
\def\bd{\begin{df}}
\def\ed{\end{df}}

%Style Proposition
\declaretheoremstyle[
  headfont=\color{BrickRed}\normalfont\bfseries,
  bodyfont=\color{black}\normalfont,
]{coloredProp}
\declaretheorem[
  style=coloredProp,
  name=Proposition,
]{prop}
\def\bp{\begin{prop}}
\def\ep{\end{prop}}

%Style exo
\declaretheoremstyle[
  headfont=\color{orange}\normalfont\bfseries,
  bodyfont=\color{black}\normalfont,
]{coloredexo}
\declaretheorem[
  style=coloredexo,
  name=Exercice,
]{exo}
\def\be{\begin{exo}}
\def\ee{\end{exo}}

%Règle soulignée
\def\regle{\color{blue}\begin{thmbox}[M]{\textbf{Règle :}}\noindent\color{blue}}
\def\endregle{\end{thmbox}}
\def\br{\begin{regle}}
\def\er{\end{regle}}

%Démonstration
\let\oldproof\proof
\renewcommand{\proof}{\color{olive}\oldproof}
\def\bpf{\begin{proof}}
\def\epf{\end{proof}}

%\theoremstyle{definition}
%\newtheorem{prop}{Proposition}
%\newtheorem{exo}{Exercice}
%\newtheorem{df}{Définition}

\newcommand{\bpropo}{
    \begin{tcolorbox}[colback=Yellow!10!Yellow, colframe=red!75!black]
    \bp
}
\newcommand{\epropo}{
    \ep
    \end{tcolorbox}}

%Solution
\newcommand{\solution}[1]{\par\noindent\textbf{\color{OliveGreen}Solution :} \textcolor{OliveGreen}{#1}}

%Titre
%\title{Espaces affines, notes   }
\author{$\mathcal{F.J}$}
%\date{2023-2024}


\title{Chapitre 1 - Calcul Matriciel}

\begin{document}
\maketitle
\section{Produits scalaires réels et complexes}
\subsection{Cas réel}
\bd
On se place dans un espace vectoriel réel ($\R$), appelons $E$. \\
Un produit scalaire (PS) sur $E$ est une forme bilinéaire symétrique définie positive, \ie une application $p : E \times E \to \R$ vérifiant les trois propriétées suivantes : 
	\ben 
		\item Bilinéarité ;
		\item Symétrie ; 
		\item Positivité.
	\een
\ed
\medskip
\bn
Le nom de produit scalaire évoque à la fois la bilinéarité et le fait que ses valeurs soient "scalaires", \ie appartiennent au corps de base (ici $\R$) de l'espace vectoriel $E$.
\en
\medskip
Il est intéressant de se rappeler qu'un produit scalaire est entièrement déterminé par sa forme quadratique associée ; si $p$ est le produit scalaire sur $E$, on a $P : E \to \R$ sa forme quadratique, définie par : $P(x) = p(x,x), ~ (x \in E).$ En utilisant la bilinéarité et la symététrie on se ramène par "polarisation" au produit scalaire à partir de la forme quadratique.
\medskip

Une conséquence de la \textbf{positivité} du produit scalaire, est l'inégalité de Cauchy-Schwarz :
\bp[Cauchy-Schwarz]
Considérons un produit scalaire sur un espace vectoriel réel $E$. Alors, pour tout couple $(x,y)$ d'éléments de $E$, on a : $$p(x,y)^2 \leq p(x,x)p(y,y);$$
On a égalité si et seulement si $(x,y)$ est lié.
\ep
\begin{proof}
Une preuve simple consiste à prendre un scalaire et à développer : $$p(x+\lambda y, x + \lambda y).$$ Ne pas oublier le cas d'égalité !
\end{proof}

\bn
De l'inégalité de Cauchy-Schwarz, on déduit l'existence d'une norme canoniquement associée à un produit scalaire :
\en
\bp[Norme canonique] 
Soit $p$ un produit scalaire sur un espace vectoriel réel $E$. L'application $$\displaystyle x \mapsto N(x) = \sqrt{p(x,x)}$$
est une norme sur $E$.
\ep

\bd
\ben
Soit $E$ un \ev réel :
\item Si $E$ est  muni d'un produit scalaire $p$, on dit que $(E,p)$ est un \textbf{espace préhilbertien réel}.
\item Si $E$ est de dimension finie, on dit que $(E,p)$ est un \textbf{espace euclidien}.
\een
\ed

\bp[Théorème de Pythagore]
Soit $E$ un espace préhilbertien réel. \textbf{Deux} éléments $x,y \in E$ sont orthogonaux si et seulement si : $$\|x+y\|^2 = \|x\|^2 + \| y \|^2.$$
\ep
\subsection{Cas complexe}
\bd
On considère cette fois le corps des complexes : \\
Soit $E$ un \ev sur le corps $\C$. On appelle produit scalaire hermitien sur $E$ une forme : $$ p : E \times E \to \C, ~ (x,y) \mapsto \bps x \mid y \eps_{\C}$$
\ben
\item $p$ est $\C$-linéaire en $y$ (parfois en $x$) : $\bps x \mid \lambda y_1 + \alpha y_2 \eps = \lambda \bps x \mid y_1 \eps + \alpha \bps x \mid y_2 \eps$ ; 
\item $p$ possède la symétrie hermitienne : $\bps x \mid y \eps = \bar{\bps y \mid x \eps }$ ;
\item $p$ est définie positive : $\bps x \mid x \eps > 0$ pour tout $x \ne 0$.
\een
\ed
\bw 
Il résulte de cette définition que le produit scalaire hermitien est $antilinéaire$ (on dit aussi $\C$ - linéaire par rapport à la première variable.
\ew
\bw
Sur $E = \C^n$ le produit scalaire hermitien "standard" est : $\displaystyle \sum_{j=1}^n \bar{z}_j t_j$ \big( \textbf{ou} parfois : $\displaystyle \sum_{j=1}^n z_j \bar{t}_j$ \big)
\ew
\bp[Inégalité de Cauchy-Schwarz]
Soit $\bps.\mid.\eps_{\C}$ un produit scalaire hermitien sur un espace vectoriel complexe $E$. On a alors, pour tout couple $(x,y)$ d'éléments de $E$, 
$$| \bps x \mid y \eps | ^2 \leq \bps x \mid x \eps \times \bps y \mid y \eps$$\\
Avec égalité si et seulement si $(x,y)$ est lié.
\ep
\begin{proof}
	idem que dans le cas réel.\\
	Toujours bien penser au cas d'égalité.
\end{proof}

\bp[Norme induite]
Soit $E$ un espace vectoriel complexe et soit  $\bps.\mid.\eps_{\C}$ un produit scalaire hermitien sur $E$. L'application $$x \mapsto \|x\| = \sqrt{\bps x \mid x\eps_{\C}}$$
\ep
\begin{proof}
	\ben
		\item Comme $\bps x \mid x\eps_{\C} = 0$ entraîne $x=0$, on a $\|x\|=0$ si et seulement si $x=0$ ;  
		\item Comme $\bps x \mid x\eps_{\C} = |\lambda |^2\bps x \mid x\eps_{\C}$, on a $\|\lambda x \| = |\lambda| \|x\|$ ; 
		\item Enfin on a $\|x + y \| ^2 = \bps x+y \mid x+y \eps_{\C} = \bps x \mid x\eps_{\C} + \bps y \mid y\eps_{\C} + \bps x \mid y\eps_{\C}\bps y \mid x\eps_{\C} = \|x\|^2 + \|y\| ^2 + \bps x \mid y\eps_{\C}\bps y \mid x\eps_{\C} = \|x\|^2 + \|y\|^2 + \bps x \mid y\eps_{\C}+\bar{\bps x \mid y\eps_{\C}} = \|x\|^2 + \| y \| ^2 + 2Re(\bps x \mid y\eps_{\C})$, on a $|Re(\bps x \mid y\eps_{\C})| \leq \bps x \mid x\eps_{\C}$, d'où, par Cauchy-Schwarz on a $\|x+y\|^2 \leq \|x\|^2 + \|y\|^2 + 2 \|x\| \|y\| \leq \|x\|^2 +\|y\|^2$.
	\een
\end{proof}

\subsection{Matrice et produit scalaire}
\bd
Donnons quelques définitions de matrice :
\ben
	\item Soit $A \in \Mm_n(\R)$ on définie sa \textbf{transposée} :  $(A^t) \in  \Mm_n(\R)$ par $(A^t)_{ij} = A_{ij}$, pour tous $i,j \in\{1,...,n\}$.
	\item Soit $A \in \Mm(\C)$ on définie son \textbf{adjointe} $A^* \in \Mm_n(\C)$ par $A^*_{i,j} = \bar{A_{i,j}}$ pour tous $i,j \in \{1,...,n\}$.
	\item Soit $A \in \Mm_n(\R)$, on dit que $A$ est \textbf{symétrique} si $A = A^t$ (ou encore $A = A^*$).
	\item Soit $A \in \Mm_n(\K)$, on dit que $A$ est \textbf{orthogonale} (ou unitaire) si $A^{-1} = A^*$.
	\item Soit $A \in \Mm_n(\K)$, on dit que $A$ est \textbf{normale} si $AA^* = A^*A$.
	\item Soit $A \in \Mm_n(\C)$, on dit que $A$ est auto-adjointe ou \textbf{hermitienne} si $A = A^*$. 
	\item Soit $A \in \Mm_n(\R)$, on dit que $A$ est \textbf{définie positive} si $x^tAx > 0, ~ \forall x \in \R^n$ et $x\ne 0$.
	\item Soit $A \in \Mm_n(\R)$, on dit que $A$ est \textbf{positive} si $x^tAx \geq 0, ~ \forall x \in \R^n$.
	\item Soit $A \in \Mm_n(\C)$, on dit que $A$ est \textbf{définie positive} si $\bar{x}^tAx > 0, ~ \forall x \in \C^n$ et $x \ne 0$.
	\item Soit $A \in \Mm_n(\C)$, on dit que $A$ est \textbf{positive} si $\bar{x}^t A x \geq 0, ~ \forall x \in \C^n$.
\een
\ed

\medskip
\bp[Caractérisation de la transposée par le produit scalaire réel]
Soit $\bps.|.\eps$ le produit scalaire euclidien sur $\R^n$. Alors on a :\\
$$\bps Au |v \eps = \bps u |A^t v \eps, ~ \forall u,v \in \R^n.$$
$A^t$ est unique.
\ep

\medskip
\bp[Caractérisation de l'adjointe par le produit scalaire hermitien ($\C$)]
Soit $\bps.|.\eps_{\C}$ le produit scalaire hermitien canonique sur $\C^n$. Alors on a :
$$\bps Au |v \eps_{\C} = \bps u | A^* v \eps_{\C}, ~ \forall u,v \in \C^n.$$
$A^*$ est unique.
\ep

\medskip
\bp[Cas réel]
Soit $A \in \Mm_n(\R)$. Alors $A$ est \textbf{symétrique} si et seulement si $$\bps Au |v \eps = \bps u |A v \eps, ~ \forall u,v \in \R^n.$$
\ep
\medskip
\bp[Cas complexe]
Soit $A \in \Mm_n(\C)$. Alors $A$ est \textbf{hermitienne} si et seulement si : $$\bps Au |v \eps_{\C} = \bps u | A v \eps_{\C}, ~ \forall u,v \in \C^n.$$
\ep
\newpage
\section{Réduction des matrices}
\subsection{Théorie spéctrale des matrices}
On suppose ici que les matrices sont carrées et qu'elles vivent dans $\C$ ou $\R$.
\bd
Quelques définitions à savoir : 
\ben
	\item Soit $A \in \Mm_n(\C)$. Le polynôme caractéristique de $A$ est défini sur $\C$ par $P_A(\lambda) = det (A - \lambda I )$, c'est un polynôme de degré égal à $n$, il admet donc $n$ racines dans $\C$, les racines sont appelé \textbf{valeurs propres} de $A$. La multiplicité d'une valeur propre est sa multiplicité en tant que racine de $P_A(\lambda)$.
	\item Soit $\lambda$ une valeur propre de $A$. On appelle \textbf{sous-espace propre} associé à la valeur propre \lambda, et on note $E_{\lambda}$, le \sev défini par $E_{\lambda} = Ker(A-\lambda I_n)$.
	\item Soit $A \in \Mm_n(\C)$. On appelle  \textbf{rayon spectral} de $A$, et on note $\rho(A)$, le maximum des modules des valeurs propres de $A$ : $\displaystyle \rho(A) := \max_{\lambda \in Sp(A)} |\lambda |$
\een
\ed

\be 
Déterminer la nature des valeurs propres d'une matrice hermitienne
\ee

\bp[Cayly-Hamilton]
Soit $P_A(\lambda) = det(A - \lambda I_n)$, le polynôme caractéristique de $A$. On a $$P_A(A) = 0.$$
\ep

\bp
Soit $A, B \in \Mm_n(\C)$ deux matrices semblables (\ie qu'il existe une matrice $P \in \Mm_n(\C)$ inversible telle que $A = PBP^{-1}$.\\
Alors les valeurs propres de $A$ et de $B$ sont les mêmes \ie:  $$Sp(A) = Sp(B).$$
\ep
\bp[Nature des valeurs propres d'une matrice hermitienne (\ie $A = A^*$)]
Soit $A \in \Mm_n(\R)$ \textbf{symétrique} ou bien $A \in \Mm_n(\C)$ \textbf{hermitienne}. \\
On a : $$Sp(A) \subset \R.$$
\ep
\subsection{Trigonalisation des matrices}
Il existe des classes de matrices particulièrement simples. Par exemple les matrices triangulaires supérieures \ie telles que $a_{i,j} = 0$ si $i>j$. Réduire une matrice c'est la transformer par un changement de base en une de ces formes particulières.
\medskip
\bd
Une matrice $A$ est dite triangulisable s'il existe une matrice inversible $P$ et une matrice triangulaire $T$ telle que $$A = PTP^{-1}.$$
\ed

\bn
On dit que les matrices $A$ et $T$ sont \textbf{semblables}.
\en
\medskip
\bp
Toute matrice $A \in \Mm_n(\C)$ est triangulisable.
\ep
\begin{proof}
	Par récurrence sur $n$.
\end{proof}

\bp[Factorisation de Schur]
Pour toute matrice $A \in \Mm_n(\C)$ il existe une matrice \textbf{unitaire} $U$ (\ie $U^{-1} = U^*$), telle que $U^{-1}AU$ soit triangulaire.
\ep
\end{document}
