\documentclass[12pt,a4paper]{article}
\usepackage[left=2cm,right=2cm,top=2cm,bottom=2cm]{geometry}
\usepackage{amsmath}
\usepackage{amsthm}
\usepackage{thmtools}
\usepackage{amsfonts}
\usepackage{amssymb}
\usepackage[utf8]{inputenc}
\usepackage[T1]{fontenc}
\usepackage[francais]{babel}
\usepackage{mathtools}   % loads »amsmath
\usepackage[dvipsnames]{xcolor}



%Raccourcies
\newcommand{\N}{\mathbb{N}}
\newcommand{\Z}{\mathbb{Z}}
\newcommand{\Q}{\mathbb{Q}}
\newcommand{\R}{\mathbb{R}}
\newcommand{\C}{\mathbb{C}}
\newcommand{\Aa}{\mathcal{A}}
\newcommand{\Tup}{\bigtriangleup}
\newcommand{\Tdn}{\bigtriangledown}

\definecolor{rltred}{rgb}{0.75,0,0}
	\definecolor{rltgreen}{rgb}{0,0.5,0}
	\definecolor{oneblue}{rgb}{0,0,0.75}
	\definecolor{marron}{rgb}{0.64,0.16,0.16}
	\definecolor{forestgreen}{rgb}{0.13,0.54,0.13}
	\definecolor{purple}{rgb}{0.62,0.12,0.94}
	\definecolor{dockerblue}{rgb}{0.11,0.56,0.98}
	\definecolor{freeblue}{rgb}{0.25,0.41,0.88}
	\definecolor{myblue}{rgb}{0,0.2,0.4}


%Style définition
\declaretheoremstyle[
  headfont=\color{blue}\normalfont\bfseries,
  bodyfont=\color{blue}\normalfont,
]{colored}
\declaretheorem[
  style=colored,
  name=Définition,
]{df}

%Style Proposition
\declaretheoremstyle[
  headfont=\color{black}\normalfont\bfseries,
  bodyfont=\color{black}\normalfont,
]{coloredProp}
\declaretheorem[
  style=coloredProp,
  name=Proposition,
]{prop}

%Style exo
\declaretheoremstyle[
  headfont=\color{orange}\normalfont\bfseries,
  bodyfont=\color{black}\normalfont,
]{coloredexo}
\declaretheorem[
  style=coloredexo,
  name=Exercice,
]{exo}

\let\oldproof\proof
\renewcommand{\proof}{\color{olive}\oldproof}

%\theoremstyle{definition}
%\newtheorem{prop}{Proposition}
%\newtheorem{exo}{Exercice}
%\newtheorem{df}{Définition}

%%%%%%%%%%%%%%%%%%%%%%%%%%%%%%%%%%%solution%%%%%%%%%%%%%%%%%%
\newcommand{\solution}[1]{\par\noindent\textbf{\color{OliveGreen}Solution :} \textcolor{OliveGreen}{#1}}
%%%%%%%%%%%%%%%%%%%%%%%%%%%%%%%%%%%%%%%%%%%%%%%%%%%%%%%%%%%%%


\title{Corrigé du TD2 EDO}
\author{$\mathcal{F.J}$}
\date{2023-2024}


\begin{document}
\maketitle
%%%%%%%%%%%%%%%%%%%%%%%%%%%%%%%%%%%%%%%%%%%%%%%%%%%%%%%%%%%%%%%%%%%%%%%%%%%%%%%%%%
%Exo 1
%%%%%%%%%%%%%%%%%%%%%%%%%%%%%%%%%%%%%%%%%%%%%%%%%%%%%%%%%%%%%%%%%%%%%%%%%%%%%%%%%%
\begin{exo}
  Soient $I\subseteq\R$ un intervalle ouvert et $U\subseteq\R$ un ouvert. Montrer que toute fonction $f:I\times U\longrightarrow\R$ de classe $C^1$ est localement Lipschitzienne par rapport à sa deuxième variable.
\end{exo}

%%%%%%%%%%%%%%%%%%%%%%%%%%%%%%%%%%%%%%%%%%%%%%%%%%%%%%%%%%%%%%%%%%%%%%%%%%%%%%%%%%
%Exo 2
%%%%%%%%%%%%%%%%%%%%%%%%%%%%%%%%%%%%%%%%%%%%%%%%%%%%%%%%%%%%%%%%%%%%%%%%%%%%%%%%%%

\begin{exo}
  Soit $y_0\in]-\infty,4[$ et le problème de Cauchy $(E)$ suivant :
 $$
\begin{cases}
y'=(4-y)^3\\
y(0)=y_0.
\end{cases}
$$
\begin{enumerate}
    \item Justifier que ce problème admet une unique solution maximale $y$, définie dans un intervalle $J\subseteq\R$. Sans calculer $y$, montrer que $y(t)\neq4$, pour tout $t\in J$.
\solution{
$(E)$ est un problème de Cauchy de la forme :
 $$
\begin{cases}
y'=f(t,y)\\
y(0)=y_0.
\end{cases}
$$
avec $f : \R \times \R \to \R$, $(t,y) \mapsto (4-y)^3$.\\
La fonction est de classe $\mathcal{C}^1$ sur $\R\times\R$ donc localement lipschitzienne par rapport à sa deuxième variable (cf exo 1).\\
Par le théorème de Cauchy Lipschitz on est assuré qu'il existe une unique solution maximale de $(E)$, $y_1 : J \to \R$, définie dans un intervalle $J \subseteq \R$ ouvert.\\
Notre but est  de montrer que $\forall t \in J, y(t) \ne 4$, pour cela on va montrer :\\
    (i) : $\nexists t \in J$ tq $y_1 = 4$\\
    (ii) : $\forall t \in J, y_1(t) < 4$\\
On a (i) $\Rightarrow$ (ii), montrons le :\\
L'énoncé nous donne $y_1(t_0) = y_0 < 4$\\
Supposons par l'absurde que $\exists t_1 \in J$ tel que $y_1(t_1) \geq 4$\\
Or $y_1$ est continue, donc d'après le TVI $\exists \bar{t} \in J$ tel que $y_1(\bar{t}) = 4$ mais cela est impossible d'après (i). Ce que l'on a supposé est donc absurde. \\
Il nous reste a montrer (i) : \\
Supposons par l'absurde que $\exists \bar{t} \in J$ tel que $y_1(\bar{t}) = 4$\\
Considérons le problème de Cauchy $(E_2)$ suivant :
$$
\begin{cases}
y'=(4-y)^3\\
y(\bar{t})=4.
\end{cases}
$$
On a $y_c \equiv 4$ solution triviale de $(E_2)$ c'est une solution globale.\\
$y_1$ est également solution.
D'où $y_1$ et $y_c$ sont deux solutions de $(E_2)$ définient sur $\R \cap J = J$.\\
		On a $y_1(\bar{t}) = 4 = y_c$ d'après le corollaire 1 on a donc $y_1 \equiv y_c$.\\
Mais on aurait également $y_1(t_0) = y_0 = y_c = 4$ ce qui est absurde car $y_0 < 4$. Ce que l'on a supposé est donc absurde.\\
Donc $\nexists t \in J$ tel que $y_1(t) = 4$\\
On a donc montré que $y(t) \ne 4$ et même que $y(t) < 4, ~ \forall t \in J$.}

    \item Calculer $y$ et $J$. Est-ce que la solution maximale $y$ est globale ? 
    \solution{pas globale...}
\end{enumerate}
\end{exo}

%%%%%%%%%%%%%%%%%%%%%%%%%%%%%%%%%%%%%%%%%%%%%%%%%%%%%%%%%%%%%%%%%%%%%%%%%%%%%%%%%%
%Exo 3
%%%%%%%%%%%%%%%%%%%%%%%%%%%%%%%%%%%%%%%%%%%%%%%%%%%%%%%%%%%%%%%%%%%%%%%%%%%%%%%%%%

\begin{exo}
Considérons le problème de Cauchy 
$$
\begin{cases}
y'=4-y^2\\
y(0)=0.
\end{cases}
$$

\begin{enumerate}
    \item Justifier que ce problème admet une unique solution maximale $y$, définie dans un intervalle $I\subseteq\R.$
\solution{problème de cauchy de la forme $f(t,y)$ avec $f$ de classe $\mathcal{C}^1$ donc lispchitzienne par rapport à $y_1$ donc le théorème de Cauchy nous assure l'existence d'une unique solution définie dans un intervalle $J$ de $\R$ ouvert. Le théorème nous dit que c'est une solution maximale.}
    \item Justifier que la solution $y$ est bornée et calculer un minorant et un majorant pour $y.$
\solution{Il faut repérer les points d'équilibre de l'EDO ce qui ramène au but de cette question qui est de montrer deux choses :
\begin{itemize}
    \item[(i)] $\nexists t \in J$ tel que $y_1(t) = 2$ ou $ y_1 = -2$ avec $2$ et $-2$ point d'équilibre.
    \item[(ii)] $\forall t \in J, f(t) \in ]-2,2[$
\end{itemize}
Le plus commode est certainement de découper en deux sous cas, de rédiger le premier pour la point d'équilibre $2$ et de conclure en disant que les deux cas se traite de la même manière. \textit{NB : (i) $\Rightarrow$ (ii), une fois cette implication démontrée il ne reste plus qu'à montrer (i).}
}
\end{enumerate}
On suppose ici que la solution $y$ est globale, {\it i. e.} que $I=\R$ (on verra au prochain cours un résultat permettant de le justifier).
\begin{enumerate}
\setcounter{enumi}{2}
\item Calculer les limites de $y$ en $\pm\infty.$
\solution{
On suppose la solution $y_1$ au problème de Cauchy est globale, c'est à dire que l'intervalle $J \subseteq R$ sur lequel la fonction solution est définie est $\R$ tout entier ie $J = ]-\infty, +\infty[$.\\
Il est facile de voir que $y'$ est strictement positive par la question 2, donc la fonction $y$ est strictement croissante ET bornée donc la fonction admet une limite $l$ FINIE qui reste encore à determiner.\\
On peut dire que : \\
$y \to l \Rightarrow y' \to 4 - l^2$ quand $t \to \infty$\\
Si $4 -l^2 \ne 0$\\
Alors $\int_{0}^{T}y'(s)ds$ diverge grossièrement.\\
Mais donc $\int_{0}^{T}y'(s)ds = y(T) - y(0) = y(T) \to \infty$ quand $T \to \infty$\\
Ce qui est absurde car on a montré qu'il existait une limite finie $l$ à $y$.\\
Donc $4-y^2 = 0 \Rightarrow l = 2 \text{ ou } l = -2$
}
\item Calculer $y''$ en fonction de $y$ et étudier le signe de $y''.$ 
\solution{$y'' = -2y'y = -2y(4-y^2)$ avec $(4-y^2) > 0$ car $y \in ]-2,2[$ et $y < 0$ si $t<0$ et $y>0$ si $t>0$ donc 
$$
\begin{cases}
y'' > 0 \text{ si } t < 0 \text{ donc y est convexe sur } $]-\infty , 0 [$; \\
y'' < 0 \text{ si } t > 0 \text{ donc y est concave sur } $]0, \infty[$.
\end{cases}
$$
}
\item Donner une allure du graphe de la solution $y.$
\end{enumerate}

\end{exo}

%%%%%%%%%%%%%%%%%%%%%%%%%%%%%%%%%%%%%%%%%%%%%%%%%%%%%%%%%%%%%%%%%%%%%%%%%%%%%%%%%%
%Exo 4
%%%%%%%%%%%%%%%%%%%%%%%%%%%%%%%%%%%%%%%%%%%%%%%%%%%%%%%%%%%%%%%%%%%%%%%%%%%%%%%%%%

\begin{exo}
Soit $f:\R\longrightarrow\R$ une fonction de classe $C^1$ et $b:\R\longrightarrow\R$ une fonction continue périodique de période $T$. Soit $y:\R\longrightarrow\R$ une solution de l'équation $y'=f(y)+b(t)$. Montrer que $y$ est $T-$périodique si et seulement si $y(T)=y(0)$.
	\solution{On a $\Rightarrow$ qui est immédiat, occupons nous de $\Leftarrow$ :\\
	On suppose donc $y(T) = y(0)$ ;\\
	On a $b(t+T) = b(t)$ car $b$ est $T$-périodique ;\\
	On a $y$ solution de $y' = f(y)+b(t)$ ; \\
	Posons $F(t,y) = f(y) + b(t)$\\
	On a $f$ $\mathcal{C}^1$, la fonction $b$ ne dépend pas de $y$, $F$ est donc lipschitzienne par rapport à $y$ ;\\
	On veut savoir si $y(t+T) = y(t)$ ??\\
	Posons $z(t) = y(t+T)$ $z$ est aussi solution de l'ED $y' = f(y) + b(t)$ ;\\
	Mais on a pour $t = 0$ : \\
	$y(0) = z(0) = y(T) = y(0)$ par hypothèse.\\
	Il existe donc un point $\bar{t} \in J = \R$ tel que deux solutions de l'ED $y' = f(y) + b(t)$ sont égales. Mais d'après le corollaire 1 du cours, si il existe un point situé sur l'intersection des deux intervalles de définitions de ces solutions tel que ces deux solutions sont égales, alors elles sont égales pour tout point $t$ de cette intersection. Donc $y\equiv z$ autrement dit $y(t) = y(t+T), \forall t \in \R$.
	}
\end{exo}

%%%%%%%%%%%%%%%%%%%%%%%%%%%%%%%%%%%%%%%%%%%%%%%%%%%%%%%%%%%%%%%%%%%%%%%%%%%%%%%%%%
%Exo 5
%%%%%%%%%%%%%%%%%%%%%%%%%%%%%%%%%%%%%%%%%%%%%%%%%%%%%%%%%%%%%%%%%%%%%%%%%%%%%%%%%%

\begin{exo}
  Soient $t_0,\ y_0\in\R$ et le problème de Cauchy
$$
(*)\ \ \ \ \ \begin{cases}
y'=y(y-1)(y-t)\\
y(t_0)=y_0.
\end{cases}
$$
\begin{enumerate}
  \item Justifier que quelque soit $(t_0,y_0)$, le problème $(*)$ admet une unique solution maximale $y:J\longrightarrow\R$, définie dans un intervalle ouvert $J\subseteq\R$ tel que $t_0\in J$. 
\solution{le problème $(*)$ est de la forme 
$$
\begin{cases}
y'=f(t,y)\\
y(t_0)=y_0.
\end{cases}
$$
Avec $f$ une fonction de classe $\mathcal{C}^1$ donc lipschitzienne par rapport à $y$, le théorème de Cauchy Lipschitz nous assure l'existence d'une solution unique maximale $y_1$ définie sur sur un intervalle ouvert $J \subseteq \R$
}
  \item Donner les solutions constantes de l'équation $y'=y(y-1)(y-t)$.
\solution{Cherchons les solutions constante, pour cela posons $y=K \in \R$ on a donc $y' = 0$ d'où : $0 = K(K-1)(K-t) \Rightarrow K = 0 \text{ et } K = 1 \forall t \in J$.\\
Donc les solutions constante sont $y = 0$ et $y = 1$ ce sont les points d'équilibre.
}
\end{enumerate}
Par la suite, $y$ désigne la solution maximale de $(*)$ et $J$ l'intervalle où elle est définie. On va s'intéresser aux cas où
où $t_0>1$ et $y_0\in]0,t_0[$.
\begin{enumerate}
  \setcounter{enumi}{2}
  \item Supposons $y_0\in]0,1[$. Montrer que $y$ est bornée sur $J$.
\solution{On a donc deux points d'équilibre $y_{c1} \equiv 1$ et $y_{c0} \equiv 0$ donc pour $y_0 \in ]0,1[$ on va montrer que la fonction solution $y_1$ est bornée entre $y_{c1}$ et $y_{c0}$. Ce qui nous ramène à montrer deux choses : 
\begin{itemize}
	\item[(i)] $\nexists t \in J \text{ tq } y_1(t) = 0 \text{ ou } y_1(t) = 1$ ;
	\item[(ii)] $\forall t \in J, 0<y_1(t)<1$
		On montre $y_1(t) < 1, \forall t \in J$, le fait que $y(t) > 0 \forall t \in J$ provient d'un raisonnement tout à fait analogue.
		Montrons donc $y(t) < 1 \forall t \in J$ : \\ $\to$ cf le raisonnement de l'exercice 2 qui est exactement le même !
\end{itemize}
}
    \item Supposons que $t_0>1$ et que $y_0\in]1,t_0[$. Montrer que pour tout $t\in J,\ t>t_0$, on a $1<y(t)<t$.
    \solution{Pour $y(t) > 1, \forall t \in J$ même raisonnement que d'habitude, pour $y(t) < t, \forall t \in J, t > t_0$, raisonnons par l'absurde :\\
    Supposons que $\exists \bar{t} \in J$  tel que $y(\bar{t}) = \bar{t}$, l'ensemble $\{t \in J \mid y(t) = t \}$ est un ensemble fermé, il est non vide par hypothèse, on peut donc prendre son minimum : $t_{min}$. On a ainsi $1 < t_0 < t_{min}$, mais pour tout les $t\in J, t \leq t_{min}$ on a $y' \geq 0$ c'est à dire que $y$ décroit, donc $t_0 > y_0 = y(t_0) > y(t_{min}) = t_{min}$, ce qui est absurde. Donc $y(t) < t, \forall t \in J, t > t_0$.
    }
    \item Supposons que $J$ est un intervalle de la forme $]T^-,T^+[$, où $T^-,\ T^+\in\R\cup\{-\infty,+\infty\}$. On veut justifier que si $t_0>1$ et $y_0\in]0,t_0[$, alors la solution maximale $y$ de $(*)$ est globale à droite, c'est-à-dire que $T^+=+\infty$.
        \begin{enumerate}
        \item Supposons $y_0=1$. Justifier que $y$ est globale, c'est-à-dire que $J=\R$.
        \item Soit $t_0>1$ et $y_0\in]0,1[$ ou $y_0\in]1,t_0[$. On va montrer par l'absurde que l'on ne peut pas avoir $T^+<+\infty$.          
              \begin{enumerate}
                \item Supposons que $T^+<+\infty$. Justifier qu'il existe la limite
        $$
\lim_{t\to T^+,t<T^+}y(t),
$$
et que cette limite est finie.
\item Soit $y^+\in\R$ la limite précédente. En considérant le problème de Cauchy pour l'EDO $y'=y(y-1)(y-t)$, de donnée initiale $(T^+,y^+)$, montrer que $y$ admet un prolongement. Conclure.
        \end{enumerate}
        \item Calculer, selon que $y_0\in]0,1]$ ou que $y_0\in]1,t_0[$, avec $t_0>1$, $\lim_{t\to+\infty}y(t)$.
        \end{enumerate}
        \end{enumerate}
\end{exo}

%%%%%%%%%%%%%
%Exo 6
%%%%%%%%%%%%%

\begin{exo}
  Soit le système d'équations différentielles
  $$
  (S)\begin{cases}
    x'=x(1-x-y/2)\\
    y'=y(1-y-x/2).
  \end{cases}
  $$
  \begin{enumerate}
    \item 
      Écrire le système $(S)$ sous la forme $Y'=F(t,Y)$, avec $Y=\Big(\begin{array}{c}x\\y\end{array}\Big)$, en explicitant la fonction $F$ et son domaine de définition.
	      \solution{On a le système (S) que l'on peut écrire : $Y' = F(t,Y)$ avec $F(t,Y) = (F_1(t,(x,y)),F_2(t,(x,y)))^T$ avec $F_1(t,(x,y)) = x(1-x-y/2)$ et $F_2(t,(x,y)) = y(1-y-x/2)$ $F$ est de classe $\mathcal{C}^1$ sur $\R\times\R^2$ donc localement lipscitzienne par rapport à $(x,y)$.}
  \item Soit $\left(J,t\in J\mapsto \Big(\begin{array}{c}x(t)\\y(t)\end{array}\Big)\right)$ une solution maximale de $(S)$. Montrer que $\left(J,t\in J\mapsto\Big(\begin{array}{c}y(t)\\x(t)\end{array}\Big)\right)$ est aussi solution maximale de $(S)$.
		  \solution{$(x_1(t),y_1(t))^T$ est solution de (S) montrons que $(y_1(t),x_1(t))^T$ est aussi solution de (S). Posons $Z(t) = (z_1(t),z_2(t))^T$ tel que $z_1(t) = y_1(t)$ et $z_2(t) = x_1(t)$ a t-on 
  $$
		  (S_1)\begin{cases}
    z_1' \stackrel{?}{=} z_1(1-z_1-z_2/2)\\
    z_2' \stackrel{?}{=} z_2(1-z_2-z_1/2)
  \end{cases}
  $$
  $$
		  (S_1)\iff (S_2) \begin{cases}
    y_1' = y_1(1-y_1-x_1/2)\\
    x_1' = x_1(1-y_1-x_1/2).
		  \end{cases} \text{OK}
  $$
}
  \end{enumerate}
  \begin{enumerate}
    \item Soient $t_0\in\R$ et $(x_0,y_0)\in\R^2$. Justifier l'existence et l'unicité d'une solution maximale du problème de Cauchy pour $(S)$, de donnée initiale $(x(t_0),y(t_0))=(x_0,y_0)$.
	    \solution{On a vu dans la question 1 que $F$ répondait aux hypothèses du Théorème de Cauchy Lipschitz donc on est assuré de l'existence d'une unique solution maximale $(x_1,y_1)$ définie dans un intervalle ouvert $J \subseteq \R$}
		  
    \item Supposons $x_0=y_0$. Soit $\left(J,t\in J\mapsto \Big(\begin{array}{c}x(t)\\y(t)\end{array}\Big)\right)$ la solution maximale du problème de Cauchy pour $(S)$, de donnée initiale $(x(t_0),y(t_0))=(x_0,y_0)$. Justifier que $x(t)=y(t)$ pour tout $t\in J$, et calculer la solution $t\mapsto\Big(\begin{array}{c}x(t)\\y(t)\end{array}\Big)$.
		    \solution{On a montrer dans la question 2 que si $\Big(\begin{array}{c}x(t)\\y(t)\end{array}\Big)$ était solution de (S), alors $\Big(\begin{array}{c}y(t)\\x(t)\end{array}\Big)$ est aussi solution de (S), on considère le PC : 
 $$
\begin{cases}
	(S)\\
	(x(t_0),y(t_0)) = (x_0,x_0).
\end{cases}
$$
		 On a  $\Big(\begin{array}{c}x(t)\\y(t)\end{array}\Big)$ solution maximale de PC, mais  $\Big(\begin{array}{c}y(t)\\x(t)\end{array}\Big)$ aussi, ces deux solutions maximales ont la même condition initiale donc sont égales, c'est à dire : 
			 $$\Big(\begin{array}{c}x(t)\\y(t)\end{array}\Big) =  \Big(\begin{array}{c}y(t)\\x(t)\end{array}\Big),~ \forall t \in J;$$
				 D'où $x\equiv y$ sur $J$.
			 }
		  
        \end{enumerate}
\end{exo}

\end{document}

\end{document}          
