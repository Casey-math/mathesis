\documentclass[12pt,a4paper]{article}
\usepackage[left=2cm,right=2cm,top=2cm,bottom=2cm]{geometry}
\usepackage{amsmath}
\usepackage{amsthm}
\usepackage{thmtools}
\usepackage{amsfonts}
\usepackage{amssymb}
\usepackage[utf8]{inputenc}
\usepackage[T1]{fontenc}
\usepackage[francais]{babel}
\usepackage{mathtools}   % loads »amsmath
\usepackage[dvipsnames]{xcolor}
\usepackage{ulem}
\usepackage{enumitem}
\usepackage{pifont}
\usepackage[most]{tcolorbox}
\usepackage[scr=rsfs]{mathalpha}
\usepackage{esvect} %vecteur avec \vv
\usepackage{dsfont}


\usepackage{calc}
\usepackage[nothm]{thmbox}



%Raccourcies
\newcommand{\N}{\mathbb{N}}
\newcommand{\1}{\mathds{1}}
\newcommand{\PP}{\mathds{P}}
\newcommand{\Z}{\mathbb{Z}}
\newcommand{\E}{\mathds{E}}
\newcommand{\K}{\mathbb{K}}
\newcommand{\Q}{\mathbb{Q}}
\newcommand{\R}{\mathbb{R}}
\newcommand{\C}{\mathbb{C}}
\newcommand{\ie}{\textit{i.e }}
\newcommand{\A}{\mathscr{A}}
\newcommand{\Aa}{\mathcal{A}}
\newcommand{\Mm}{\mathcal{M}}
\newcommand{\Chi}{\mathcal{X}}
\newcommand{\Ll}{\mathcal{L}}
\newcommand{\Tup}{\bigtriangleup}
\newcommand{\Tdn}{\bigtriangledown}
\newcommand{\bps}{\langle}
\newcommand{\eps}{\rangle}
\newcommand{\pv}{\wedge}
\newcommand{\doigt}{\ding{43} ~}
\newcommand{\cray}{\ding{46} ~}


% Définir une commande pour la boîte encadrée
\newcommand{\todo}[1]{%
    \begin{tcolorbox}[colback=red!10!white, colframe=red!75!black, title=To-do :]
        #1
    \end{tcolorbox}%
}
%\begin{tcolorbox}[colback=red!10!white, colframe=red!75!black, title=Ma Boîte Encadrée]
%    Ceci est un exemple de texte à l'intérieur de la boîte.
%\end{tcolorbox}





\def\BC{base canonique }
\def\ev{espace vectoriel }
\def\evs{espaces vectoriels }
\def\sevs{sous-espaces vectoriels }
\def\sev{sous-espace vectoriel }
\def\sep{sous-espace propre }
\def\seps{sous-espaces propres }
\def\sea{sous-espace affine }
\def\seas{sous-espaces affines }
\def\AL{application lin\'eaire }
\def\ALs{applications lin\'eaires }
\def\AA{application affine }
\def\AAs{applications affines }
\def\Aff{ \hbox{\it Aff}}
\def\vep{vecteur propre }
\def\vap{valeur propre }
\def\ssi{si et seulement si }

\newcommand{\dessous}[2]{\underset{#1}{#2}}

\def\bar#1{\overline{#1}}
\def\e#1{{\hbox{e}^{#1}}}
\def\mat#1{\begin{pmatrix}#1\end{pmatrix}}
\def\vect#1{\overrightarrow{\kern-1pt#1\kern 2pt}}
\def\card#1{{\hbox{Card}(#1)}}
\def\bin(#1,#2){ \left(\!\begin{smallmatrix} #2 \\ #1 \end{smallmatrix}\!\right)}
\def\Aff{ \hbox{\it Aff}}
\def\lims{\,\overline{\lim}\;}
\def\limi{\,\underline{\lim}\;}
\def\vide{\emptyset}
\def\lims{\,\overline{\lim}\;}
\def\limi{\,\underline{\lim}\;}
\def\vide{\emptyset}
\def\vfi{\varphi}
\def\dim#1{\text{dim }(#1)}

\def\ben{\begin{enumerate}}
\def\een{\end{enumerate}}
\def\bi{\begin{itemize}}
\def\ei{\end{itemize}}

\DeclareMathOperator{\Id}{Id}
\DeclareMathOperator{\Tr}{Tr}
\DeclareMathOperator{\rg}{rg}

\definecolor{rltred}{rgb}{0.75,0,0}
	\definecolor{rltgreen}{rgb}{0,0.5,0}
	\definecolor{oneblue}{rgb}{0,0,0.75}
	\definecolor{marron}{rgb}{0.64,0.16,0.16}
	\definecolor{forestgreen}{rgb}{0.13,0.54,0.13}
	\definecolor{purple}{rgb}{0.62,0.12,0.94}
	\definecolor{dockerblue}{rgb}{0.11,0.56,0.98}
	\definecolor{freeblue}{rgb}{0.25,0.41,0.88}
	\definecolor{myblue}{rgb}{0,0.2,0.4}

%Note
\newenvironment{note}{\par\medskip\noindent\begin{tabular}{l|p{\linewidth-8cm}}
\ding{46}& \color{black}}
{\end{tabular}\par\medskip}
\def\bn{\begin{note}}
\def\en{\end{note}}


%Attention
\def\war{\color{red}\medskip\begin{thmbox}[M]{\ding{39}\textbf{Attention :}}\noindent\color{black}}
\def\endwar{\end{thmbox}}
\def\bw{\begin{war}}
\def\ew{\end{war}}


%Style définition
\declaretheoremstyle[
  headfont=\color{RoyalBlue}\normalfont\bfseries,
  bodyfont=\color{NavyBlue}\normalfont,
]{colored}
\declaretheorem[
  style=colored,
  name=Définition,
]{df}
\def\bd{\begin{df}}
\def\ed{\end{df}}

%Style Proposition
\declaretheoremstyle[
  headfont=\color{BrickRed}\normalfont\bfseries,
  bodyfont=\color{black}\normalfont,
]{coloredProp}
\declaretheorem[
  style=coloredProp,
  name=Proposition,
]{prop}
\def\bp{\begin{prop}}
\def\ep{\end{prop}}

%Style exo
\declaretheoremstyle[
  headfont=\color{orange}\normalfont\bfseries,
  bodyfont=\color{black}\normalfont,
]{coloredexo}
\declaretheorem[
  style=coloredexo,
  name=Exercice,
]{exo}
\def\be{\begin{exo}}
\def\ee{\end{exo}}

%Règle soulignée
\def\regle{\color{blue}\begin{thmbox}[M]{\textbf{Règle :}}\noindent\color{blue}}
\def\endregle{\end{thmbox}}
\def\br{\begin{regle}}
\def\er{\end{regle}}

%Démonstration
\let\oldproof\proof
\renewcommand{\proof}{\color{olive}\oldproof}
\def\bpf{\begin{proof}}
\def\epf{\end{proof}}

%\theoremstyle{definition}
%\newtheorem{prop}{Proposition}
%\newtheorem{exo}{Exercice}
%\newtheorem{df}{Définition}

\newcommand{\bpropo}{
    \begin{tcolorbox}[colback=Yellow!10!Yellow, colframe=red!75!black]
    \bp
}
\newcommand{\epropo}{
    \ep
    \end{tcolorbox}}

%Solution
\newcommand{\solution}[1]{\par\noindent\textbf{\color{OliveGreen}Solution :} \textcolor{OliveGreen}{#1}}

%Titre
%\title{Espaces affines, notes   }
\author{$\mathcal{F.J}$}
%\date{2023-2024}

\usepackage[utf8]{inputenc}
\usepackage{enumitem}

\title{Création d'un Terrarium Diversifié et Autonome}
\author{Anito Kodama}
\date{\today}

\begin{document}

\maketitle

\section*{Introduction}
Créer un terrarium complet, comprenant des plantes, des insectes et un équilibre écologique durable, demande une planification minutieuse. Ce guide vous propose une liste d'espèces de plantes et d'insectes à intégrer progressivement tout au long de l'année, en prenant soin de maintenir les conditions idéales pour chaque organisme. Nous détaillerons également les moments appropriés pour l'arrosage, l'ajustement de la lumière et d'autres soins nécessaires à l'entretien du terrarium.

\section*{Matériel nécessaire}
\begin{itemize}
    \item \textbf{Récipient} : Choisissez un terrarium de taille adaptée (20-30L minimum), fermé ou avec une aération.
    \item \textbf{Drainage} : Une couche de billes d'argile (2-3 cm) et une fine couche de charbon actif pour éviter les moisissures.
    \item \textbf{Substrat} : Un mélange de terreau forestier, de feuilles mortes broyées, de mousse et de copeaux de bois pour recréer l'humus forestier.
    \item \textbf{Plantes} : Une sélection variée de plantes adaptées à un environnement humide et ombragé.
    \item \textbf{Faune} : Insectes décomposeurs, prédateurs, et autres espèces d'invertébrés pour maintenir l'équilibre.
\end{itemize}

\section*{Sélection des Plantes et des Insectes}

\subsection*{Plantes}
Optez pour des plantes de petite taille et à croissance lente, adaptées aux conditions humides et ombragées du terrarium. Voici quelques suggestions :

\begin{itemize}
    \item \textbf{Fougères miniatures} : Asplenium, Nephrolepis.
    \item \textbf{Mousses} : Mousse de forêt (Polytrichum), mousse de Java.
    \item \textbf{Plantes couvre-sol} : Helxine (Soleirolia), Pilea.
    \item \textbf{Plantes à fleurs} : Petites violettes ou épiphytes comme Tillandsia.
    \item \textbf{Plantes vivaces} : Epipremnum (Pothos), Fittonia.
\end{itemize}

\subsection*{Insectes et Invertébrés}
Pour maintenir un écosystème équilibré, introduisez des décomposeurs et des prédateurs :

\begin{itemize}
    \item \textbf{Décomposeurs} :
    \begin{itemize}
        \item Cloportes (Porcellio scaber).
        \item Lombrics (vers de terre).
        \item Collemboles (pour contrôler les moisissures).
    \end{itemize}
    \item \textbf{Prédateurs} :
    \begin{itemize}
        \item Araignées sauteuses.
        \item Opilions (faucheux).
        \item Pseudoscorpions.
    \end{itemize}
    \item \textbf{Autres} :
    \begin{itemize}
        \item Fourmis (espèce locale, non invasive).
        \item Grillon, limaces, escargots (en quantité contrôlée).
    \end{itemize}
\end{itemize}

\section*{Calendrier d'Introduction des Espèces}

\subsection*{1. Préparation Initiale (Jour 1-2)}
\begin{itemize}
    \item \textbf{Substrat} : Disposez d'abord les couches de drainage (billes d'argile et charbon actif), puis ajoutez le substrat de terreau forestier, de feuilles mortes et de copeaux de bois.
    \item \textbf{Plantes} : Plantez les fougères, les mousses et les petites plantes couvre-sol.
    \item \textbf{Arrosage} : Arrosez légèrement pour humidifier le substrat. Maintenez l'humidité mais évitez l'excès d'eau.
    \item \textbf{Lumière} : Fournissez une lumière indirecte (lumière tamisée ou sous lampe à faible intensité). Évitez la lumière directe.
\end{itemize}

\subsection*{2. Introduction des Décomposeurs (Après 2-3 Semaines)}
\begin{itemize}
    \item \textbf{Cloportes, Lombrics et Collemboles} : Introduisez les décomposeurs une fois que les plantes sont bien établies et que le microclimat est stable. Ils commenceront à recycler la matière organique et maintenir le sol aéré.
    \item \textbf{Arrosage} : Continuez à maintenir une humidité constante. Ne laissez pas le sol se dessécher complètement.
    \item \textbf{Lumière} : Veillez à ce que la lumière soit indirecte et douce.
\end{itemize}

\subsection*{3. Introduction des Prédateurs (Après 4-6 Semaines)}
\begin{itemize}
    \item \textbf{Araignées et Opilions} : Une fois que les décomposeurs sont établis et qu'il y a suffisamment de matière organique, introduisez les prédateurs pour maintenir l'équilibre de la faune.
    \item \textbf{Grillons et Fourmis} : Ajoutez des grillons et des fourmis après les décomposeurs, pour diversifier la chaîne alimentaire.
    \item \textbf{Arrosage} : Surveillez l'humidité et ajustez en fonction des besoins des espèces. Les prédateurs n'ont pas besoin d'une humidité excessive.
    \item \textbf{Lumière} : Veillez à ce que la lumière soit tamisée pour ne pas perturber l'équilibre de l'écosystème.
\end{itemize}

\subsection*{4. Maintien et Suivi (Au Cours de l'Année)}
\begin{itemize}
    \item \textbf{Observation régulière} : Surveillez la santé des plantes et des animaux. Si la population d'insectes augmente trop, ajustez le nombre de prédateurs ou d'herbivores.
    \item \textbf{Arrosage} : En été, veillez à ce que le terrarium ne devienne pas trop sec. En hiver, réduisez légèrement l'humidité pour simuler une période plus sèche. Assurez-vous que le substrat reste légèrement humide mais pas détrempé.
    \item \textbf{Lumière} : En hiver, si la lumière naturelle est insuffisante, vous pouvez augmenter la lumière indirecte à l'aide d'une lampe à faible consommation (LED). En été, réduisez la lumière pour éviter une surchauffe.
\end{itemize}

\section*{Conseils d'Entretien}

\begin{itemize}
    \item \textbf{Arrosage} : Arrosez régulièrement mais légèrement. Utilisez un pulvérisateur pour maintenir une humidité constante sans saturer le sol. Laissez sécher légèrement entre les arrosages pour éviter la stagnation.
    \item \textbf{Lumière} : Fournissez une lumière indirecte. Ajustez l'intensité en fonction des saisons. En hiver, une lumière plus forte peut être nécessaire.
    \item \textbf{Température} : Maintenez une température modérée (20-25°C). Évitez les changements brusques de température.
    \item \textbf{Nettoyage} : Retirez les débris végétaux en décomposition et les corps d'animaux morts pour maintenir l'équilibre de l'écosystème.
\end{itemize}

\section*{Conclusion}
La création d'un terrarium diversifié et équilibré est un projet passionnant qui demande patience et attention. En introduisant progressivement les plantes et les insectes tout au long de l'année et en surveillant les conditions d'humidité et de lumière, vous créerez un écosystème autonome et fascinant à observer.

\end{document}

