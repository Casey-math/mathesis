\documentclass[12pt,a4paper]{article}
\usepackage[left=2cm,right=2cm,top=2cm,bottom=2cm]{geometry}
\usepackage{amsmath}
\usepackage{amsthm}
\usepackage{thmtools}
\usepackage{amsfonts}
\usepackage{amssymb}
\usepackage[utf8]{inputenc}
\usepackage[T1]{fontenc}
\usepackage[francais]{babel}
\usepackage{mathtools}   % loads »amsmath
\usepackage[dvipsnames]{xcolor}
\usepackage{ulem}
\usepackage{enumitem}
\usepackage{pifont}
\usepackage[most]{tcolorbox}
\usepackage[scr=rsfs]{mathalpha}
\usepackage{esvect} %vecteur avec \vv
\usepackage{dsfont}


\usepackage{calc}
\usepackage[nothm]{thmbox}



%Raccourcies
\newcommand{\N}{\mathbb{N}}
\newcommand{\1}{\mathds{1}}
\newcommand{\PP}{\mathds{P}}
\newcommand{\Z}{\mathbb{Z}}
\newcommand{\E}{\mathds{E}}
\newcommand{\K}{\mathbb{K}}
\newcommand{\Q}{\mathbb{Q}}
\newcommand{\R}{\mathbb{R}}
\newcommand{\C}{\mathbb{C}}
\newcommand{\ie}{\textit{i.e }}
\newcommand{\A}{\mathscr{A}}
\newcommand{\Aa}{\mathcal{A}}
\newcommand{\Mm}{\mathcal{M}}
\newcommand{\Chi}{\mathcal{X}}
\newcommand{\Ll}{\mathcal{L}}
\newcommand{\Tup}{\bigtriangleup}
\newcommand{\Tdn}{\bigtriangledown}
\newcommand{\bps}{\langle}
\newcommand{\eps}{\rangle}
\newcommand{\pv}{\wedge}
\newcommand{\doigt}{\ding{43} ~}
\newcommand{\cray}{\ding{46} ~}


% Définir une commande pour la boîte encadrée
\newcommand{\todo}[1]{%
    \begin{tcolorbox}[colback=red!10!white, colframe=red!75!black, title=To-do :]
        #1
    \end{tcolorbox}%
}
%\begin{tcolorbox}[colback=red!10!white, colframe=red!75!black, title=Ma Boîte Encadrée]
%    Ceci est un exemple de texte à l'intérieur de la boîte.
%\end{tcolorbox}





\def\BC{base canonique }
\def\ev{espace vectoriel }
\def\evs{espaces vectoriels }
\def\sevs{sous-espaces vectoriels }
\def\sev{sous-espace vectoriel }
\def\sep{sous-espace propre }
\def\seps{sous-espaces propres }
\def\sea{sous-espace affine }
\def\seas{sous-espaces affines }
\def\AL{application lin\'eaire }
\def\ALs{applications lin\'eaires }
\def\AA{application affine }
\def\AAs{applications affines }
\def\Aff{ \hbox{\it Aff}}
\def\vep{vecteur propre }
\def\vap{valeur propre }
\def\ssi{si et seulement si }

\newcommand{\dessous}[2]{\underset{#1}{#2}}

\def\bar#1{\overline{#1}}
\def\e#1{{\hbox{e}^{#1}}}
\def\mat#1{\begin{pmatrix}#1\end{pmatrix}}
\def\vect#1{\overrightarrow{\kern-1pt#1\kern 2pt}}
\def\card#1{{\hbox{Card}(#1)}}
\def\bin(#1,#2){ \left(\!\begin{smallmatrix} #2 \\ #1 \end{smallmatrix}\!\right)}
\def\Aff{ \hbox{\it Aff}}
\def\lims{\,\overline{\lim}\;}
\def\limi{\,\underline{\lim}\;}
\def\vide{\emptyset}
\def\lims{\,\overline{\lim}\;}
\def\limi{\,\underline{\lim}\;}
\def\vide{\emptyset}
\def\vfi{\varphi}
\def\dim#1{\text{dim }(#1)}

\def\ben{\begin{enumerate}}
\def\een{\end{enumerate}}
\def\bi{\begin{itemize}}
\def\ei{\end{itemize}}

\DeclareMathOperator{\Id}{Id}
\DeclareMathOperator{\Tr}{Tr}
\DeclareMathOperator{\rg}{rg}

\definecolor{rltred}{rgb}{0.75,0,0}
	\definecolor{rltgreen}{rgb}{0,0.5,0}
	\definecolor{oneblue}{rgb}{0,0,0.75}
	\definecolor{marron}{rgb}{0.64,0.16,0.16}
	\definecolor{forestgreen}{rgb}{0.13,0.54,0.13}
	\definecolor{purple}{rgb}{0.62,0.12,0.94}
	\definecolor{dockerblue}{rgb}{0.11,0.56,0.98}
	\definecolor{freeblue}{rgb}{0.25,0.41,0.88}
	\definecolor{myblue}{rgb}{0,0.2,0.4}

%Note
\newenvironment{note}{\par\medskip\noindent\begin{tabular}{l|p{\linewidth-8cm}}
\ding{46}& \color{black}}
{\end{tabular}\par\medskip}
\def\bn{\begin{note}}
\def\en{\end{note}}


%Attention
\def\war{\color{red}\medskip\begin{thmbox}[M]{\ding{39}\textbf{Attention :}}\noindent\color{black}}
\def\endwar{\end{thmbox}}
\def\bw{\begin{war}}
\def\ew{\end{war}}


%Style définition
\declaretheoremstyle[
  headfont=\color{RoyalBlue}\normalfont\bfseries,
  bodyfont=\color{NavyBlue}\normalfont,
]{colored}
\declaretheorem[
  style=colored,
  name=Définition,
]{df}
\def\bd{\begin{df}}
\def\ed{\end{df}}

%Style Proposition
\declaretheoremstyle[
  headfont=\color{BrickRed}\normalfont\bfseries,
  bodyfont=\color{black}\normalfont,
]{coloredProp}
\declaretheorem[
  style=coloredProp,
  name=Proposition,
]{prop}
\def\bp{\begin{prop}}
\def\ep{\end{prop}}

%Style exo
\declaretheoremstyle[
  headfont=\color{orange}\normalfont\bfseries,
  bodyfont=\color{black}\normalfont,
]{coloredexo}
\declaretheorem[
  style=coloredexo,
  name=Exercice,
]{exo}
\def\be{\begin{exo}}
\def\ee{\end{exo}}

%Règle soulignée
\def\regle{\color{blue}\begin{thmbox}[M]{\textbf{Règle :}}\noindent\color{blue}}
\def\endregle{\end{thmbox}}
\def\br{\begin{regle}}
\def\er{\end{regle}}

%Démonstration
\let\oldproof\proof
\renewcommand{\proof}{\color{olive}\oldproof}
\def\bpf{\begin{proof}}
\def\epf{\end{proof}}

%\theoremstyle{definition}
%\newtheorem{prop}{Proposition}
%\newtheorem{exo}{Exercice}
%\newtheorem{df}{Définition}

\newcommand{\bpropo}{
    \begin{tcolorbox}[colback=Yellow!10!Yellow, colframe=red!75!black]
    \bp
}
\newcommand{\epropo}{
    \ep
    \end{tcolorbox}}

%Solution
\newcommand{\solution}[1]{\par\noindent\textbf{\color{OliveGreen}Solution :} \textcolor{OliveGreen}{#1}}

%Titre
%\title{Espaces affines, notes   }
\author{$\mathcal{F.J}$}
%\date{2023-2024}

\usepackage[utf8]{inputenc}
\usepackage{enumitem}

\title{Créer un terrarium forestier équilibré}
\author{Anito Kodama}
\date{\today}

\begin{document}

\maketitle

Créer un terrarium forestier équilibré avec des insectes et leurs prédateurs est une expérience fascinante, vous pouvez également le faire avec des enfants ! Voici les étapes et conseils pour préparer un terrarium réussi, incluant substrat, végétation et faune pour créer un petit écosystème fonctionnel :

\section{Matériel nécessaire}
\begin{itemize}
    \item \textbf{Terrarium} : Choisissez un récipient adapté, transparent et avec une bonne capacité (minimum 20-30 litres), idéalement fermé ou avec une aération.
    \item \textbf{Drainage} : Cela est crucial pour éviter l'accumulation d'eau stagnante.
    \begin{itemize}
        \item \textbf{Billes d'argile} : Une couche de 2-3 cm au fond pour le drainage.
        \item \textbf{Charbon actif} : Ajoutez une fine couche (environ 0,5-1 cm) au-dessus des billes pour limiter les odeurs et la prolifération de moisissures.
    \end{itemize}
\end{itemize}

\section{Substrat}
\begin{itemize}
    \item \textbf{Terreau forestier} : Un mélange riche en humus, idéalement composé de terreau, de feuilles mortes broyées, de bois en décomposition (éventuellement des copeaux de bois), et de mousse pour recréer l'humus forestier. Vous pouvez aussi mélanger un peu de sable pour améliorer le drainage et l'aération.
    \item \textbf{Couches de décomposition} : Inclure des feuilles mortes fraîches sur le dessus permet d'alimenter le système.
\end{itemize}

\section{Plantes à intégrer}
Optez pour des plantes de petite taille et à croissance lente, adaptées à des environnements humides et ombragés :
\begin{itemize}
    \item \textbf{Fougères miniatures} : Comme les \textit{Asplenium} ou les \textit{Nephrolepis}.
    \item \textbf{Mousse} : Prévoyez différentes mousses forestières qui s'adaptent bien à l'humidité et favorisent la rétention d'eau.
    \item \textbf{Petits couvre-sol} : Plantes comme \textit{Helxine (Soleirolia)}, \textit{Pilea}, ou même de petites variétés de \textit{Fittonia}.
    \item \textbf{Petites plantes à fleurs} : Vous pouvez intégrer de petites violettes forestières si vous le souhaitez.
\end{itemize}

\section{Faune du terrarium}
Pour créer un écosystème équilibré, vous pouvez introduire :

\subsection{Décomposeurs}
\begin{itemize}
    \item \textbf{Cloportes (Porcellio scaber ou Armadillidium)} : Ils aident à décomposer la matière organique.
    \item \textbf{Lombrics} : Petits vers de terre pour aider à aérer le sol et décomposer les matières organiques.
    \item \textbf{Collemboles} : Petits insectes souvent utilisés dans les terrariums pour contrôler les moisissures et décomposer la matière organique.
\end{itemize}

\subsection{Prédateurs}
\begin{itemize}
    \item \textbf{Araignées (petites espèces locales)} : Des espèces de petite taille et non agressives pour les humains, telles que des araignées sauteuses.
    \item \textbf{Pseudoscorpions} (si vous pouvez en trouver) : De petits prédateurs naturels, intéressants mais souvent difficiles à obtenir.
\end{itemize}

\section{Équilibre de l'écosystème}
\begin{itemize}
    \item \textbf{Humidité} : Maintenez un bon niveau d'humidité en pulvérisant régulièrement (sans inonder).
    \item \textbf{Lumière} : Fournissez une lumière indirecte. La lumière du soleil directe peut surchauffer et déséquilibrer l'écosystème.
    \item \textbf{Alimentation} : Les décomposeurs se nourriront des feuilles mortes, tandis que les prédateurs s'attaqueront aux populations d'insectes. Il faudra surveiller la population pour éviter une surpopulation ou un effondrement.
\end{itemize}

\section{Conseils supplémentaires}
\begin{itemize}
    \item \textbf{Observation et ajustement} : Surveillez régulièrement l'état des plantes et des animaux. Si vous constatez une surpopulation ou des signes de déséquilibre, ajustez en conséquence.
    \item \textbf{Évitez les produits chimiques} : Aucun engrais ou pesticide ne doit être utilisé.
\end{itemize}

Avec ces conseils, vous aurez un terrarium forestier autonome et fascinant à observer !
\newpage

\section{Calendrier d'introduction des espèces}
Pour créer un terrarium forestier équilibré, il est important d'introduire les plantes et les animaux de manière progressive afin de permettre à chaque composant de s'adapter et de se stabiliser avant d'ajouter d'autres éléments. Voici un calendrier d'introduction étape par étape :

\subsection*{1. Préparation initiale (jour 1)}
\begin{itemize}
    \item \textbf{Installation du substrat} : Commencez par mettre en place les couches de drainage (billes d'argile et charbon actif) puis ajoutez le substrat principal (terreau, feuilles mortes, bois en décomposition). Humidifiez légèrement le substrat pour créer un environnement favorable.
    \item \textbf{Plantes} : Plantez les végétaux en premier. Disposez les fougères miniatures, la mousse, les petites plantes couvre-sol, etc. de manière à créer un environnement naturel et harmonieux. Assurez-vous que le sol est bien stabilisé et humide avant d'aller plus loin.
    \item \textbf{Laissez reposer} : Attendez au moins une à deux semaines pour permettre aux plantes de s'enraciner et pour que le microclimat s'établisse dans le terrarium. Pendant ce temps, surveillez le niveau d'humidité et ajustez si nécessaire.
\end{itemize}

\subsection*{2. Introduction des décomposeurs (après 1 à 2 semaines)}
\begin{itemize}
    \item \textbf{Cloportes, collemboles et lombrics} : Une fois les plantes bien enracinées, introduisez les décomposeurs.
    \begin{itemize}
        \item \textbf{Cloportes} : Ils consommeront les matières végétales en décomposition, contribuant à recycler les nutriments.
        \item \textbf{Collemboles} : Ils se nourrissent de moisissures et de matières organiques en décomposition, aidant à maintenir un équilibre microbien sain.
        \item \textbf{Lombrics} : Ils aéreront le substrat et décomposeront les matières organiques. Introduisez une petite population pour commencer et laissez-les se stabiliser.
    \end{itemize}
\item \textbf{Observation} : Laissez les décomposeurs agir et s'établir dans le terrarium pendant une ou deux semaines. Cela leur permettra de se nourrir de la matière organique disponible et de commencer à remplir leur rôle.
\end{itemize}

\subsection*{3. Introduction des prédateurs (après 3 à 4 semaines)}
\begin{itemize}
    \item \textbf{Araignées ou petits prédateurs (ex. pseudoscorpions)} : Une fois que les populations de décomposeurs sont établies et semblent prospérer, introduisez les prédateurs en petites quantités.
    \begin{itemize}
        \item \textbf{Choisissez soigneusement les prédateurs} pour ne pas perturber trop brusquement l'équilibre ; il ne faut pas qu'ils éliminent la population de décomposeurs rapidement.
        \item \textbf{Suivez l'évolution} : Observez les interactions entre les prédateurs et les décomposeurs. Si la population de prédateurs diminue trop rapidement ou devient trop nombreuse, ajustez leur nombre en ajoutant ou en retirant des spécimens.
    \end{itemize}
\end{itemize}

\section{Conseils pour maintenir l'équilibre}
\begin{itemize}
    \item \textbf{Niveau d'humidité} : Gardez une bonne humidité en pulvérisant légèrement le terrarium, surtout après l'introduction de chaque groupe. Évitez les excès d'eau qui pourraient noyer les décomposeurs ou favoriser les moisissures indésirables.
    \item \textbf{Observation régulière} : Surveillez l'état des plantes et des animaux. Si les plantes semblent souffrir ou si certaines populations d'insectes augmentent ou diminuent trop, vous devrez ajuster l'écosystème (ajout d'éléments organiques, retrait de prédateurs, etc.).
    \item \textbf{Alimentation minimale} : Ajoutez de petites quantités de matière organique (feuilles mortes) si les décomposeurs semblent manquer de nourriture.
\end{itemize}

\section{Résumé des étapes d'introduction}
\begin{itemize}
    \item \textbf{Jour 1-2} : Préparation du substrat et plantation.
    \item \textbf{Après 1-2 semaines} : Introduction des décomposeurs (cloportes, collemboles, lombrics).
    \item \textbf{Après 3-4 semaines} : Introduction des prédateurs (araignées, pseudoscorpions, etc.).
\end{itemize}

Cette progression permet à chaque composant de s'adapter et de trouver sa place dans l'écosystème, garantissant une meilleure stabilité à long terme.

\newpage

\section{Pour appronfondir}

\subsection{Contrôler la moisissure}
Certains insectes et micro-organismes peuvent aider à contrôler la moisissure dans un terrarium. En voici quelques-uns qui sont couramment utilisés pour réguler la croissance des moisissures dans les écosystèmes fermés :
\ben
    \item Collemboles (Springtails) :  Ce sont de minuscules arthropodes qui se nourrissent principalement de matière organique en décomposition, y compris les champignons et les moisissures. Ils sont extrêmement efficaces pour maintenir un environnement sain en contrôlant les excès de moisissure. Leur petite taille leur permet de se déplacer facilement à travers le substrat, aidant à aérer le sol et à réguler l'équilibre microbien.
    \item Cloportes (Porcellionides) : Bien qu'ils soient principalement des décomposeurs, les cloportes peuvent également consommer certaines formes de matière organique en décomposition, comme les champignons. Leur présence peut contribuer à limiter les excès de moisissure, bien qu'ils ne soient pas aussi spécialisés que les collemboles pour cette tâche.
    \item Acariens bénéfiques : Certains types d'acariens peuvent aider à réguler les moisissures et les débris organiques en décomposition. Contrairement aux acariens nuisibles, les acariens bénéfiques ne nuisent pas aux plantes et peuvent contribuer au maintien de l'équilibre microbien dans le terrarium.
    \item Petites fourmis (à petite échelle et selon la compatibilité avec votre écosystème) : Bien que cela soit moins courant dans les terrariums, certaines espèces de fourmis peuvent jouer un rôle dans le contrôle des champignons en nettoyant la matière organique en décomposition. Cependant, leur introduction nécessite une grande précaution, car elles peuvent devenir envahissantes.
\een
\subsubsection{Autres conseils pour contrôler la moisissure :}
\ben
    \item Aération : Assurez une bonne circulation de l'air dans votre terrarium, ce qui peut réduire la prolifération de moisissures excessives.
    \item Niveau d'humidité : Évitez les excès d'humidité, car un environnement trop humide favorise les moisissures. Maintenez un juste milieu.
    \item Retrait manuel : Si des moisissures apparaissent en grande quantité, vous pouvez les retirer manuellement pour limiter leur expansion.
\een

\newpage

\section*{Terrarium Forestier Équilibré plus complexe}

Pour créer un terrarium équilibré avec des escargots, grillons, limaces, un serpent de terre (\textit{Carphophis amoenus}), des fourmis, des myriapodes noirs, des scorpions (\textit{Cercophonius squama}) et des opilions, il est important de suivre une progression étape par étape afin d'assurer l'équilibre écologique du système.

\section*{1. Préparation du Terrarium (Jour 1-2)}
\begin{itemize}
    \item \textbf{Substrat et végétation} : Préparez le substrat avec des couches de drainage (billes d'argile et charbon actif) et du terreau forestier riche en humus, avec des feuilles mortes. Plantez les végétaux pour fournir des cachettes et un environnement adapté aux espèces.
    \item \textbf{Attente et stabilisation (1-2 semaines)} : Laissez les plantes s'enraciner et le microclimat se stabiliser avant d'introduire les animaux.
\end{itemize}

\section*{2. Introduction des Organismes et Rôles}

\subsection*{2.1 Décomposeurs et Herbivores (après 1 à 2 semaines)}
\begin{itemize}
    \item \textbf{Escargots} :
    \begin{itemize}
        \item \textbf{Rôle} : Consommateurs de matières végétales en décomposition, d'algues et de certaines plantes tendres.
        \item \textbf{Moment d'introduction} : Après que le substrat est stable et les plantes enracinées.
        \item \textbf{Remarque} : Nécessitent un environnement humide. Contrôlez leur population pour éviter une consommation excessive de végétation.
    \end{itemize}
    \item \textbf{Limaces} :
    \begin{itemize}
        \item \textbf{Rôle} : Similaires aux escargots, elles consomment de la matière végétale, d'algues et de déchets organiques.
        \item \textbf{Moment d'introduction} : En même temps que les escargots.
        \item \textbf{Remarque} : Elles peuvent devenir problématiques si elles consomment trop de végétation.
    \end{itemize}
    \item \textbf{Myriapodes noirs (mille-pattes)} :
    \begin{itemize}
        \item \textbf{Rôle} : Décomposeurs qui consomment la matière organique en décomposition, participant à l'aération du sol et à la libération de nutriments.
        \item \textbf{Moment d'introduction} : En même temps que les escargots et les limaces.
        \item \textbf{Remarque} : Ils sont pacifiques et jouent un rôle clé dans la décomposition.
    \end{itemize}
\end{itemize}

\subsection*{2.2 Contrôle des Populations et Prédateurs (après 3 à 4 semaines)}
\begin{itemize}
    \item \textbf{Grillons} :
    \begin{itemize}
        \item \textbf{Rôle} : Omnivores, ils consomment des matières végétales, des insectes morts ou de la nourriture disponible, et peuvent servir de proies pour les prédateurs.
        \item \textbf{Moment d'introduction} : Après l'introduction des décomposeurs.
        \item \textbf{Remarque} : Leur population doit être contrôlée, car ils peuvent devenir nombreux et bruyants.
    \end{itemize}
    \item \textbf{Fourmis} :
    \begin{itemize}
        \item \textbf{Rôle} : Nettoyeuses et prédatrices, elles consomment des restes de nourriture, des insectes morts, et peuvent prédater de petits organismes.
        \item \textbf{Moment d'introduction} : Avec précaution, après l'introduction des herbivores et des décomposeurs.
        \item \textbf{Remarque} : Certaines espèces de fourmis peuvent devenir envahissantes ou agressives et déséquilibrer l'écosystème.
    \end{itemize}
    \item \textbf{Opilions (faucheux)} :
    \begin{itemize}
        \item \textbf{Rôle} : Prédateurs généralistes qui consomment de petits insectes, des matières organiques et de petites proies.
        \item \textbf{Moment d'introduction} : Une fois les populations de grillons et de petits insectes stabilisées.
        \item \textbf{Remarque} : Ils ne posent généralement pas de danger pour les autres animaux du terrarium et sont efficaces pour réguler les populations de petits insectes.
    \end{itemize}
    \item \textbf{Scorpions (Cercophonius squama)} :
    \begin{itemize}
        \item \textbf{Rôle} : Prédateurs qui chassent de petits insectes, des grillons et d'autres petits organismes.
        \item \textbf{Moment d'introduction} : Une fois que la population de proies (grillons, petits insectes) est établie.
        \item \textbf{Remarque} : Le contrôle de leur population est essentiel pour éviter une déséquilibre.
    \end{itemize}
\end{itemize}

\subsection*{2.3 Prédateurs Supérieurs (après 4 à 5 semaines)}
\begin{itemize}
    \item \textbf{Carphophis amoenus (serpent de terre)} :
    \begin{itemize}
        \item \textbf{Rôle} : Se nourrit principalement de petits invertébrés (vers, larves d'insectes). Aide à réguler certaines populations de proies.
        \item \textbf{Moment d'introduction} : Une fois que les populations de petits invertébrés sont stables.
        \item \textbf{Remarque} : Surveillez l'équilibre entre les proies et leur prédateur.
    \end{itemize}
\end{itemize}

\section*{3. Points de Vigilance et Conseils pour l'Équilibre}
\begin{itemize}
    \item \textbf{Niveau d'humidité} : Maintenez une humidité adaptée aux besoins des différentes espèces.
    \item \textbf{Observation régulière} : Surveillez les interactions entre les espèces et ajustez les populations si nécessaire.
    \item \textbf{Ajustement de la nourriture} : Ajoutez des matières organiques (feuilles mortes, fruits, légumes) pour les décomposeurs si nécessaire.
\end{itemize}

\section*{Résumé des Étapes d'Introduction}
\begin{itemize}
    \item \textbf{Jour 1-2} : Préparation du substrat et des plantes.
    \item \textbf{Après 1-2 semaines} : Introduction des escargots, limaces, myriapodes.
    \item \textbf{Après 3-4 semaines} : Introduction des grillons, fourmis, opilions.
    \item \textbf{Après 4-5 semaines} : Introduction du Cercophonius squama (scorpion) et du Carphophis amoenus (serpent de terre).
\end{itemize}
\end{document}

